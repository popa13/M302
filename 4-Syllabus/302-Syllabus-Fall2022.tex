\documentclass[12pt]{amsart}

\usepackage{xcolor}
\usepackage[hidelinks]{hyperref}
\usepackage[margin=1.1in, tmargin=1in, bmargin=1in]{geometry}
\usepackage{footmisc}
\usepackage{multicol}

\pagestyle{myheadings}
%\markleft{Syllabus for Math 331, Spring 2020}
%\markright{Syllabus for Math 331, Spring 2020}

\newcommand{\spacer}{\vspace{.2cm}}
\newcommand{\svs}{\vspace{.1cm}}

\newcommand{\red}[1]{\textcolor{red}{#1}}
\definecolor{gold}{rgb}{0.80,0.68,0.00}\newcommand{\gold}[1]{\textcolor{gold}{#1}}

\begin{document}

\begin{center}
\textsc{Math 302} \hfill {\Large\textsc{Syllabus}} \hfill \textsc{Fall 2022}
\end{center}

\noindent\hrulefill

\noindent\textbf{Lecture Sections 02} \hfill \textbf{Room} \\
\noindent TR 12:00--1:15pm \hfill KUY 210

\noindent\hrulefill

\spacer

\noindent\textbf{Instructor:} Pierre-Olivier Paris{\'e} (email: \texttt{parisepo@hawaii.edu})\\
Office: Physical Science Building (PSB) 302\\
%Zoom Office: 976 6163 4762 (ID), 359469 (Passcode)\\
Office hours: W 1:00--3:00pm \& R 2:30--4:30pm

\noindent\hrulefill

\section*{Course description}
First order ordinary differential equations, constant coefficient linear equations, oscillations, Laplace transform, convolution, Green’s function.\svs

\section*{Course material}

\noindent{\bf Textook:} Trench, William F., \emph{Elementary Differential Equations with Boundary Value Problems} (2013). Free textbook available through: \url{https://digitalcommons.trinity.edu/mono/9/}  

\noindent{\bf Course website:} \url{https://mathopo.ca/courses-website/math-302/math-302}. All the information about the course (like the schedule, lecture notes, homework solutions) is posted on the course website.

\noindent{\bf Laulima:} The Laulima course website will also be used. This is where the homework assignments and your grades will be posted. The course website will also be embeded in the Laulima course website.

\section*{Important Dates}
%Please make sure you write down somewhere these important dates:
Refer to the google calendar on the course website for a detailed schedule.
	\begin{itemize}
	\item Midterms:
		\begin{itemize}
		\item[--] Midterm 1 on September 27th.
		\item[--] Midterm 2 on November 1st.
		\end{itemize}
	\item Final: -- December 15th, 12:00--2:00pm.
	\item Non-instructional day(s):
		\begin{itemize}
		\item[--] Election Day, November 8th.
		\item[--] Thanksgiving, November 24th.
		\item[--] Study Day, December 8th.
		\end{itemize}
	\end{itemize}

\section*{Grading components}
Your course average will be determined by a weighted average of the components below.
 %the homework on Gradescope (\texttt{http://gradescope.com}, entry code \textbf{BPNBW7}).

\begin{enumerate}
\item {\bf Midterm exams (50\%):} There will be two(2) midterms exams. Each exam will have a duration of 75min and will be in-person (the situation may change due to the COVID pandemic). No extra time will be allowed. Midterms are not cummulative and are in-person. 
\item {\bf Final exam (25\%):} There will be a commun final exam as scheduled by the university on December, 15th, 12:00--2:00pm. This exam is in class.
\item{\bf Homework (25\%):} There will be homework each week assigned on each Tuesdays (starting on August, 23th and due for the next Tuesday, before 11:59pm). The homework problems will be posted on Laulima, in Assignments. You will be required to scan and upload your solutions to the homework problems on Laulima. The solutions will be posted on the course website, in the homework webpage. The best 10 scores will be used to compute your final score for this component.%If the guideline for a homework is not followed, at most 5\% will be retrieve from this homework total points. The guideline is available in the Homework webpage of the course website.
\end{enumerate}


%\begin{table}[h]
%\begin{tabular}{c|c|c}
%Evaluation & Number & \% average \\ \hline\hline
%Midterms & 2 & 40\% \\\hline
%Final & 1 & 30\% \\\hline
%Homework & 12 & 15\% \\\hline
%Worksheets & 12 & 15\% \\\hline\hline
%Total & 27 & 100\%
%\end{tabular}
%\end{table}

\section*{Lectures}
If you miss a lecture or recital, you are responsible for any assignments and/or announcements made. Unavoidable absences should be explained to the instructor. 

The lectures will also be recorded and will be available for students in the Google Calendar after each class. The lectures won't be streaming online.  

\section*{Missed assignment policies}
\noindent{\bf Policies for exams:}  Attendance on the exams is compulsory; otherwise, a grade of zero will be recorded. Any student who has an excused, documented conflict with a test time must inform their instructor \textbf{within the two weeks prior to the midterm targeted} when possible.  Late requests will either be denied or will result in an automatic deduction from the exam score. An absence due to a positive COVID-19 test result must be notified to the instructor with a doctor note and is not subject to the previous 2 weeks rule.

For those students with an excused absence for a midterm, there will be a make-up exam. \textbf{The final exam cannot be taken before the scheduled time for any reason.}

Conflicts arising from work or social obligations, or from personal travel plans do \textbf{not} qualify as excused absences. By registering for this course, you are agreeing to take all exams at the scheduled times.

\svs

\noindent{\bf Policies for homework:} No late homework will be accepted. The mark zero(0) will be attributed to a late homework. For each homework, there will be 5\% of the total point for respecting the template.
\svs

\noindent{\bf Academic integrity:}
All students are expected to abide by the university's Conduct Code. Academic sanctions for dishonesty may include receiving an F in the assignment or receiving an F in the class. There may be additional administrative sanctions, see
\newline {\url{https://www.hawaii.edu/policy/index.php?action=home&policySection=ep}}
\svs

\noindent{\bf Academic Expectations:} You should be familiar with the  academic expectations for UH Math courses outlined here: 
\newline \url{http://www.math.hawaii.edu/~dale/Expectations.html}.

\section*{Concerns}
If at any time during the semester you have any questions or concerns about the class, please contact me during regularly scheduled office hours or via email to make an appointment. You may also contact the following people:
\spacer

\begin{multicols}{2}
\noindent {\bf Director of Undergraduate Studies}\\
Mirjana Jovovic \\
\texttt{undergrad-dir@math.hawaii.edu}

\begin{flushright}
\noindent {\bf Associate Chair}\\
Bj{\o}rn Kjos-Hanssen \\
\texttt{assoc-chair@math.hawaii.edu}
\end{flushright}
\end{multicols}

\end{document}


\documentclass[12pt,a4paper]{article}
\usepackage[utf8]{inputenc}
\usepackage[english]{babel}

\usepackage{amsmath}
\usepackage{amsfonts}
\usepackage{amssymb}

\usepackage{graphicx}
\usepackage{caption}
\usepackage{subcaption}
\usepackage{lmodern}
\usepackage{tikz}
\usepackage{titlesec}
\usepackage{environ}
\usepackage{xcolor}
\usepackage{fancyhdr}
\usepackage[colorlinks = true, linkcolor = black]{hyperref}
\usepackage{xparse}
\usepackage{enumitem}
\usepackage{comment}
\usepackage{wrapfig}
\usepackage[capitalise]{cleveref}

\usepackage[left=2cm,right=2cm,top=2cm,bottom=2cm]{geometry}
\usepackage{multicol}
\usepackage[indent=0pt]{parskip}

\newcommand{\spaceP}{\vspace*{0.5cm}}
\newcommand{\Span}{\mathrm{Span}\,}
\newcommand{\range}{\mathrm{range}\,}
\newcommand{\ra}{\rightarrow}

%% Redefining sections
\newcommand{\sectionformat}[1]{%
    \begin{tikzpicture}[baseline=(title.base)]
        \node[rectangle, draw] (title) {#1};
    \end{tikzpicture}
    
    \noindent\hrulefill
}

\newif\ifhNotes 

\hNotesfalse

\ifhNotes
	\newcommand{\hideNotes}[1]{%
	\phantom{#1}
	}
	\newcommand{\hideNotesU}[1]{%
	\underline{\hspace{1mm}\phantom{#1}\hspace{1mm}}
	}
\else
	\newcommand{\hideNotes}[1]{#1}
	\newcommand{\hideNotesU}[1]{\textcolor{blue}{#1}}
\fi

% default values copied from titlesec documentation page 23
% parameters of \titleformat command are explained on page 4
\titleformat%
    {\section}% <command> is the sectioning command to be redefined, i. e., \part, \chapter, \section, \subsection, \subsubsection, \paragraph or \subparagraph.
    {\normalfont\large\scshape}% <format>
    {}% <label> the number
    {0em}% <sep> length. horizontal separation between label and title body
    {\centering\sectionformat}% code preceding the title body  (title body is taken as argument)

%% Set counters for sections to none
\setcounter{secnumdepth}{0}

%% Set the footer/headers
\pagestyle{fancy}
\fancyhf{}
\renewcommand{\headrulewidth}{0pt}
\renewcommand{\footrulewidth}{2pt}
\lfoot{P.-O. Paris{\'e}}
\cfoot{MATH 302}
\rfoot{Page \thepage}

%% Defining example environment
\newcounter{example}[section]
\NewEnviron{example}%
	{%
	\noindent\refstepcounter{example}\fcolorbox{gray!40}{gray!40}{\textsc{\textcolor{red}{Example~\theexample.}}}%
	%\fcolorbox{black}{white}%
		{  %\parbox{0.95\textwidth}%
			{
			\BODY
			}%
		}%
	}

% Theorem environment
\NewEnviron{theorem}%
	{%
	\noindent\refstepcounter{example}\fcolorbox{gray!40}{gray!40}{\textsc{\textcolor{blue}{Theorem~\theexample.}}}%
	%\fcolorbox{black}{white}%
		{  %\parbox{0.95\textwidth}%
			{
			\BODY
			}%
		}%
	}

\NewEnviron{notes}%
	{%
	\noindent \fcolorbox{gray!40}{gray!40}{\textsc{\textcolor{blue}{Solution.}}}%
	%\fcolorbox{black}{white}%
		{  %\parbox{0.95\textwidth}%
			{
			\textcolor{blue}{%
			\BODY
			}
			}%
		}%
	}
%%% Ignorer les notes
\excludecomment{notes}

%%%%
\begin{document}
\thispagestyle{empty}

\begin{center}
\vspace*{2.5cm}

{\Huge \textsc{Math 302}}

\vspace*{2cm}

{\LARGE \textsc{Chapter 8}} 

\vspace*{0.75cm}

\noindent\textsc{Section 8.4: Convolution}

\vspace*{0.75cm}

\tableofcontents

\vfill

\noindent \textsc{Created by: Pierre-Olivier Paris{\'e}} \\
\textsc{Fall 2022}
\end{center}

\newpage

\section{What Does The Word ``Convolution'' Mean?}

\subsection{The Story of The Matches}

	\begin{itemize}
	\item Suppose we have a number of matches we need to light.
	\item At each second, so at $t = 0$, $t = 1$, $t = 2$, $t = 3$, $...$, $t = n$, we light a certain number of matches. Denote by $f(t)$ the number of matches lit at time $t$.
	\item Each matches give off smoke. Denote by $g(t)$ the smoke produced by a match after $t$ seconds.
	\end{itemize}
	
	\begin{figure}[ht]
	\centering
	\begin{tikzpicture}
	\begin{scope}[scale=2]
	\draw[brown, fill=brown] (-3, 0) -- (-3,2) -- (-2.7, 2) -- (-2.7, 0) -- (-3, 0);
	\draw[red, fill=red] (-3.1, 2) -- (-2.6, 2) -- (-2.6, 2.4) -- (-3.1, 2.4) -- (-3.1, 2);
	\draw[black, fill=black] (-2, 0.75) -- (-2, 1.25) -- (-1, 1.25) -- (-1, 1.5) -- (-.5, 1) -- (-1,0.5) -- (-1, 0.75) -- (-2, 0.75);
	\draw[black] (-1.25, 1.75)node{light match};
	\foreach \x in {3.4,3.5,3.6, 3.7, 3.8, 3.9}{%
		\begin{scope}[shift={(\x, 0)}]
		\draw[black!50] (-2.95, 2.75) ..controls +(-0.25,0.25) and +(0.25,-0.25) .. (-2.95, 3.25);
		\end{scope}}
	\begin{scope}[shift={(3.5, 0)}]
	\draw[brown, fill=brown] (-3, 0) -- (-3,2) -- (-2.7, 2) -- (-2.7, 0) -- (-3, 0);
	\draw[red, fill=red] (-2.85, 2) ..controls +(-0.5, 0) and +(-0.5,-0.25) .. (-2.85,2.7) ..controls +(0,-0.25) and +(0.75, 0) .. (-2.85, 2);
	\draw[black, ->, >=latex, very thick] (-1.5, 3)node[right]{$g(t)$} -- (-2, 3);
	\draw[black, ->, >=latex, very thick] (-1.5, 1.5)node[right]{$f(t)$} -- (-2, 1.5);
	\end{scope}
	\begin{scope}[shift={(2.05, 1)}, scale=0.5]
	\draw[orange, fill=orange] (-2.85, 2) ..controls +(-0.5, 0) and +(-0.5,-0.25) .. (-2.85,2.7) ..controls +(0,-0.25) and +(0.75, 0) .. (-2.85, 2);
	\end{scope}
	\end{scope}
	\end{tikzpicture}
	\caption{The Matches Problem}
	\end{figure}
	
\vspace*{16pt}
	
\underline{Question:} What is the total quantity of smoke in the air after a certain time $t$?

\vspace*{12pt}

\begin{center}
\begin{tabular}{c|c}
Times ($t$) & $Q(t)$ \\\hline\hline
\phantom{2} & \phantom{22222222222222222222222222222222222222} \\[12pt]\hline
\phantom{2} & \phantom{22222222222222222222222222222222222222} \\[12pt]\hline
\phantom{2} & \phantom{22222222222222222222222222222222222222} \\[12pt]\hline
\phantom{2} &  \phantom{22222222222222222222222222222222222222} \\[12pt]
\end{tabular}
\end{center}

\vspace*{12pt}

The total contribution of the matches after $n$ seconds:
	\begin{align*}
	Q(t) = \phantom{2222222222222222222222222222222222222222222222222222222222222222} 
	\end{align*}
	
\vspace*{16pt}

\underline{What if we have a continuous phenomena?}

\newpage

\section{Convolution And Laplace Transform}

\subsection{Definition}
The convolution of a function $f(t)$ with another function $g(t)$ is the new function $(f \ast g) (t)$ defined by
	\begin{align*}
	(f \ast g) (t) = \int_0^t f(x) g(t - x) \, dx .
	\end{align*}
	
\vspace*{16pt}

\begin{example}
Let 
	\begin{align*}
	f(t) = u (t) - u(t - 1) \quad \text{ and } \quad g(t) = u (t) - u(t - 1) .
	\end{align*}
Compute $f \ast g$.
\end{example}

\vfill

\newpage

\phantom{2}

\vfill

\underline{Desmos:} \url{https://www.desmos.com/calculator/h50sct4xeq}

\newpage

\subsection{Laplace Transform}

The nice properties of the convolution is a direct connection with the Laplace transform.

\vspace*{16pt}

	\begin{example}
	Let $f(t) = e^t$ and $g(t) = e^{-t}$. 
		\begin{enumerate}[label=\textbf{(\alph*)}]
		\item Compute $f \ast g$.
		\item Find $L (f \ast g )$.
		\item Compare with $L(f) L(g)$.
		\end{enumerate}
	\end{example}
	
\vfill

\underline{Tranform of Convolution:} If 
	\begin{itemize}
	\item $f(t)$ is a function with Laplace transform $F(s)$;
	\item $g(t)$ is a function with Laplace transform $G(s)$;
	\end{itemize}
then
	\begin{align*}
	L(f \ast g ) = L(f) L(g) = F(s) G(s) .
	\end{align*}
	
	\newpage
	
	\begin{example}
	Find the inverse Laplace transform of the following function:
		%\begin{enumerate}[label=\textbf{(\alph*)}]
		%\item 
		$$\displaystyle \frac{1}{s^2 (s^2 + 4)}.$$
		%\item $\displaystyle \frac{s (s + 3)}{(s^2 + 4) (s^2 + 6s + 10)}$.
		%\end{enumerate}
	\end{example}
	
	\newpage
	
	\begin{comment}
	\section{Applications to ODE}
	
	\begin{example}
	Find the solution $y(t)$ to the following IVP:
		\begin{align*}
		y'' + 3y' + y = f(t) , \quad y(0) = 0,\, y' (0) = 0 .
		\end{align*}
	\end{example}
	
	\newpage
	
	\phantom{2}
	
	\vfill
	
	\underline{General Convolution Formula:}
	The solution $y(t)$ to the following IVP
		\begin{align*}
		ay'' + by' + cy = f(t) , \quad y(0) = k_0 , \, y'(0)= k_1
		\end{align*}
	is
		\begin{align*}
		y (t) = k_0 y_1 (t) + k_1 y_2 (t) + (w \ast f) (t)
		\end{align*}
	where
		\begin{itemize}
		\item $y_1$ is a solution to the following IVP
			\begin{align*}
			ay_1'' + by_1'' + cy_1 = 0, \quad y_1 (0) = 1, \, y_1'(0) = 0 ;
			\end{align*}
		\item $y_2$ is a solution to the following IVP
			\begin{align*}
			ay_2'' + by_2'' + cy_2 = 0 , \quad y_2 (0) = 0 , \, y_2' (0) = 1 ;
			\end{align*}
		\item $w(t)$ satisfies
			\begin{align*}
			w(t) = \frac{1}{a} y_2 (t) .
			\end{align*}
		\end{itemize}
		
	\newpage
	\end{comment}
	
	\section{Laplace Transforms of Integrals}
	As a special case of the Laplace transform of a convolution, we can take the Laplace transform of an integral.
	
	\vspace*{16pt}
	
	\begin{example}
	Suppose $f$ has a Laplace transform given by $F(s)$. Find the Laplace transform of
		\begin{align*}
		h(t) = \int_0^t f(x) \, dx .
		\end{align*}
	\end{example}
	
	\vfill
	
	\underline{Other related results:}
		\begin{itemize}
		\item For $g(t) = \displaystyle \int_0^t \int_0^{x} f(u) \, du \, dx$, we have $G(s) = F(s)/s^2$.
		\item For a function $g(t)$ given as three integrals, then $G(s) = F(s)/s^3$.
		\item For a function $g(t)$ given as $n$ integrals, then $G(s) = F(s)/s^n$.
		\end{itemize}
	
	\newpage
	
	\section{Integro-Differential Equations}
	We can solve more than just an ODE!
	
	\vspace*{16pt}
	
	\begin{example}
	Find the solution to the following integro-differential equation
		\begin{align*}
		\int_0^t y(u) \, du + y'(t) = t,
		\end{align*}
	where $y(0) = 0$.
	\end{example}
	
	\newpage
	
	\begin{example}
	Find the general solution to the following integral equation
		\begin{align*}
		y(t) = \sin (t) - 2 \int_0^t y(u) \cos (t - u) \, du .
		\end{align*}
	\end{example}
	
	
\end{document}
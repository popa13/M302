\documentclass[12pt]{article}
%\usetheme{CambridgeUS}%CambridgeUS
\usepackage[english]{babel}
\usepackage[utf8]{inputenc}
%\frenchbsetup{StandardLists=true}

\usepackage{enumitem}
\usepackage[margin=1.5cm]{geometry}

\usepackage{amsmath, amsfonts, amssymb}
\usepackage{graphicx}
\usepackage{tikz}
\usepackage{pgfplots}
\usepackage{multicol}
\usepackage{subfigure}
\usepackage{arydshln}

\usepackage{comment}
\usepackage[nottoc]{tocbibind}
%\usepackage{array}      % Pour les tables
\usepackage{caption} 
\usepackage{setspace}
\usepackage{marvosym}
\usepackage[framemethod=tikz]{mdframed}
\usetikzlibrary{arrows, shapes.geometric, decorations.markings, decorations.shapes, decorations.pathreplacing,decorations.text, shadows, matrix, decorations.pathmorphing, patterns, shadings, backgrounds}

\usepackage[thmmarks]{ntheorem}

\newif\ifhNotes 

\hNotesfalse

\ifhNotes
	\newcommand{\hideNotes}[1]{%
	\phantom{#1}
	}
	\newcommand{\hideNotesU}[1]{%
	\underline{\hspace{1mm}\phantom{#1}\hspace{1mm}}
	}
\else
	\newcommand{\hideNotes}[1]{#1}
	\newcommand{\hideNotesU}[1]{\textcolor{blue}{#1}}
\fi

% MATH commands
\newcommand{\bC}{\mathbb{C}}
\newcommand{\bR}{\mathbb{R}}
\newcommand{\bN}{\mathbb{N}}
\newcommand{\bZ}{\mathbb{Z}}
\newcommand{\bT}{\mathbb{T}}
\newcommand{\bD}{\mathbb{D}}

\newcommand{\cL}{\mathcal{L}}
\newcommand{\cM}{\mathcal{M}}
\newcommand{\cP}{\mathcal{P}}
\newcommand{\cH}{\mathcal{H}}
\newcommand{\cB}{\mathcal{B}}
\newcommand{\cK}{\mathcal{K}}
\newcommand{\cJ}{\mathcal{J}}
\newcommand{\cU}{\mathcal{U}}
\newcommand{\cO}{\mathcal{O}}
\newcommand{\cA}{\mathcal{A}}
\newcommand{\cC}{\mathcal{C}}

\newcommand{\fK}{\mathfrak{K}}
\newcommand{\fM}{\mathfrak{M}}

\newcommand{\ga}{\left\langle}
\newcommand{\da}{\right\rangle}
\newcommand{\oa}{\left\lbrace}
\newcommand{\fa}{\right\rbrace}
\newcommand{\oc}{\left[}
\newcommand{\fc}{\right]}
\newcommand{\op}{\left(}
\newcommand{\fp}{\right)}

\newcommand{\ra}{\rightarrow}
\newcommand{\Ra}{\Rightarrow}

\renewcommand{\Re}{\mathrm{Re}\,}
\renewcommand{\Im}{\mathrm{Im}\,}
\newcommand{\Arg}{\mathrm{Arg}\,}
\newcommand{\sech}{\mathrm{sech}\,}
\newcommand{\csch}{\mathrm{csch}\,}
\newcommand{\Log}{\mathrm{Log}\,}
\newcommand{\cis}{\mathrm{cis}\,}

\newcommand{\ran}{\mathrm{ran}\,}
\newcommand{\bi}{\mathbf{i}}
\newcommand{\Sp}{\mathrm{span}\,}
\newcommand{\Inv}{\mathrm{Inv}\,}
\newcommand\smallO{
  \mathchoice
    {{\scriptstyle\mathcal{O}}}% \displaystyle
    {{\scriptstyle\mathcal{O}}}% \textstyle
    {{\scriptscriptstyle\mathcal{O}}}% \scriptstyle
    {\scalebox{.7}{$\scriptscriptstyle\mathcal{O}$}}%\scriptscriptstyle
  }
\newcommand{\HOL}{\mathrm{Hol}}
\newcommand{\cl}{\mathrm{clos}}
\newcommand{\ve}{\varepsilon}

\newcommand{\Arccos}{\mathrm{Arc}\,\mathrm{cos}\,}
\newcommand{\Arcsin}{\mathrm{Arc}\,\mathrm{sin}\,}
\newcommand{\Arctan}{\mathrm{Arc}\,\mathrm{tan}\,}
\newcommand{\Arccsc}{\mathrm{Arc}\,\mathrm{csc}\,}
\newcommand{\Arcsec}{\mathrm{Arc}\,\mathrm{sec}\,}
\newcommand{\Arccot}{\mathrm{Arc}\,\mathrm{cot}\,}
\newcommand{\dst}{\displaystyle}


\tikzstyle{myboxT} = [draw=black, fill=black!0,line width = 1pt,
    rectangle, rounded corners = 0pt, inner sep=8pt, inner ysep=8pt]
    
\newcommand{\MyC}[1]{\begin{tikzpicture}
\node (boxIntro) at (0,0) {};
\node [myboxT](Intro) at (boxIntro){%
	\begin{minipage}{0.9\textwidth}
	#1
	\end{minipage}};
\end{tikzpicture}}

%%%%  Environnement exer et solutionnaire
{\theorembodyfont{}
\theoremstyle{plain}
\theoremseparator{\textbf{.}}
\theoremsymbol{}
\newtheorem{exer}{\textbf{Exercice}}}

{\theorembodyfont{\color{blue}}
\theoremstyle{plain}
\theoremseparator{\textbf{:}}
\theoremsymbol{$\square$}
\newtheorem*{sol}{\textbf{Solution}}}

\renewcommand*{\theexer}{\arabic{exer}}
\renewcommand*{\thesol}{\arabic{sol}}

\tikzstyle{myboxT} = [draw=black, fill=black!0,line width = 1pt,
    rectangle, rounded corners = 0pt, inner sep=8pt, inner ysep=8pt]

\author{Pierre-Olivier Parisé \\ Mathématiques de l'ingénieur III (MAT-2900)}
\title{Table de transformées de Laplace}
\date{Université Laval \\ Québec, Canada\\
Automne 2020}

\pagestyle{empty}

\newcommand{\spc}{\vspace*{0.5cm}}

\begin{document}
\noindent \hrulefill \\
	MATH-241 \hfill Pierre-Olivier Paris{\'e}\\
	Table Laplace Transforms \hfill Fall 2022\\\vspace*{-0.75cm}
	
	\noindent\hrulefill
	
	\spc
	
\newcounter{colonneUne}
\newcounter{colonneDeux}
\setcounter{colonneDeux}{15}
\newcommand{\noFormule}{%
	\stepcounter{colonneUne}\arabic{colonneUne}.}
\newcommand{\noFormuleDeux}{%
	\stepcounter{colonneDeux}\arabic{colonneDeux}.}
\begin{table}[ht!]
\centering
{\renewcommand{\arraystretch}{2.49}
	\setlength{\tabcolsep}{2pt}
\begin{tabular}{|c|c||c|c|}
\hline
\hspace{0.75cm} \textbf{Function} \hspace{0.75cm} & \hspace{0.75cm} \textbf{Transform} \hspace{0.75cm} & \hspace{0.75cm} \textbf{Function} \hspace{0.75cm}  & \hspace{0.75cm} \textbf{Transform} \hspace{0.75cm} \\\hline
\noFormule \hfill $1$ \hfill\hfill & $\hideNotes{\displaystyle\frac{1}{s}}$ & \noFormuleDeux \hfill $f'(t)$ \hfill\hfill & $\hideNotes{sF(s) - f(0)}$ \\\hline
\noFormule \hfill $t$ \hfill\hfill & $\hideNotes{\displaystyle\frac{1}{s^2}}$ & \noFormuleDeux \hfill $f''(t)$ \hfill\hfill & $\hideNotes{s^2F(s) - f(0)s - f'(0)}$ \\\hline
\noFormule \hfill $t^n$ ($n\geq 0$ entier) & $\hideNotes{\displaystyle\frac{n!}{s^{n+1}}}$ & \noFormuleDeux \hfill $f^{(3)}(t)$ \hfill\hfill & $\hideNotes{s^3 F(s) - s^2f(0) - sf'(0) - f''(0)}$ \\\hline
\noFormule \hfill $e^{at}$ \hfill \hfill & $\hideNotes{\displaystyle\frac{1}{s-a}}$ & \noFormuleDeux \hfill $f^{(n)}(t)$ \hfill\hfill & $\hideNotes{s^n F(s) - s^{n-1} f(0) - \cdots - f^{(n-1)}(0)}$ \\\hline
\noFormule \hfill $te^{at}$ \hfill \hfill & $\hideNotes{\displaystyle\frac{1}{(s-a)^2}}$ & \noFormuleDeux \hfill $t^n f(t)$ \hfill\hfill & $\hideNotes{(-1)^n \displaystyle \frac{d^n}{ds^n} F(s)}$ \\\hline
\noFormule \hfill $\sin (\omega t )$ \hfill \hfill & $\hideNotes{\displaystyle\frac{\omega}{s^2 + \omega^2}}$ & \noFormuleDeux \hfill $e^{at} f(t)$ \hfill\hfill & $\hideNotes{F(s - a)}$ \\\hline
\noFormule \hfill $\cos (\omega t )$ \hfill \hfill & $\hideNotes{\displaystyle \frac{s}{s^2 + \omega^2}}$ & \noFormuleDeux \hfill $\displaystyle \int_0^{t} f(\tau ) \, d\tau$ \hfill\hfill & $\hideNotes{\displaystyle \frac{F(s)}{s}}$ \\\hline
\noFormule \hfill $e^{at} \sin (\omega t )$ \hfill \hfill & $\hideNotes{\displaystyle \frac{\omega}{(s-a)^2 + \omega^2}}$ & \noFormuleDeux \hfill $\displaystyle \int_0^{t} \int_0^{\tau} f(\tilde{\tau}) \, d\tilde{\tau} \, d\tau$ \hfill\hfill & $\hideNotes{\displaystyle \frac{F(s)}{s^2}}$ \\\hline 
\noFormule \hfill $e^{at} \cos (\omega t )$ \hfill \hfill & $\hideNotes{\displaystyle \frac{(s - a)}{(s - a)^2 + \omega^2}}$ & \noFormuleDeux \hfill $u(t-a)f(t-a)$ \hfill\hfill & $\hideNotes{e^{-as} F(s)}$ \\\hline
\noFormule \hfill $t \sin (\omega t )$ \hfill \hfill & $\hideNotes{\displaystyle\frac{2\omega s}{(s^2 + \omega^2)^2}}$ & \noFormuleDeux \hfill $f(t) \delta (t-a)$ \hfill\hfill & $\hideNotes{f(a) e^{-as}}$ \\\hline
\noFormule \hfill $t \cos (\omega t )$ \hfill \hfill & $\hideNotes{\displaystyle\frac{s^2 - \omega^2}{(s^2 + \omega^2)^2}}$ & \noFormuleDeux \hfill $f(t) \ast g(t)$ \hfill\hfill & $\hideNotes{F(s)G(s)}$ \\\hline
\noFormule \hfill $\cosh (\omega t)$ \hfill \hfill & $\hideNotes{\displaystyle\frac{s}{s^2 - \omega^2}}$ & \noFormuleDeux \hfill $u(t)$ \hfill\hfill & $\hideNotes{\displaystyle\frac{1}{s}}$\\\hline
\noFormule \hfill $\sinh (\omega t)$ \hfill \hfill  & $\hideNotes{\displaystyle\frac{\omega}{s^2 - \omega^2}}$ & \noFormuleDeux \hfill $u(t-a)$ \hfill\hfill & $\hideNotes{\displaystyle\frac{e^{-as}}{s}}$\\\hline
\noFormule \hfill \hfill & & \noFormuleDeux \hfill $\delta (t)$ \hfill\hfill & \hideNotes{1} \\\hline
\noFormule \hfill \hfill & & \noFormuleDeux \hfill $\delta (t-a)$ \hfill\hfill & $\hideNotes{e^{-as}}$ \\\hline
\end{tabular}}
\caption{Table of Laplace Transforms}
\end{table}

\end{document}

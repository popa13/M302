\documentclass[12pt]{article}
\usepackage[utf8]{inputenc}

\usepackage{lmodern}

\usepackage{enumitem}
\usepackage[margin=2cm]{geometry}

\usepackage{amsmath, amsfonts, amssymb}
\usepackage{graphicx}
\usepackage{subfigure}
\usepackage{tikz}
\usepackage{pgfplots}
\usepackage{multicol}

\usepackage{comment}
\usepackage{url}
\usepackage{calc}
%\usepackage{subcaption}
\usepackage[indent=0pt]{parskip}

\usepackage{array}
\usepackage{blkarray,booktabs, bigstrut}

\pgfplotsset{compat=1.16}

% MATH commands
\newcommand{\ga}{\left\langle}
\newcommand{\da}{\right\rangle}
\newcommand{\oa}{\left\lbrace}
\newcommand{\fa}{\right\rbrace}
\newcommand{\oc}{\left[}
\newcommand{\fc}{\right]}
\newcommand{\op}{\left(}
\newcommand{\fp}{\right)}

\newcommand{\bi}{\mathbf{i}}
\newcommand{\bj}{\mathbf{j}}
\newcommand{\bk}{\mathbf{k}}
\newcommand{\bF}{\mathbf{F}}

\newcommand{\mR}{\mathbb{R}}

\newcommand{\ra}{\rightarrow}
\newcommand{\Ra}{\Rightarrow}

\newcommand{\sech}{\mathrm{sech}\,}
\newcommand{\csch}{\mathrm{csch}\,}
\newcommand{\curl}{\mathrm{curl}\,}
\newcommand{\dive}{\mathrm{div}\,}

\newcommand{\ve}{\varepsilon}
\newcommand{\spc}{\vspace*{0.5cm}}

\DeclareMathOperator{\Ran}{Ran}
\DeclareMathOperator{\Dom}{Dom}

\newcommand{\exo}[3]{\noindent\textcolor{red}{\fbox{\textbf{Section {#1} | Problem {#2} | {#3} points}}}\\}
\newcommand{\qu}[2]{\noindent\textcolor{red}{\fbox{\textbf{Section {#1} | Problem {#2}}}}\\}
\newcommand{\prob}[2]{\noindent\textcolor{#2}{\fbox{\textbf{Problem {#1}}}}}

\begin{document}
	\noindent \hrulefill \\
	MATH-302 \hfill Pierre-Olivier Paris{\'e}\\
	Problems set, Section 8.3 \hfill Fall 2022\\\vspace*{-1cm}
	
	\noindent\hrulefill
	
	\spc
	
	\prob{A}{blue}
	
	Find an expression of the following functions involving only unit step functions.
	\begin{enumerate}[label=\textcolor{blue}{\arabic*)}]
	\item $f(t) = \oa \begin{matrix}
	0 & t < 2 \\
	(t - 2)^2 & t > 2 \text{.}
	\end{matrix}\right.$
	\item $f(t) = \oa \begin{matrix}
	0 & t < \pi \\
	t - \pi & t \in [\pi , 2\pi ] \\
	0 & t > 2\pi \text{.}
	\end{matrix} \right.$
	\end{enumerate}
	
	\spc
	
	\prob{B}{blue}
	
	Find the Laplace transform of the following functions.
	\begin{multicols}{2}
	\begin{enumerate}[label=\textcolor{blue}{\arabic*)}]
	\item $f(t) := t u (t - 1)$.
	\item $f(t) := te^{2t} u (t - 1)$.
	\item $f(t) := (t^2 - 1) u(t - 1)$.
	\item $f(t) := t \sin t u (t - 2\pi )$. 
	\item $f(t) := \cos t u (t - a )$.
	\end{enumerate}
	\end{multicols}
	
	\spc
	
	\prob{C}{blue}
	
	Find the inverse Laplace transform of the following functions.
	\begin{multicols}{2}
	\begin{enumerate}[label=\textcolor{blue}{\arabic*)}]
	\item $F(s) = \frac{3!}{(s - 2)^4}$.
	\item $F(s) = \frac{2 (s - 1)e^{-2s}}{s^2 - 2s + 2}$.
	\item $F(s) = \frac{(s - 2)e^{-s}}{s^2 - 4s + 3}$.
	\item $F(s) = \frac{e^{-2s}}{s^2 + s - 2}$.
	\item $F(s) = \frac{2e^{-2s}}{s^2 - 4}$.
	\item $F(s) = \frac{e^{-s} + e^{-2s} - e^{-3s} - e^{-4s}}{s}$.
	\end{enumerate}
	\end{multicols}
	

\begin{comment}

	\newpage
	
	\begin{center}
	\large
	\textcolor{red}{\textbf{Complete Solutions}}
	\end{center}
	
	\noindent\textcolor{red}{\hrulefill}
	
	\prob{A}{red}
	
	\begin{enumerate}[label=\textcolor{red}{\arabic*)}]
\item The expression of $f$ in terms of unit step functions is
	\begin{align*}
	(t - 2)^2 u_2 (t) \text{.}
	\end{align*}
\item First of all, the expression of the function is $t - \pi$, when $t \geq \pi$. Therefore, the function should take the following form: 
	\begin{align*}
	(t - \pi ) u_{\pi} (t) \text{.}
	\end{align*}
Second of all, the expression $t - \pi$ vanish when $t > 2\pi$. Therefore, the function should also have the following part
	\begin{align*}
	- (t - \pi ) u_{2\pi} (t) \text{.}
	\end{align*}	
Therefore, we obtain
	\begin{align*}
	f(t) = (t - \pi ) \op u_{\pi} (t) - u_{2\pi} (t) \fp \text{.}
\end{align*}	 
\end{enumerate}
	
	\prob{B}{red}
	
	\begin{enumerate}[label=\textcolor{red}{\arabic*)}]
\item We rewrite the expression of the function as
	\begin{align*}
	tu(t - 1) = (t - 1) u(t - 1) + u (t - 1) .
	\end{align*}
In this form, we can apply the Laplace transform directly. We find that
	\begin{align*}
	F (s) = e^{-s} L (t) + e^{-s} L (1) = \frac{e^{-s}}{s^2} + \frac{e^{-s}}{s} \text{.}
	\end{align*}
\item First of all, we have
	\begin{align*}
	te^{2t} = (t - 1)e^{2t} + e^{2t} = e^{2} (t - 1)e^{2(t - 1)} + e^{2} e^{2(t -1)} \text{.}
	\end{align*}
Therefore, the expression of the function becomes
	\begin{align*}
	f(t) = e^2 (t - 1)e^{2(t - 1)} u(t-1) + e^2 e^{2(t - 1)} u(t - 1) \text{.}
	\end{align*}
Also, since $L (te^{2t}) = \frac{1}{(s - 2)^2}$ and $L (e^{2t}) = \frac{1}{s - 2}$, applying the Laplace transform implies that
	\begin{align*}
	F(s) = \frac{e^2 e^{-s}}{(s - 2)^2} + \frac{e^2 e^{-s}}{s - 2} = \frac{e^{-(s - 2)}}{(s - 2)^2} + \frac{e^{-(s - 2)}}{s - 2} \text{.}
	\end{align*}
Finally, after simplifying the answer, we get
	\begin{align*}
	F(s) = e^{-(s - 2)} \op \frac{1}{(s - 2)^2} + \frac{1}{s - 2} \fp \text{.}
	\end{align*}
\item The expression of the function is not in the form to apply the Laplace transform correctly. We therefore rewrite the expression in the following way: 
	\begin{align*}
	f(t) = (t^2 + 2t - 2t + 1 - 1 - 1 ) u(t - 1) &= (t^2 - 2t + 1) u(t - 1) + (2t - 2) u(t - 1) \\
	&= (t - 1)^2 u(t - 1) + 2 (t - 1) u(t - 1) \text{.}
	\end{align*}
We can now apply the Laplace transform and then obtain
	\begin{align*}
	F(s) = \frac{2 e^{-s}}{s^3} + \frac{2e^{-s}}{s^2} \text{.}
	\end{align*}
The simplified answer is
	\begin{align*}
	F(s) = 2e^{-s} \op \frac{1}{s^3} + \frac{1}{s^2} \fp \text{.}
	\end{align*}
\item We have to put the expression of the function in an appropriate form to apply the Laplace transform. The expression can be rewritten as followed:
	\begin{align*}
	f(t) = (t - 2\pi ) \sin t u(t - 2\pi ) + 2\pi \sin t u(t - 2\pi ) \text{.}
	\end{align*}
From a trigonometry identity (or the periodicity of the $\sin$ function), we find that
	\begin{align*}
	\sin t = \sin (t - 2\pi )
	\end{align*}
and then we see that
	\begin{align*}
	f(t) = (t - 2\pi ) \sin (t - 2\pi ) u(t - 2\pi ) + 2\pi \sin (t - 2\pi ) u(t - 2\pi ) \text{.}
	\end{align*}
We can now apply the Laplace transform to the last expression. Starting with the fact that $L (t g ) = -G'$, we get $L (t \sin t ) = \frac{2s}{(s^2 + 1)^2}$ et therefore
	\begin{align*}
	F (s) = \frac{2se^{-2\pi s}}{(s^2 + 1)^2} + \frac{2\pi e^{-2\pi s}}{s^2 + 1} \text{.}
	\end{align*}
The simplified answer is
	\begin{align*}
	F(s) = 2e^{-2\pi s} \op \frac{s}{(s^2 + 1)^2} + \frac{\pi}{s^2 + 1}\fp \text{.}
	\end{align*}
\item From a trigonometry identity, we have
	\begin{align*}
	\cos (t) = \cos (t - a + a) = \cos (t - a) \cos (a) - \sin (t - a) \sin (a) \text{.}
	\end{align*}
	Therefore, the expression of the function can be rewritten as
		\begin{align*}
		f(t) = \cos (a) \cos (t - a) u(t - a) - \sin (a) \sin (t - a) u(t - a) \text{.}
		\end{align*}
	We can now apply the Laplace transform and get
		\begin{align*}
		F(s) = \cos (a) \frac{se^{-sa}}{s^2 + 1} - \sin (a) \frac{e^{-sa}}{s^2 + 1} \text{.}
		\end{align*}
	Therefore, we get
		\begin{align*}
		F(s) = \frac{e^{-sa}}{s^2 + 1} \op s \cos a - \sin a \fp \text{.}
		\end{align*}
\end{enumerate}
	
	\prob{C}{red}
	
	In this problem, to simplify the notation, we use the following shortcuts:
		\begin{align*}
		f_a (t) = f(t - a) \quad \text{ and} \quad u_a (t) = u (t - a) .
		\end{align*}
	Therefore, when you see $f_a$ and/or $u_a$ in the text, these refer to the above.
	
	\begin{enumerate}[label=\textcolor{red}{\arabic*)}]
\item From the table, we immediately have
	\begin{align*}
	f(t) = t^3 e^{2t} \text{.}
	\end{align*}
\item The denominator can be rewritten as $(s - 1)^2 + 1$. Then, from the table, we get
	\begin{align*}
	L^{-1} \op \frac{s - 1}{(s - 1)^2 + 1} \fp = e^t \cos t \text{.}
	\end{align*}
Since the expression of $F$ contained an $e^{-2s}$, we find that
	\begin{align*}
	f(t) = 2e^{t - 2} \cos (t - 2) u (t - 2)
	\end{align*}
because $L^{-1} (F e^{-as}) = f_a u_a$.
\item The denominator can be factored:
	\begin{align*}
	s^2 - 4s + 3 = (s - 2)^2 - 1 \text{.}
	\end{align*}
	Then from the table, we get
		\begin{align*}
		L^{-1} \op \frac{s - 2}{(s - 2)^2 - 1} \fp = e^{2t} \cosh (t) \text{.}
		\end{align*}
	Finaly, since the expression of $F$ contained an $e^{-s}$, we find that
		\begin{align*}
		f(t) = e^{2(t - 1)} \cosh (t - 1) u (t - 1)
		\end{align*}
	because $L^{-1} (F e^{-as}) = f_a u_a$.
\item The denominator can be rewritten as
	\begin{align*}
	s^2 + s - 2 = (s + 2) (s - 1)
	\end{align*}
and then the expression of $F$ becomes
	\begin{align*}
	F(s) = \frac{e^{-2s}}{3(s - 1)} - \frac{e^{-2s}}{3(s +2)}
	\end{align*}
after using a partial fractions decomposition. Therefore, applying the inverse Laplace transform and keeping in mind that $L^{-1} (F e^{-as}) = f_a u_a$, we find that
	\begin{align*}
	f(t) = \tfrac{1}{3} e^{t - 2} u(t - 2) - \tfrac{1}{3} e^{-2(t - 2)} u (t - 2) \text{.}
	\end{align*}
\item Using a partial fractions decomposition, we can rewrite $F$ as followed:
	\begin{align*}
	F(s) = \op \tfrac{1}{4(s - 2)} - \tfrac{1}{4(s + 2)} \fp 2e^{-2s} \text{.}
	\end{align*}
Applying the inverse Laplace transform and keeping in mind that $L^{-1} (F e^{-as}) = f_a u_a$, we find that
	\begin{align*}
	f(t) = \op \tfrac{1}{2} e^{2(t - 2)} - \tfrac{1}{2} e^{-2 (t - 2)} \fp u (t - 2) \text{.}
	\end{align*}
	This solution is also equivalent to:
		\begin{align*}
		f(t) = \sinh (2(t - 2)) u(t - 2)
		\end{align*}
	because  $\sinh (x) = \frac{e^x - e^{-x}}{2}$.
	
	Notice that we could use the table directly. Indeed, we know that $L (\sinh (2t)) = \frac{2}{s^2 - 4}$. Therefore, since $L (g(t-a)u(t - a)) = e^{-as} G(s)$, we find that
		\begin{align*}
		f(t) = \sinh (2(t - 2)) u(t - 2) \text{.}
		\end{align*}
\item Applying directly the fact that $L^{-1} \op \frac{e^{-as}}{s} \fp = U_a$, we find that
	\begin{align*}
	f(t) = u(t - 1) + u(t - 2) - u(t- 3) - u (t - 4) \text{.}
	\end{align*}
	\end{enumerate}
	
\end{comment}
	
\end{document}

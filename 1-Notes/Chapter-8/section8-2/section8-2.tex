\documentclass[12pt,a4paper]{article}
\usepackage[utf8]{inputenc}
\usepackage[english]{babel}

\usepackage{amsmath}
\usepackage{amsfonts}
\usepackage{amssymb}

\usepackage{graphicx}
\usepackage{lmodern}
\usepackage{tikz}
\usepackage{titlesec}
\usepackage{environ}
\usepackage{xcolor}
\usepackage{fancyhdr}
\usepackage[colorlinks = true, linkcolor = black]{hyperref}
\usepackage{xparse}
\usepackage{enumitem}
\usepackage{comment}
\usepackage{wrapfig}
\usepackage[capitalise]{cleveref}

\usepackage[left=2cm,right=2cm,top=2cm,bottom=2cm]{geometry}
\usepackage{multicol}
\usepackage[indent=0pt]{parskip}

\newcommand{\spaceP}{\vspace*{0.5cm}}
\newcommand{\Span}{\mathrm{Span}\,}
\newcommand{\range}{\mathrm{range}\,}
\newcommand{\ra}{\rightarrow}

%% Redefining sections
\newcommand{\sectionformat}[1]{%
    \begin{tikzpicture}[baseline=(title.base)]
        \node[rectangle, draw] (title) {#1};
    \end{tikzpicture}
    
    \noindent\hrulefill
}

\newif\ifhNotes 

\hNotesfalse

\ifhNotes
	\newcommand{\hideNotes}[1]{%
	\phantom{#1}
	}
	\newcommand{\hideNotesU}[1]{%
	\underline{\hspace{1mm}\phantom{#1}\hspace{1mm}}
	}
\else
	\newcommand{\hideNotes}[1]{#1}
	\newcommand{\hideNotesU}[1]{\textcolor{blue}{#1}}
\fi

% default values copied from titlesec documentation page 23
% parameters of \titleformat command are explained on page 4
\titleformat%
    {\section}% <command> is the sectioning command to be redefined, i. e., \part, \chapter, \section, \subsection, \subsubsection, \paragraph or \subparagraph.
    {\normalfont\large\scshape}% <format>
    {}% <label> the number
    {0em}% <sep> length. horizontal separation between label and title body
    {\centering\sectionformat}% code preceding the title body  (title body is taken as argument)

%% Set counters for sections to none
\setcounter{secnumdepth}{0}

%% Set the footer/headers
\pagestyle{fancy}
\fancyhf{}
\renewcommand{\headrulewidth}{0pt}
\renewcommand{\footrulewidth}{2pt}
\lfoot{P.-O. Paris{\'e}}
\cfoot{MATH 302}
\rfoot{Page \thepage}

%% Defining example environment
\newcounter{example}[section]
\NewEnviron{example}%
	{%
	\noindent\refstepcounter{example}\fcolorbox{gray!40}{gray!40}{\textsc{\textcolor{red}{Example~\theexample.}}}%
	%\fcolorbox{black}{white}%
		{  %\parbox{0.95\textwidth}%
			{
			\BODY
			}%
		}%
	}

% Theorem environment
\NewEnviron{theorem}%
	{%
	\noindent\refstepcounter{example}\fcolorbox{gray!40}{gray!40}{\textsc{\textcolor{blue}{Theorem~\theexample.}}}%
	%\fcolorbox{black}{white}%
		{  %\parbox{0.95\textwidth}%
			{
			\BODY
			}%
		}%
	}

\NewEnviron{notes}%
	{%
	\noindent \fcolorbox{gray!40}{gray!40}{\textsc{\textcolor{blue}{Solution.}}}%
	%\fcolorbox{black}{white}%
		{  %\parbox{0.95\textwidth}%
			{
			\textcolor{blue}{%
			\BODY
			}
			}%
		}%
	}
%%% Ignorer les notes
\excludecomment{notes}

%%%%
\begin{document}
\thispagestyle{empty}

\begin{center}
\vspace*{2.5cm}

{\Huge \textsc{Math 302}}

\vspace*{2cm}

{\LARGE \textsc{Chapter 8}} 

\vspace*{0.75cm}

\noindent\textsc{Section 8.2: Laplace Transforms}

\vspace*{0.75cm}

\tableofcontents

\vfill

\noindent \textsc{Created by: Pierre-Olivier Paris{\'e}} \\
\textsc{Fall 2022}
\end{center}

\newpage

\section{From ODE to Algebra}
	The Laplace transform is $\text{REALLY}^{\text{REALLY}}$ useful to solve ODE.
	
	\vspace*{14pt}
	
	\begin{example}\label{Ex:SolvingODE}
	Consider the ODE
		\begin{align*}
		2y'' (t) + 3y' (t) + y(t) = 8e^{-2t} 
		\end{align*}
	with $y(0) = -4$ and $y'(0) = 2$.
	\end{example}
	
	\newpage
	
	\phantom{2}
	
	\vfill
	
	\underline{\textbf{General Procedure:}}
	\begin{enumerate}
	\item Apply the Laplace transform to your ODE $ay'' + by' + cy = f(t)$.
	\item Apply the properties of the Laplace transform to get
		\begin{align*}
		a(s^2 Y - sf(0) - f'(0)) + b (s Y - f(0)) + c Y = F .
		\end{align*}
	\item Isolate $Y$:
		\begin{align*}
		Y = \frac{F + (as + b) f(0) + af'(0)}{as^2 + bs + c} .
		\end{align*}
	\end{enumerate}
	
The last step: 
	\begin{itemize}
	\item Take the inverse Laplace transform!
	\end{itemize}
	
	\newpage
	
	\section{Inverse Laplace Transform}
	Given a Laplace transform $F(s)$ of an unknown function $f$, we can go backward to find $f$.
	
		\begin{itemize}
		\item We denote the \textbf{inverse Laplace transform} by $L^{-1}$.
		\item We therefore have $f = L^{-1} (F)$.
		\item How do we find $L^{-1} (F)$?
		\end{itemize}
		
	\underline{Trick:} Use the table in the opposite direction!
		
	\vspace*{16pt}
	
	\begin{example}
	Find the inverse Laplace transform of the following functions:
		\begin{enumerate}[label=\textbf{(\alph*)}]
		\item $\displaystyle \frac{1}{s^2 - 1}$.
		\item $\displaystyle \frac{s}{s^2 + 9}$.
		\end{enumerate}
	\end{example}
	
	\newpage
	
	\subsection{Linearity of Inverse Transform}
	If $F$ and $G$ are Laplace transforms of two unknown functions $f$ and $g$, then
		\begin{align*}
		L^{-1} (aF + bG) = a L^{-1} (F) + b L^{-1} (G) .
		\end{align*}
	
	\begin{example}
	Find
		\begin{align*}
		L^{-1} \Big( \frac{8}{s + 5} + \frac{7}{s^2 + 3} \Big) .
		\end{align*}
	\end{example}
	
	\newpage
	
	\begin{example}
	Find
		\begin{align*}
		L^{-1} \Big( \frac{3s + 8}{s^2 + 2s + 5} \Big) .
		\end{align*}
	\end{example}
	
	\newpage
	
	\subsection{Inverse Laplace Transform of Rational Functions}
	
	\begin{example}
	Find the inverse Laplace transform of
		\begin{align*}
		F(s) = \frac{3s + 2}{s^2 - 3s + 2} .
		\end{align*}
	\end{example}
	
	\newpage
	
	\begin{example}
	Find the inverse transform of
		\begin{align*}
		F(s) = \frac{6 + (s + 1) (s^2 - 5s + 11)}{s (s - 1) (s - 2) (s + 1)} .
		\end{align*}
	\end{example}
	
	\newpage
	\phantom{2}
	
	\newpage
	
	\underline{General Case:}\\
	Suppose your Laplace transform is
		\begin{align*}
		F(s) = \frac{P(s)}{(s - s_1) (s - s_2) \cdots (s - s_n)}
		\end{align*}
	where $s_1$, $s_2$, $\ldots$, $s_n$ are distinct and $P$ is a polynomial of degree less than $n$. Then
		\begin{align*}
		F(s) = \frac{A_1}{s - s_1} + \frac{A_2}{s - s_2} + \cdots + \frac{A_n}{s - s_n},
		\end{align*}
	\begin{itemize}
	\item $A_1$ is computed by letting $s = s_1$ in $G(s) = \displaystyle\frac{P(s)}{(s - s_2) \cdots (s - s_n )}$.
	\item $A_2$ is computed by letting $s = s_2$ in $G(s) = \displaystyle\frac{P(s)}{(s - s_1 ) (s - s_3) \cdots (s - s_n)}$.
	\item \hfill $\vdots$ \hfill \phantom{@} \hfill \phantom{2} \hfill \phantom{2}
	\item $A_n$ is computed by letting $s = s_n$ in $G(s) = \displaystyle \frac{P(s)}{(s - s_1) (s - s_2) \cdots (s - s_{n-1})}$.
	\end{itemize}
	
	\vspace*{16pt}
	
	\begin{comment}
	\begin{example}
	Write in partial fractions the following rational function:
		\begin{align*}
		F(s) = \frac{s^2 + 1}{s^3 - s} .
		\end{align*}
	\end{example}
	\end{comment}
	
	\newpage
	
	\subsection{Rational Functions with Powers in the Denominator}
	The Heaviside method doesn't work if we encounters powers of monomonials in the denominator. What do we do then?
	
	\vspace*{16pt}
	
	\begin{example}
	Find the partial fraction expansion of
		\begin{align*}
		F(s) = \frac{8 - (s + 2) (4s + 10)}{(s + 1) (s + 2)^2} .
		\end{align*}
	\end{example}
	
	\newpage
	
	\section{The ODE Problem}
	
	\begin{example}
	Using the inverse Laplace transform, complete Example \ref{Ex:SolvingODE}.
	\end{example}
	
	\newpage
	
	\begin{example}
	Use the Laplace transform to solve the initial value problem:
		\begin{align*}
		y'' - 6y' + 5y = 3e^{2t} , \quad y(0) = 2 , \, y' (0) = 3 .
		\end{align*}
	\end{example}
	
	\newpage
	
	\phantom{2}

\end{document}
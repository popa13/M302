\documentclass[12pt,a4paper]{article}
\usepackage[utf8]{inputenc}
\usepackage[english]{babel}

\usepackage{amsmath}
\usepackage{amsfonts}
\usepackage{amssymb}

\usepackage{graphicx}
\usepackage{lmodern}
\usepackage{tikz}
\usepackage{titlesec}
\usepackage{environ}
\usepackage{xcolor}
\usepackage{fancyhdr}
\usepackage[colorlinks = true, linkcolor = black]{hyperref}
\usepackage{xparse}
\usepackage{enumitem}
\usepackage{comment}
\usepackage{wrapfig}
\usepackage[capitalise]{cleveref}

\usepackage[left=2cm,right=2cm,top=2cm,bottom=2cm]{geometry}
\usepackage{multicol}
\usepackage[indent=0pt]{parskip}

\newcommand{\spaceP}{\vspace*{0.5cm}}
\newcommand{\Span}{\mathrm{Span}\,}
\newcommand{\range}{\mathrm{range}\,}
\newcommand{\ra}{\rightarrow}

%% Redefining sections
\newcommand{\sectionformat}[1]{%
    \begin{tikzpicture}[baseline=(title.base)]
        \node[rectangle, draw] (title) {#1};
    \end{tikzpicture}
    
    \noindent\hrulefill
}

\newif\ifhNotes 

\hNotesfalse

\ifhNotes
	\newcommand{\hideNotes}[1]{%
	\phantom{#1}
	}
	\newcommand{\hideNotesU}[1]{%
	\underline{\hspace{1mm}\phantom{#1}\hspace{1mm}}
	}
\else
	\newcommand{\hideNotes}[1]{#1}
	\newcommand{\hideNotesU}[1]{\textcolor{blue}{#1}}
\fi

% default values copied from titlesec documentation page 23
% parameters of \titleformat command are explained on page 4
\titleformat%
    {\section}% <command> is the sectioning command to be redefined, i. e., \part, \chapter, \section, \subsection, \subsubsection, \paragraph or \subparagraph.
    {\normalfont\large\scshape}% <format>
    {}% <label> the number
    {0em}% <sep> length. horizontal separation between label and title body
    {\centering\sectionformat}% code preceding the title body  (title body is taken as argument)

%% Set counters for sections to none
\setcounter{secnumdepth}{0}

%% Set the footer/headers
\pagestyle{fancy}
\fancyhf{}
\renewcommand{\headrulewidth}{0pt}
\renewcommand{\footrulewidth}{2pt}
\lfoot{P.-O. Paris{\'e}}
\cfoot{MATH 302}
\rfoot{Page \thepage}

%% Defining example environment
\newcounter{example}[section]
\NewEnviron{example}%
	{%
	\noindent\refstepcounter{example}\fcolorbox{gray!40}{gray!40}{\textsc{\textcolor{red}{Example~\theexample.}}}%
	%\fcolorbox{black}{white}%
		{  %\parbox{0.95\textwidth}%
			{
			\BODY
			}%
		}%
	}

% Theorem environment
\NewEnviron{theorem}%
	{%
	\noindent\refstepcounter{example}\fcolorbox{gray!40}{gray!40}{\textsc{\textcolor{blue}{Theorem~\theexample.}}}%
	%\fcolorbox{black}{white}%
		{  %\parbox{0.95\textwidth}%
			{
			\BODY
			}%
		}%
	}

\NewEnviron{notes}%
	{%
	\noindent \fcolorbox{gray!40}{gray!40}{\textsc{\textcolor{blue}{Solution.}}}%
	%\fcolorbox{black}{white}%
		{  %\parbox{0.95\textwidth}%
			{
			\textcolor{blue}{%
			\BODY
			}
			}%
		}%
	}
%%% Ignorer les notes
\excludecomment{notes}

%%%%
\begin{document}
\thispagestyle{empty}

\begin{center}
\vspace*{2.5cm}

{\Huge \textsc{Math 302}}

\vspace*{2cm}

{\LARGE \textsc{Chapter 8}} 

\vspace*{0.75cm}

\noindent\textsc{Section 8.1: Laplace Transforms}

\vspace*{0.75cm}

\tableofcontents

\vfill

\noindent \textsc{Created by: Pierre-Olivier Paris{\'e}} \\
\textsc{Fall 2022}
\end{center}

\newpage

\section{The Laplace Transform}
From now on,
	\begin{itemize}
	\item the variable $t$ stands for the independent variable (time).
	\end{itemize}
\subsection{Discrete Process: Power series}

\begin{notes}
Take a power series
	\begin{align*}
	L(x) = \sum_{n = 0}^\infty a_n x^n .
	\end{align*}

What is the relationship between $L (x)$ and $a_n$?
	\begin{itemize}
	\item $L(x)$ is the discrete sum of the $a_n$.
	\item $L(x)$ represente the contributions of each $a_n$.
	\end{itemize}
\end{notes}

\vspace*{8cm}
	
\subsection{Continuous Process}

\begin{notes}
	
Imagine that we replace $a_n$ by a continuous $a(t)$, where $t \in [0, \infty )$. Then, the analogue of the series in the function world is the integral:
	\begin{align*}
	L(x) = \int_0^\infty a(t) x^t \, dt .
	\end{align*}
	\begin{itemize}
	\item $L(x)$ is now the continuous sum of $a(t)$.
	\item $L(x)$ represents the contributions of each $a(t)$.
	\end{itemize}
	
The integral will be defined if $x^t < 1$, and so $x < 1$. Changing the variable $x$ to $e^{-s}$ for some variable $s$ gives
	\begin{align*}
	L(s) = \int_0^\infty a(t) e^{-st} \, dt .
	\end{align*}
This last expression is the Laplace Transform of the function $a(t)$!

\underline{\textbf{Laplace Transform.}}

The Laplace transform of a function $f$ is the new function $F(s)$ given by
	\begin{align*}
	F(s) = \int_0^\infty f(t) e^{-st} \, dt .
	\end{align*}
\end{notes}
	
\vfill
	
\underline{Remark:}
	\begin{itemize}
	\item Recall that, with power series, we were able to solve a differential equation by solving a recurrence relation (so, basically, doing some algebra with a discrete number of data).
	\item With the Laplace transform, we will also be able to reduce an ODE problem into an algebra one.
	\item We use the symbol $L(f(t))$ to also denote the Laplace transform $F(s)$.
	\end{itemize}
	
\newpage

\begin{example}
Compute the Laplace transform of the function $f(t) = t$.
\end{example}

\vfill

Here is a sample table of Laplace Transforms.

\begin{table}[ht!]
\centering
{\renewcommand{\arraystretch}{2.49}
	\setlength{\tabcolsep}{2pt}
\begin{tabular}{|c|c||c|c|}
\hline
\hspace{0.75cm} \textbf{Function} \hspace{0.75cm} & \hspace{0.75cm} \textbf{Transform} \hspace{0.75cm} & \hspace{0.75cm} \textbf{Function} \hspace{0.75cm}  & \hspace{0.75cm} \textbf{Transform} \hspace{0.75cm} \\\hline
\hfill $1$ \hfill\hfill & $\hideNotes{\displaystyle\frac{1}{s}}$ &  \hfill $\sin (\omega t )$ \hfill \hfill & $\hideNotes{\displaystyle\frac{\omega}{s^2 + \omega^2}}$ \\\hline
 \hfill $t$ \hfill\hfill & $\hideNotes{\displaystyle\frac{1}{s^2}}$ &  \hfill $\cos (\omega t )$ \hfill \hfill & $\hideNotes{\displaystyle \frac{s}{s^2 + \omega^2}}$ \\\hline
 \hfill $t^n $ \hfill \hfill & $\displaystyle\frac{n!}{s^{n + 1}}$ &  \hfill $\sinh (\omega t)$ \hfill \hfill  & $\hideNotes{\displaystyle\frac{\omega}{s^2 - \omega^2}}$ \\\hline
 \hfill $e^{at}$ \hfill \hfill & $\displaystyle \frac{1}{s - a}$ &  \hfill $\cosh (\omega t)$ \hfill \hfill & $\hideNotes{\displaystyle\frac{s}{s^2 - \omega^2}}$ \\\hline
\end{tabular}}
\caption{Laplace Transforms (sample)}
\end{table}

\newpage

\section{Some Comments on Existence}
	It is important to check if a function possesses a Laplace transform.
	
	\vspace*{16pt}
	
	\underline{\textbf{Exponential Order Criterion.}}
	
	If $f(t)$ is a function satisfying
		\begin{align*}
		|f(t)| \leq M e^{s_0 t} , \, t \geq t_0
		\end{align*}
	for some numbers $s_0$, $t_0$, and $M$, then $F(s)$ exists for $s > s_0$.
	
	\vspace*{16pt}
	
	\underline{Remarks:}
		\begin{itemize}
		\item Later on, we will see that the Laplace transform exists for discontinuous functions.
		\item Even more than that, we will apply the Laplace transform on functions taking $\infty$ as values!
		\end{itemize}
		
	\vspace*{16pt}
	
	\begin{example}
	The function $f(t) = e^{t^2}$ doesn't have a Laplace transform.
	\end{example}
	
	\newpage

\section{Linearity of Laplace Transform}

\begin{example}
Justify that
	\begin{align*}
	L (\sinh (\omega t )) = \frac{\omega}{s^2 - \omega^2} .
	\end{align*}
\end{example}

\vfill

\underline{\textbf{Linearity of Laplace transform:}}

If $f$ and $g$ are two functions, and $a$, $b$ are two real numbers, then
	\begin{align*}
	L(af(t) + bg(t)) = aL(f(t)) + bL(g(t)) = a F(s) + b G(s) .
	\end{align*}
You can apply this repeatedly to more than two functions.

\newpage

\section{First Shift Theorem}

Did you notice that
	\begin{align*}
	L (e^{at}) = \frac{1}{s - a} ?
	\end{align*}

	\begin{itemize}
	\item This is $L(1)$, but with a shift $s - a$!!! 
	\item Since $e^{at} = 1 \cdot e^{at}$, we have the following shifting result.
	\end{itemize}
	\vspace*{18pt}
	
	\underline{\textbf{Shifting Theorem:}}
	
	If $f(t)$ is a function with a Laplace transform $F(s)$, then
		\begin{align*}
		L(e^{at} f(t)) = F(s - a) .
		\end{align*}
		
	\vspace*{16pt}
	
	\begin{example}
	Find the Laplace transform of
	\begin{multicols}{2}
		\begin{enumerate}[label=\textbf{(\alph*)}]
		\item $f(t) = e^{at} \sin (\omega t )$.
		\item $f(t) = e^{at} \cos (\omega t)$.
		%\item $f(t) = e^{at} \sinh (\omega t )$.
		%\item $f(t) = e^{at} \cosh (\omega t )$.
		\end{enumerate}
	\end{multicols}
	\end{example}
	
	\newpage
	
	\section{Powers and Derivatives}
	Did you notice that
		\begin{align*}
		L (t) = \frac{1}{s^2} = - \frac{d}{ds} \Big( \frac{1}{s} \Big) ?
		\end{align*}
	\begin{itemize}
	\item This is the derivative of $L(1)$, but with a different sign.
	\item Since $t = 1 \cdot t$, we have the following result.
	\end{itemize}
	
	\vspace*{16pt}
	
	\underline{\textbf{Powers Transformed in Derivatives.}}
	
	If $f$ has a Laplace transform and $n$ is a positive integer, then
		\begin{align*}
		L (t^n f(t)) = (-1)^n F^{(n)} (s) .
		\end{align*}
		
	\vspace*{16pt}
	
	\begin{example}
	Find the Laplace transform of
		\begin{multicols}{2}
		\begin{enumerate}[label=\textbf{(\alph*)}]
		\item $f(t) = t \cos (\omega t) $.
		\item $f(t) = t \sinh (\omega t )$.
		\item $f(t) = t e^{at}$.
		\item $f(t) = t \sin (2t ) + t^2 \cos (t) \sin (t)$.
		\end{enumerate}
		\end{multicols}
	\end{example}
	
	\newpage
	
	Did you notice that
		\begin{align*}
		L (\cos (t)) = \frac{s}{s^2 + 1} = \phantom{s L (\sin (t)) - \sin (0) 2222222222222222222222} ?
		\end{align*}
		
	\vspace*{16pt}
	
	\underline{\textbf{Derivatives Transformed in Powers:}}
	
	If $f, f', \ldots , f^{(n)}$ have a Laplace transform for $n \geq 1$, then
		\begin{align*}
		L ( f^{(n)} (t)) = s^n F(s) - s^{n - 1} f(0) - s^{n - 2} f ' (0) - s f^{(n-2)}(0) - f^{(n-1)}(0) .
		\end{align*}
	Most relevant formulas:
		\begin{itemize}
		\item $n = 1$: $L (f'(t)) = s F(s) - f(0)$.
		\item $n = 2$: $L (f''(t)) = s^2 F(s) - s f(0) - f'(0)$.
		\item $n = 3$: $L(f^{(3)} (t)) = s^3 F(s) - s^2 f(0) - s f'(0) - f''(0)$.
		\end{itemize}
		
	\vspace*{16pt}
	
	\begin{example}
	Find the Laplace transform of
		\begin{enumerate}[label=\textbf{(\alph*)}]
		\item $f(t) = \cos^2 (t)$.
		\item $g(t) = \sin^2 (t)$.
		\end{enumerate}
	\end{example}

\end{document}
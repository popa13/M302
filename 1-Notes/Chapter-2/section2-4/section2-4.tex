\documentclass[12pt,a4paper]{article}
\usepackage[utf8]{inputenc}
\usepackage[english]{babel}

\usepackage{amsmath}
\usepackage{amsfonts}
\usepackage{amssymb}

\usepackage{graphicx}
\usepackage{lmodern}
\usepackage{tikz}
\usepackage{titlesec}
\usepackage{environ}
\usepackage{xcolor}
\usepackage{fancyhdr}
\usepackage[colorlinks = true, linkcolor = black]{hyperref}
\usepackage{xparse}
\usepackage{enumerate}

\usepackage[left=2cm,right=2cm,top=2cm,bottom=2cm]{geometry}
\usepackage{multicol}
\usepackage[indent=0pt]{parskip}

\newcommand{\spaceP}{\vspace*{0.5cm}}
\newcommand{\Span}{\mathrm{Span}\,}
\newcommand{\range}{\mathrm{range}\,}

%% Redefining sections
\newcommand{\sectionformat}[1]{%
    \begin{tikzpicture}[baseline=(title.base)]
        \node[rectangle, draw] (title) {#1};
    \end{tikzpicture}
    
    \noindent\hrulefill
}

% default values copied from titlesec documentation page 23
% parameters of \titleformat command are explained on page 4
\titleformat%
    {\section}% <command> is the sectioning command to be redefined, i. e., \part, \chapter, \section, \subsection, \subsubsection, \paragraph or \subparagraph.
    {\normalfont\large\scshape}% <format>
    {}% <label> the number
    {0em}% <sep> length. horizontal separation between label and title body
    {\centering\sectionformat}% code preceding the title body  (title body is taken as argument)

%% Set counters for sections to none
\setcounter{secnumdepth}{0}

%% Set the footer/headers
\pagestyle{fancy}
\fancyhf{}
\renewcommand{\headrulewidth}{0pt}
\renewcommand{\footrulewidth}{2pt}
\lfoot{P.-O. Paris{\'e}}
\cfoot{MATH 302}
\rfoot{Page \thepage}

%% Defining example environment
\newcounter{example}[section]
\NewEnviron{example}%
	{%
	\noindent\refstepcounter{example}\fcolorbox{gray!40}{gray!40}{\textsc{\textcolor{red}{Example~\theexample.}}}%
	%\fcolorbox{black}{white}%
		{  %\parbox{0.95\textwidth}%
			{
			\BODY
			}%
		}%
	}

% Theorem environment
\NewEnviron{theorem}%
	{%
	\noindent\refstepcounter{example}\fcolorbox{gray!40}{gray!40}{\textsc{\textcolor{blue}{Theorem~\theexample.}}}%
	%\fcolorbox{black}{white}%
		{  %\parbox{0.95\textwidth}%
			{
			\BODY
			}%
		}%
	}
	

%%%%
\begin{document}
\thispagestyle{empty}

\begin{center}
\vspace*{2.5cm}

{\Huge \textsc{Math 302}}

\vspace*{2cm}

{\LARGE \textsc{Chapter 2}} 

\vspace*{0.75cm}

\noindent\textsc{Section 2.4: Transformation of Nonlinear Equations Into Separable Equations}

\vspace*{0.75cm}

\tableofcontents

\vfill

\noindent \textsc{Created by: Pierre-Olivier Paris{\'e}} \\
\textsc{Fall 2022}
\end{center}

\newpage

\section{A Specific Case}
We were able to solve
	\begin{align*}
	y' + p(x) y = f(x)
	\end{align*}
by 
	\begin{itemize}
	\item finding a solution $y_1$ to the complementary equation and 
	\item setting $y = u y_1$ where $u$ is the solution to the separable equation
	\begin{align*}
	u' = \frac{f(x)}{y_1 (x)} .
	\end{align*}
	\end{itemize}

\subsection{Bernoulli Equation}

A \textbf{Bernoulli equation} is an equation of the form
	\begin{align*}
	y' + p(x) y = f(x) y^r
	\end{align*}
where $r$ is any real number different from $0$ and $1$.

\underline{Trick to solve it:}

\newpage

\begin{example}
Solve the Bernoulli equation
	\begin{align*}
	y' - y = xy^2 .
	\end{align*}
\end{example}

\newpage

\section{Homogeneous Nonlinear Equation}
The first order ODE 
	\begin{align*}
	y' = f(x, y)
	\end{align*}
is said to be \textbf{homogeneous of the second kind} if it takes the form
	\begin{align*}
	y' = q(y/x)
	\end{align*}
where $q = q(u)$ is a function of a single variable.

\vspace*{16pt}

\begin{example}
The following ODEs are homogeneous of the second kind. Explain why.
	\begin{enumerate}
	\item $\displaystyle y' = \frac{y + xe^{-y/x}}{x}$.
	\item $\displaystyle x^2 y' = y^2 + xy - x^2$.
	\end{enumerate}
\end{example}
	
\vspace*{24pt}

\underline{The trick:}

\newpage

\begin{example}
\begin{enumerate}
\item Solve
	\begin{align*}
	y' = \frac{y + xe^{-y/x}}{x} .
	\end{align*}
\item Solve the boundary value problem
	\begin{align*}
	y' = \frac{y + xe^{-y/x}}{x}, \quad y(1) = 0 .
	\end{align*}
\end{enumerate}
\end{example}

\end{document}
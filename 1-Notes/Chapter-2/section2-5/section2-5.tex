\documentclass[12pt,a4paper]{article}
\usepackage[utf8]{inputenc}
\usepackage[english]{babel}

\usepackage{amsmath}
\usepackage{amsfonts}
\usepackage{amssymb}

\usepackage{graphicx}
\usepackage{lmodern}
\usepackage{tikz}
\usepackage{titlesec}
\usepackage{environ}
\usepackage{xcolor}
\usepackage{fancyhdr}
\usepackage[colorlinks = true, linkcolor = black]{hyperref}
\usepackage{xparse}
\usepackage{enumitem}

\usepackage[left=2cm,right=2cm,top=2cm,bottom=2cm]{geometry}
\usepackage{multicol}
\usepackage[indent=0pt]{parskip}

\newcommand{\spaceP}{\vspace*{0.5cm}}
\newcommand{\Span}{\mathrm{Span}\,}
\newcommand{\range}{\mathrm{range}\,}

%% Redefining sections
\newcommand{\sectionformat}[1]{%
    \begin{tikzpicture}[baseline=(title.base)]
        \node[rectangle, draw] (title) {#1};
    \end{tikzpicture}
    
    \noindent\hrulefill
}

% default values copied from titlesec documentation page 23
% parameters of \titleformat command are explained on page 4
\titleformat%
    {\section}% <command> is the sectioning command to be redefined, i. e., \part, \chapter, \section, \subsection, \subsubsection, \paragraph or \subparagraph.
    {\normalfont\large\scshape}% <format>
    {}% <label> the number
    {0em}% <sep> length. horizontal separation between label and title body
    {\centering\sectionformat}% code preceding the title body  (title body is taken as argument)

%% Set counters for sections to none
\setcounter{secnumdepth}{0}

%% Set the footer/headers
\pagestyle{fancy}
\fancyhf{}
\renewcommand{\headrulewidth}{0pt}
\renewcommand{\footrulewidth}{2pt}
\lfoot{P.-O. Paris{\'e}}
\cfoot{MATH 302}
\rfoot{Page \thepage}

%% Defining example environment
\newcounter{example}[section]
\NewEnviron{example}%
	{%
	\noindent\refstepcounter{example}\fcolorbox{gray!40}{gray!40}{\textsc{\textcolor{red}{Example~\theexample.}}}%
	%\fcolorbox{black}{white}%
		{  %\parbox{0.95\textwidth}%
			{
			\BODY
			}%
		}%
	}

% Theorem environment
\NewEnviron{theorem}%
	{%
	\noindent\refstepcounter{example}\fcolorbox{gray!40}{gray!40}{\textsc{\textcolor{blue}{Theorem~\theexample.}}}%
	%\fcolorbox{black}{white}%
		{  %\parbox{0.95\textwidth}%
			{
			\BODY
			}%
		}%
	}
	

%%%%
\begin{document}
\thispagestyle{empty}

\begin{center}
\vspace*{2.5cm}

{\Huge \textsc{Math 302}}

\vspace*{2cm}

{\LARGE \textsc{Chapter 2}} 

\vspace*{0.75cm}

\noindent\textsc{Section 2.5: Exact Equations}

\vspace*{0.75cm}

\tableofcontents

\vfill

\noindent \textsc{Created by: Pierre-Olivier Paris{\'e}} \\
\textsc{Fall 2022}
\end{center}

\newpage

\section{Another Way to Present an ODE}

\begin{example}
Consider $y' = dy/dx$ and use this to rewrite the ODE
	\begin{align*}
	y' = \frac{y + xe^{-y/x}}{x}
	\end{align*}
in terms of $dx$ and $dy$.
\end{example}

\vfill

\underline{Convenient form:}

We will now consider an homogeneous first order ODE in the form
	\begin{align}
	M(x, y) dx + N (x, y) dy = 0 \label{Eq:ExactForm}
	\end{align}
where $M$ and $N$ are two functions of the variables $x$ and $y$.

\underline{Two interpretations:}
	\begin{itemize}
	\item the equation \eqref{Eq:ExactForm} can be interpreted as
		\begin{align}
		M(x, y) + N(x, y) \frac{dy}{dx} = 0 \label{Eq:Exactdydx}
		\end{align}
	where $x$ is the independent variable and $y$ is the dependent variable.
	\item the equation \eqref{Eq:ExactForm} can be interpreted as
		\begin{align}
		M (x, y) \frac{dx}{dy} + N(x, y) = 0 \label{Eq:Exactdxdy}
		\end{align}
	where $x$ is the dependent variable and $y$ is the independent variable.
	\item An implicit equation $F(x, y) = c$ is said to be an \textbf{implicit solution} to \eqref{Eq:ExactForm} if
		\begin{itemize}
		\item every function $y = y(x)$ satisfying $F(x, y(x)) = c$ is a solution to \eqref{Eq:Exactdydx}.
		\item every function $x = x(y)$ satisfying $F(x(y), y) = c$ is a solution to \eqref{Eq:Exactdxdy}
		\end{itemize}
	\end{itemize}
	
\newpage
	
	\section{Exactness Condition}
	
	\begin{example}
	Show that
		\begin{align*}
		x^4 y^3 + x^2 y^5 + 2xy = c
		\end{align*}
	is an implicit solution of
		\begin{align*}
		(4x^3 y^3 + 2xy^5 + 2y) dx + (3x^4y^2 + 5x^2 y^4 + 2x) dy = 0 .
		\end{align*}
	\end{example}
	
	\vfill
	
	\underline{General Fact:}
	
	If $F(x, y) = c$ with $F$ having continuous partial derivatives $F_x$ and $F_y$, then
		\begin{align*}
		F(x, y) = c
		\end{align*}
	is an implicit solution to the differential equation
		\begin{align*}
		F_x (x, y) dx + F_y (x, y) dy = 0 .
		\end{align*}
		
	\newpage
	
	So, a differential equation is said to be \textbf{exact} on an open rectangle $R$ if there is a function $F = F(x, y)$ such that
		\begin{align*}
		F_x (x, y) = M (x, y) \quad \text{ and } \quad F_y = N (x, y) .
		\end{align*}
		
	\underline{Useful fact (the exactness condition):}
	
	A differentiel equation is exact if and only if
		\begin{align*}
		M_y (x, y) = N_x (x, y) .
		\end{align*}
	
	\vspace*{16pt}
	
	\begin{example}
	Check if the following ODEs are exact or not.
		\begin{enumerate}
		\item $3x^2 y dx + 4x^3 dy = 0$.
		\item $(4x^3 y^3 + 3x^2)dx + (3x^4 y^2 + 6y^2) dy = 0$.
		\end{enumerate}
	\end{example}
	
	\newpage
	
	\section{How to Solve Exact ODEs}
	
	\begin{example}
	Solve
		\begin{align*}
		y' = -\frac{4x^3 y^3 + 3x^2}{3x^4 y^2 + 6y^2} .
		\end{align*}
	\end{example}
	
	\newpage
	
	\subsection{Non Rigorous but ``Fast'' Procedure to Solve An Exact ODE}
	\begin{enumerate}[label=\textbf{[\Roman*]}]
	\item Check that the equation
		\begin{align*}
		M (x, y) dx + N(x, y) dy = 0
		\end{align*}
	satisfies the exactness condition.
	\item Integrate the equation $F_x = M (x, y)$ with respect to $x$ to get
		\begin{align*}
		F (x, y) = G(x, y) .
		\end{align*}
	\item Integrate the equation $F_y = N(x, y)$ with respect to $y$ to get
		\begin{align*}
		F(x, y) = H (x, y) .
		\end{align*}
	\item Identity what is in common in the expressions of the functions $G$ and $H$. Call this common part $F_1 (x, y)$.
	\item Identity what is not in common in the expressions of the functions $G$ and $H$. Gather the uncommon part in a function $F_2 (x, y)$.
	\item Write $F (x, y) = F_1 (x, y) + F_2 (x, y)$.
	\end{enumerate}
	
\vspace*{16pt}
	
\underline{Remarks:}
	\begin{itemize}
	\item This shortcut may not work if one of the function $G$ or $H$ has an integral that can't be simplified.
	\item Sometimes, the rigorous procedure is faster (see next section).
	\item For the step-by-step rigorous procedure, see Example 2.5.3 (p.75) and p.77 of the textbook.
	\end{itemize}

\end{document}
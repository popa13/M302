\documentclass[12pt,a4paper]{article}
\usepackage[utf8]{inputenc}
\usepackage[english]{babel}

\usepackage{amsmath}
\usepackage{amsfonts}
\usepackage{amssymb}

\usepackage{graphicx}
\usepackage{lmodern}
\usepackage{tikz}
\usepackage{titlesec}
\usepackage{environ}
\usepackage{xcolor}
\usepackage{fancyhdr}
\usepackage[colorlinks = true, linkcolor = black]{hyperref}
\usepackage{xparse}
\usepackage{enumitem}

\usepackage[left=2cm,right=2cm,top=2cm,bottom=2cm]{geometry}
\usepackage{multicol}
\usepackage[indent=0pt]{parskip}

\newcommand{\spaceP}{\vspace*{0.5cm}}
\newcommand{\Span}{\mathrm{Span}\,}
\newcommand{\range}{\mathrm{range}\,}

%% Redefining sections
\newcommand{\sectionformat}[1]{%
    \begin{tikzpicture}[baseline=(title.base)]
        \node[rectangle, draw] (title) {#1};
    \end{tikzpicture}
    
    \noindent\hrulefill
}

% default values copied from titlesec documentation page 23
% parameters of \titleformat command are explained on page 4
\titleformat%
    {\section}% <command> is the sectioning command to be redefined, i. e., \part, \chapter, \section, \subsection, \subsubsection, \paragraph or \subparagraph.
    {\normalfont\large\scshape}% <format>
    {}% <label> the number
    {0em}% <sep> length. horizontal separation between label and title body
    {\centering\sectionformat}% code preceding the title body  (title body is taken as argument)

%% Set counters for sections to none
\setcounter{secnumdepth}{0}

%% Set the footer/headers
\pagestyle{fancy}
\fancyhf{}
\renewcommand{\headrulewidth}{0pt}
\renewcommand{\footrulewidth}{2pt}
\lfoot{P.-O. Paris{\'e}}
\cfoot{MATH 302}
\rfoot{Page \thepage}

%% Defining example environment
\newcounter{example}[section]
\NewEnviron{example}%
	{%
	\noindent\refstepcounter{example}\fcolorbox{gray!40}{gray!40}{\textsc{\textcolor{red}{Example~\theexample.}}}%
	%\fcolorbox{black}{white}%
		{  %\parbox{0.95\textwidth}%
			{
			\BODY
			}%
		}%
	}

% Theorem environment
\NewEnviron{theorem}%
	{%
	\noindent\refstepcounter{example}\fcolorbox{gray!40}{gray!40}{\textsc{\textcolor{blue}{Theorem~\theexample.}}}%
	%\fcolorbox{black}{white}%
		{  %\parbox{0.95\textwidth}%
			{
			\BODY
			}%
		}%
	}
	

%%%%
\begin{document}
\thispagestyle{empty}

\begin{center}
\vspace*{2.5cm}

{\Huge \textsc{Math 302}}

\vspace*{2cm}

{\LARGE \textsc{Chapter 2}} 

\vspace*{0.75cm}

\noindent\textsc{Section 2.6: Integrating Factors}

\vspace*{0.75cm}

\tableofcontents

\vfill

\noindent \textsc{Created by: Pierre-Olivier Paris{\'e}} \\
\textsc{Fall 2022}
\end{center}

\newpage

\section{What's An Integrating Factor}

\begin{example}
Verify if 
	\begin{align*}
	(3x + 2y^3) dx + 2xy dy = 0
	\end{align*}
is exact.
\end{example}

\vfill

A function $\mu = \mu (x ,y)$ is an \textbf{integrating factor} for
	\begin{align*}
	M (x, y) dx + N (x, y) dy = 0
	\end{align*}
if the equation
	\begin{align*}
	\mu (x, y) M (x, y) dx + \mu (x, y) N (x ,y) dy = 0 
	\end{align*}
is exact.

\newpage

\section{Finding Integrating Factors}

Let's start with the equation
	\begin{align}
	\mu (x, y) M (x, y) dx + \mu (x, y) N (x, y) dy = 0 . \label{Eq:ExactForm}
	\end{align}
	
\underline{Trick:}

\vfill

\underline{General Facts:}
Let $M$, $N$, $M_y$, $N_x$ be continuous on an open rectangle $R$.
	\begin{itemize}
	\item if $(M_y - N_x)/N$ is independent of $y$, then
		\begin{align*}
		\mu (x, y) = \pm e^{\int p (x) \, dx}
		\end{align*}
	is an integrating factor for \eqref{Eq:ExactForm} where $p(x) = (M_y - N_x)/N$.
	\item if $(N_x - M_y)/M$ is independent of $x$, then
		\begin{align*}
		\mu (x, y) = \pm e^{\int q (y) \, dy}
		\end{align*}
	is an integrating factor for \eqref{Eq:ExactForm} where $q(y) = (N_x - M_y)/M$.
	\end{itemize}
	
\newpage

\begin{example}
Find an integrating factor for the equation
	\begin{align*}
	(2xy^3 - 2x^3 y^3 - 4xy^2 + 2x) dx + (3x^2y^2 + 4y) dy = 0 .
	\end{align*}
\end{example}

\newpage

\begin{example}
Find an integrating factor for the equation
	\begin{align*}
	2xy^3 dx + (3x^2y^2 + x^2 y^3 + 1) dy = 0 
	\end{align*}
and solve the equation.
\end{example}

\newpage

\phantom{2}

\end{document}
\documentclass[12pt,a4paper]{article}
\usepackage[utf8]{inputenc}
\usepackage[english]{babel}

\usepackage{amsmath}
\usepackage{amsfonts}
\usepackage{amssymb}

\usepackage{graphicx}
\usepackage{lmodern}
\usepackage{tikz}
\usepackage{titlesec}
\usepackage{environ}
\usepackage{xcolor}
\usepackage{fancyhdr}
\usepackage[colorlinks = true, linkcolor = black]{hyperref}
\usepackage{xparse}
\usepackage{enumerate}

\usepackage[left=2cm,right=2cm,top=2cm,bottom=2cm]{geometry}
\usepackage{multicol}
\usepackage[indent=0pt]{parskip}

\newcommand{\spaceP}{\vspace*{0.5cm}}
\newcommand{\Span}{\mathrm{Span}\,}
\newcommand{\range}{\mathrm{range}\,}

%% Redefining sections
\newcommand{\sectionformat}[1]{%
    \begin{tikzpicture}[baseline=(title.base)]
        \node[rectangle, draw] (title) {#1};
    \end{tikzpicture}
    
    \noindent\hrulefill
}

% default values copied from titlesec documentation page 23
% parameters of \titleformat command are explained on page 4
\titleformat%
    {\section}% <command> is the sectioning command to be redefined, i. e., \part, \chapter, \section, \subsection, \subsubsection, \paragraph or \subparagraph.
    {\normalfont\large\scshape}% <format>
    {}% <label> the number
    {0em}% <sep> length. horizontal separation between label and title body
    {\centering\sectionformat}% code preceding the title body  (title body is taken as argument)

%% Set counters for sections to none
\setcounter{secnumdepth}{0}

%% Set the footer/headers
\pagestyle{fancy}
\fancyhf{}
\renewcommand{\headrulewidth}{0pt}
\renewcommand{\footrulewidth}{2pt}
\lfoot{P.-O. Paris{\'e}}
\cfoot{MATH 302}
\rfoot{Page \thepage}

%% Defining example environment
\newcounter{example}[section]
\NewEnviron{example}%
	{%
	\noindent\refstepcounter{example}\fcolorbox{gray!40}{gray!40}{\textsc{\textcolor{red}{Example~\theexample.}}}%
	%\fcolorbox{black}{white}%
		{  %\parbox{0.95\textwidth}%
			{
			\BODY
			}%
		}%
	}

% Theorem environment
\NewEnviron{theorem}%
	{%
	\noindent\refstepcounter{example}\fcolorbox{gray!40}{gray!40}{\textsc{\textcolor{blue}{Theorem~\theexample.}}}%
	%\fcolorbox{black}{white}%
		{  %\parbox{0.95\textwidth}%
			{
			\BODY
			}%
		}%
	}
	

%%%%
\begin{document}
\thispagestyle{empty}

\begin{center}
\vspace*{2.5cm}

{\Huge \textsc{Math 302}}

\vspace*{2cm}

{\LARGE \textsc{Chapter 2}} 

\vspace*{0.75cm}

\noindent\textsc{Section 2.1: Linear First Order Differential Equation}

\vspace*{0.75cm}

\tableofcontents

\vfill

\noindent \textsc{Created by: Pierre-Olivier Paris{\'e}} \\
\textsc{Fall 2022}
\end{center}

\newpage

\section{What Is A LFODE?}

A first order ODE is said to be \textbf{linear} (abbreviated LFODE) if it can be written as
	\begin{align}
	y' + p(x) y = f(x) . \label{Eq:LFODE}
	\end{align}
	\begin{itemize}
	\item Example: $y' + 3y/x^2 = 1$.
	\item Example: $xy' - 8x^2 y = \sin x$.
	\end{itemize}
\vspace*{10pt}

\subsection{More Terminology}
\begin{itemize}
\item A first order ODE that is not of the form \eqref{Eq:LFODE}, then the ODE is said to be \textbf{nonlinear}.
	\begin{itemize}
	\item Example: $xy' + 3y^2 = 2x$.
	\item Example: $yy' + e^y = \tan (xy)$.
	\end{itemize}
\item When $f (x) = 0$ for any $x$, then $y' + p(x) y = 0$ is said to be \textbf{homogeneous}.
	\begin{itemize}
	\item Example: $y' + 3y/x^2 = 0$.
	\item Example: $xy' - 8x^2 y = 0$.
	\end{itemize}
\item When $f(x)$ is not zero, then the LODE is said to be \textbf{nonhomogeneous}.
\end{itemize}

\newpage

\section{General Solution to a LFODE}

\begin{example}
Find all the solutions to
	\begin{align*}
	y' = \frac{1}{x^2}
	\end{align*}
\end{example}

\vfill

\subsection{General Solution}

We say that a function $y = y(x, c)$ is a \textbf{general solution} to \eqref{Eq:LFODE} if
	\begin{itemize}
	\item For each fixed parameter $c$, the resulting function $y = y(x, c)$ is a solution to \eqref{Eq:LFODE} on an an open interval $(a, b)$.
	\item If $y_1 = y_1 (x)$ is a solution to \eqref{Eq:LFODE} on $(a, b)$, then $y_1$ can be obtained from the formula $y = y (x, c)$ by choosing $c$ appropriately.
	\end{itemize}
	
\vfill

\newpage

\section{Homogeneous LFODE}

We now find the general solution to
	\begin{align}
	y' + p(x) y = 0 \label{Eq:HLFODE}
	\end{align}
where $p$ is continuous on an interval $(a, b)$.

\vspace*{16pt}

\begin{example}
Let $a$ be a constant (fixed).
	\begin{enumerate}
	\item Find the general solution of $y'- a y = 0$.
	\item Solve the initial value problem
		\begin{align*}
		y' - ay = 0 , \quad y(x_0) = y_0 .
		\end{align*}
	\end{enumerate}
\end{example}

\newpage

\phantom{2}

\newpage

\begin{example}
	\begin{enumerate}
	\item Find the general solution of $xy' + y = 0$.
	\item Solve the initial value problem
		\begin{align*}
		xy' + y = 0, \quad y (1) = 3 .
		\end{align*}
	\end{enumerate}
\end{example}

\newpage

\phantom{2}

\vfill

\underline{General facts:} 

	\begin{itemize}
	\item The general solution to \eqref{Eq:HLFODE} is given by
	\begin{align*}
	y = c e^{-P(x)}
	\end{align*}
where $P(x) = \displaystyle\int p(x) \, dx$ is any antiderivative of $p(x)$.
	\item The solution to the IVP
		\begin{align*}
		y' + p(x) y = 0 , \quad y(x_0) = y_0
		\end{align*}
	is given by
		\begin{align*}
		y(x) = y_0 e^{-\int_{x_0}^x p(x) \, dx} .
		\end{align*}
	\end{itemize}
	
\newpage

\section{Nonhomogeneous LFODE}
We now want to find the general solution to
	\begin{align*}
	y' + p(x) y = f(x)
	\end{align*}
where the functions $p(x)$ and $f(x)$ are continuous on an open interval $(a, b)$.

\underline{Remark:}
	\begin{itemize}
	\item The homogeneous part $y' + p(x) y = 0$ is called the \textbf{complementary equation}.
	\end{itemize}
	
	\vspace*{16pt}
	
	\begin{example}
	Find the general solution of
		\begin{align*}
		y' + 2y = x^3 e^{-2x} .
		\end{align*}
	\end{example}
	
\newpage

\subsection{Summary of The Method}

	\begin{itemize}
	\item Find a function $y_1$ such that $y_1' + p(x) y_1' = 0$
	\item Write $y = u y_1$ where $u$ is an unknown function.
	\item Solve $u' y_1 = f(x)$.
	\item Substitute $u$ in $y$.
	\end{itemize}
	
\vspace*{16pt}

\begin{example}
\begin{enumerate}
\item Find the general solution
	\begin{align*}
	y' + (\cot x) y = x \csc x .
	\end{align*}
\item Solve the initial value problem
	\begin{align*}
	y' + (\cot x) y = x \csc x , \quad y(\pi/2 ) = 1 .
	\end{align*}
\end{enumerate}
\end{example}

\newpage

\phantom{2}

\newpage

\subsection{General Theorem}
Suppose
	\begin{itemize}
	\item $p(x)$ and $f(x)$ are continuous on an interval $(a, b)$ 	\item $y_1$ is a solution to the complementary equation.
	\end{itemize}
Then the general solution to $y' + p(x) y = f(x)$ is
	\begin{align*}
	y (x) = y_1 (x) \left( c + \int \frac{f(x)}{y_1 (x)} \, dx \right) 
	\end{align*}
for each $x$ in $(a, b)$.
	
	\vspace*{16pt}
	
\subsection{Existence Theorem}
Suppose
	\begin{itemize}
	\item $p(x)$ and $f(x)$ are continuous on an interval $(a, b)$.
	\item $y_1$ is a solution to the complementary equation.
	\item $x_0$ is an arbitrary number in $(a, b)$ and $y_0$ is an arbitrary number.
	\end{itemize}
Then the boundary value problem
	\begin{align*}
	y' + p(x) y + f(x) , \quad y(x_0) = y_0
	\end{align*}
has a unique solution which is of the form
	\begin{align*}
	y (x) = y_1 (x) \left( \frac{y_0}{y_1 (x_0)} + \int_{x_0}^x \frac{f(t)}{y_1 (t)} \, dt \right)
	\end{align*}
for each $x$ in $(a, b)$.
\end{document}
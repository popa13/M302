\documentclass[12pt,a4paper]{article}
\usepackage[utf8]{inputenc}
\usepackage[english]{babel}

\usepackage{amsmath}
\usepackage{amsfonts}
\usepackage{amssymb}

\usepackage{graphicx}
\usepackage{lmodern}
\usepackage{tikz}
\usepackage{titlesec}
\usepackage{environ}
\usepackage{xcolor}
\usepackage{fancyhdr}
\usepackage[colorlinks = true, linkcolor = black]{hyperref}
\usepackage{xparse}
\usepackage{enumitem}

\usepackage[left=2cm,right=2cm,top=2cm,bottom=2cm]{geometry}
\usepackage{multicol}
\usepackage[indent=0pt]{parskip}

\newcommand{\spaceP}{\vspace*{0.5cm}}
\newcommand{\Span}{\mathrm{Span}\,}
\newcommand{\range}{\mathrm{range}\,}
\newcommand{\ra}{\rightarrow}

%% Redefining sections
\newcommand{\sectionformat}[1]{%
    \begin{tikzpicture}[baseline=(title.base)]
        \node[rectangle, draw] (title) {#1};
    \end{tikzpicture}
    
    \noindent\hrulefill
}

% default values copied from titlesec documentation page 23
% parameters of \titleformat command are explained on page 4
\titleformat%
    {\section}% <command> is the sectioning command to be redefined, i. e., \part, \chapter, \section, \subsection, \subsubsection, \paragraph or \subparagraph.
    {\normalfont\large\scshape}% <format>
    {}% <label> the number
    {0em}% <sep> length. horizontal separation between label and title body
    {\centering\sectionformat}% code preceding the title body  (title body is taken as argument)

%% Set counters for sections to none
\setcounter{secnumdepth}{0}

%% Set the footer/headers
\pagestyle{fancy}
\fancyhf{}
\renewcommand{\headrulewidth}{0pt}
\renewcommand{\footrulewidth}{2pt}
\lfoot{P.-O. Paris{\'e}}
\cfoot{MATH 302}
\rfoot{Page \thepage}

%% Defining example environment
\newcounter{example}[section]
\NewEnviron{example}%
	{%
	\noindent\refstepcounter{example}\fcolorbox{gray!40}{gray!40}{\textsc{\textcolor{red}{Example~\theexample.}}}%
	%\fcolorbox{black}{white}%
		{  %\parbox{0.95\textwidth}%
			{
			\BODY
			}%
		}%
	}

% Theorem environment
\NewEnviron{theorem}%
	{%
	\noindent\refstepcounter{example}\fcolorbox{gray!40}{gray!40}{\textsc{\textcolor{blue}{Theorem~\theexample.}}}%
	%\fcolorbox{black}{white}%
		{  %\parbox{0.95\textwidth}%
			{
			\BODY
			}%
		}%
	}
	

%%%%
\begin{document}
\thispagestyle{empty}

\begin{center}
\vspace*{2.5cm}

{\Huge \textsc{Math 302}}

\vspace*{2cm}

{\LARGE \textsc{Chapter 4}} 

\vspace*{0.75cm}

\noindent\textsc{Section 4.2: Cooling and Mixing}

\vspace*{0.75cm}

\tableofcontents

\vfill

\noindent \textsc{Created by: Pierre-Olivier Paris{\'e}} \\
\textsc{Fall 2022}
\end{center}

\newpage

\section{Newton's Law of Cooling: A Rematch}

Recall that Newton's law of cooling is given by
	\begin{align}
	T' = -k (T - T_m ) \label{Eq:NewtonLawCool}
	\end{align}
where $k > 0$ is a constant, $T$ is the temperature of the object, and $T_m$ is the temperature of the medium (surrounding).

\begin{example}
Find the solution to \eqref{Eq:NewtonLawCool} subject to the additional condition $T_0 = T(0)$.
\end{example}

\newpage

\begin{example}
A ceramic insulator is baked at $400^{\circ}\text{C}$ and cooled in a room in which the temperature is $25^{\circ}\text{C}$. After $4$ minutes the temperature of the insulator is $200^{\circ} \text{C}$. What is its temperature after $8$ minutes?
\end{example}

\newpage

\section{Mixing Problems}


\begin{example}
A tank initially contains $40$ pounds of salt dissolved in $600$ gallons of water. Starting at $t_0 = 0$, water that contains $1/2$ pound of salt per gallon is poured into the tank at the rate of $4 \text{gal}/\text{min}$ and the mixture is drained from the tank at the same rate. We assume that the mixture is stirred instantly so that the salt is always uniformly distributed throughout the mixture.
\begin{enumerate}
\item Find a differential equation for the quantity $Q(t)$ of salt in the tank at time $t > 0$, and solve the equation to determine $Q(t)$.
\item Find $\lim_{t \rightarrow \infty} Q(t)$.
\end{enumerate}
\end{example}

\newpage

\phantom{2}

\end{document}
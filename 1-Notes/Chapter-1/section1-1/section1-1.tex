\documentclass[12pt,a4paper]{article}
\usepackage[utf8]{inputenc}
\usepackage[english]{babel}

\usepackage{amsmath}
\usepackage{amsfonts}
\usepackage{amssymb}

\usepackage{graphicx}
\usepackage{lmodern}
\usepackage{tikz}
\usepackage{titlesec}
\usepackage{environ}
\usepackage{xcolor}
\usepackage{fancyhdr}
\usepackage[colorlinks = true, linkcolor = black]{hyperref}
\usepackage{xparse}
\usepackage{enumerate}

\usepackage[left=2cm,right=2cm,top=2cm,bottom=2cm]{geometry}
\usepackage{multicol}
\usepackage[indent=0pt]{parskip}

\newcommand{\spaceP}{\vspace*{0.5cm}}
\newcommand{\Span}{\mathrm{Span}\,}
\newcommand{\range}{\mathrm{range}\,}

%% Redefining sections
\newcommand{\sectionformat}[1]{%
    \begin{tikzpicture}[baseline=(title.base)]
        \node[rectangle, draw] (title) {#1};
    \end{tikzpicture}
    
    \noindent\hrulefill
}

% default values copied from titlesec documentation page 23
% parameters of \titleformat command are explained on page 4
\titleformat%
    {\section}% <command> is the sectioning command to be redefined, i. e., \part, \chapter, \section, \subsection, \subsubsection, \paragraph or \subparagraph.
    {\normalfont\large\scshape}% <format>
    {}% <label> the number
    {0em}% <sep> length. horizontal separation between label and title body
    {\centering\sectionformat}% code preceding the title body  (title body is taken as argument)

%% Set counters for sections to none
\setcounter{secnumdepth}{0}

%% Set the footer/headers
\pagestyle{fancy}
\fancyhf{}
\renewcommand{\headrulewidth}{0pt}
\renewcommand{\footrulewidth}{2pt}
\lfoot{P.-O. Paris{\'e}}
\cfoot{MATH 302}
\rfoot{Page \thepage}

%% Defining example environment
\newcounter{example}[section]
\NewEnviron{example}%
	{%
	\noindent\refstepcounter{example}\fcolorbox{gray!40}{gray!40}{\textsc{\textcolor{red}{Example~\theexample.}}}%
	%\fcolorbox{black}{white}%
		{  %\parbox{0.95\textwidth}%
			{
			\BODY
			}%
		}%
	}

% Theorem environment
\NewEnviron{theorem}%
	{%
	\noindent\refstepcounter{example}\fcolorbox{gray!40}{gray!40}{\textsc{\textcolor{blue}{Theorem~\theexample.}}}%
	%\fcolorbox{black}{white}%
		{  %\parbox{0.95\textwidth}%
			{
			\BODY
			}%
		}%
	}
	

%%%%
\begin{document}
\thispagestyle{empty}

\begin{center}
\vspace*{2.5cm}

{\Huge \textsc{Math 302}}

\vspace*{2cm}

{\LARGE \textsc{Chapter 1}} 

\vspace*{0.75cm}

\noindent\textsc{Section 1.1: Applications Leading to DEs}

\vspace*{0.75cm}

\tableofcontents

\vfill

\noindent \textsc{Created by: Pierre-Olivier Paris{\'e}} \\
\textsc{Fall 2022}
\end{center}

\newpage

\section{Little experiment}

\begin{example}
Poor some hot water in a teapod and take its temperature with a thermometer. Take the temperature every 5 minutes. Record your data in a table and plot them in a Times VS Temperature graph. \end{example}

\underline{\textsc{Tables}}

\begin{multicols}{2}
\begin{center}
\begin{tabular}{c|c}
Time & Temperature \\\hline\hline
\phantom{222222222} & \phantom{222222222} \\\hline
\phantom{222222222} & \phantom{222222222} \\\hline
\phantom{222222222} & \phantom{222222222} \\\hline
\phantom{222222222} & \phantom{222222222} \\\hline
\phantom{222222222} & \phantom{222222222} \\\hline
\phantom{222222222} & \phantom{222222222} \\\hline
\phantom{222222222} & \phantom{222222222} \\\hline
\phantom{222222222} & \phantom{222222222} \\\hline
\phantom{222222222} & \phantom{222222222} \\\hline
\phantom{222222222} & \phantom{222222222} \\\hline
\phantom{222222222} & \phantom{222222222} \\\hline
\phantom{222222222} & \phantom{222222222} \\\hline
\phantom{222222222} & \phantom{222222222} \\\hline
\phantom{222222222} & \phantom{222222222} \\
\end{tabular}
\end{center}

\begin{center}
\begin{tabular}{c|c}
Time & Temperature \\\hline\hline
\phantom{222222222} & \phantom{222222222} \\\hline
\phantom{222222222} & \phantom{222222222} \\\hline
\phantom{222222222} & \phantom{222222222} \\\hline
\phantom{222222222} & \phantom{222222222} \\\hline
\phantom{222222222} & \phantom{222222222} \\\hline
\phantom{222222222} & \phantom{222222222} \\\hline
\phantom{222222222} & \phantom{222222222} \\\hline
\phantom{222222222} & \phantom{222222222} \\\hline
\phantom{222222222} & \phantom{222222222} \\\hline
\phantom{222222222} & \phantom{222222222} \\\hline
\phantom{222222222} & \phantom{222222222} \\\hline
\phantom{222222222} & \phantom{222222222} \\\hline
\phantom{222222222} & \phantom{222222222} \\\hline
\phantom{222222222} & \phantom{222222222} \\
\end{tabular}
\end{center}
\end{multicols}


\underline{\textsc{Plots}}

\newpage

\section{Newton's Law of Cooling}

\begin{example}
Let $T = T(t)$ be the temperature of a body at time $t$ and let $T_m$ be the temperature of its surrounding. Assuming that
	\begin{itemize}
	\item the rate of cooling of the body is directly proportial to the temperature difference of the surface area exposed
	\item the temperature of the surrounding does not change
	\end{itemize}
deduce a model describing the evolution of the temperature $T(t)$ of the body.
\end{example}

\newpage

\section{Second Version of Newton's Law of Cooling}
Assuming that the medium (surrounding) remains at constant temperature seems reasonable if we're considering a cup of tea/coffee cooling in a room.

What if the body warms or cools its surrounding, resulting in changing drastically the surrounding temperature?

\vspace*{12pt}

\begin{example}
Let $T = T(t)$ be the temperature of the body at time $t$ and let $T_m = T_m (t)$ be the temperature of its surrounding. Assuming that
	\begin{itemize}
	\item the rate of cooling of the body is directly proportial to the temperature difference of the surface area exposed
	\item the energy is preserved
	\end{itemize}
deduce a model describing the evolution of the temperature $T(t)$ of the body.
\end{example}

\newpage

\phantom{2}

\end{document}
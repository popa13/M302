\documentclass[12pt,a4paper]{article}
\usepackage[utf8]{inputenc}
\usepackage[english]{babel}

\usepackage{amsmath}
\usepackage{amsfonts}
\usepackage{amssymb}

\usepackage{graphicx}
\usepackage{lmodern}
\usepackage{tikz}
\usepackage{titlesec}
\usepackage{environ}
\usepackage{xcolor}
\usepackage{fancyhdr}
\usepackage[colorlinks = true, linkcolor = black]{hyperref}
\usepackage{xparse}
\usepackage{enumerate}

\usepackage[left=2cm,right=2cm,top=2cm,bottom=2cm]{geometry}
\usepackage{multicol}
\usepackage[indent=0pt]{parskip}

\newcommand{\spaceP}{\vspace*{0.5cm}}
\newcommand{\Span}{\mathrm{Span}\,}
\newcommand{\range}{\mathrm{range}\,}

%% Redefining sections
\newcommand{\sectionformat}[1]{%
    \begin{tikzpicture}[baseline=(title.base)]
        \node[rectangle, draw] (title) {#1};
    \end{tikzpicture}
    
    \noindent\hrulefill
}

% default values copied from titlesec documentation page 23
% parameters of \titleformat command are explained on page 4
\titleformat%
    {\section}% <command> is the sectioning command to be redefined, i. e., \part, \chapter, \section, \subsection, \subsubsection, \paragraph or \subparagraph.
    {\normalfont\large\scshape}% <format>
    {}% <label> the number
    {0em}% <sep> length. horizontal separation between label and title body
    {\centering\sectionformat}% code preceding the title body  (title body is taken as argument)

%% Set counters for sections to none
\setcounter{secnumdepth}{0}

%% Set the footer/headers
\pagestyle{fancy}
\fancyhf{}
\renewcommand{\headrulewidth}{0pt}
\renewcommand{\footrulewidth}{2pt}
\lfoot{P.-O. Paris{\'e}}
\cfoot{MATH 302}
\rfoot{Page \thepage}

%% Defining example environment
\newcounter{example}[section]
\NewEnviron{example}%
	{%
	\noindent\refstepcounter{example}\fcolorbox{gray!40}{gray!40}{\textsc{\textcolor{red}{Example~\theexample.}}}%
	%\fcolorbox{black}{white}%
		{  %\parbox{0.95\textwidth}%
			{
			\BODY
			}%
		}%
	}

% Theorem environment
\NewEnviron{theorem}%
	{%
	\noindent\refstepcounter{example}\fcolorbox{gray!40}{gray!40}{\textsc{\textcolor{blue}{Theorem~\theexample.}}}%
	%\fcolorbox{black}{white}%
		{  %\parbox{0.95\textwidth}%
			{
			\BODY
			}%
		}%
	}
	

%%%%
\begin{document}
\thispagestyle{empty}

\begin{center}
\vspace*{2.5cm}

{\Huge \textsc{Math 302}}

\vspace*{2cm}

{\LARGE \textsc{Chapter 1}} 

\vspace*{0.75cm}

\noindent\textsc{Section 1.2: Basic Concepts}

\vspace*{0.75cm}

\tableofcontents

\vfill

\noindent \textsc{Created by: Pierre-Olivier Paris{\'e}} \\
\textsc{Fall 2022}
\end{center}

\newpage

\section{What's a DE?}

\begin{itemize}
\item A \textbf{differential equation} (abbreviated by DE) is an equation that contains one or more derivatives of an unknown function.
	\begin{itemize}
	\item Examples: $T' = -k (T - T_m)$, $y' = x^2$, $x^2 y'' + xy' + 2 = 0$.
	\end{itemize}
\item The \textbf{order} of a DE is the order of the highest derivatives that it contains.
	\begin{itemize}
	\item Example: $y' = x^2$ is of order \underline{\phantom{222222}}.
	\item Example: $x^2 y'' + xy' + 2 = 0$ is of order \underline{\phantom{222222}}.
	\end{itemize}
\item An \textbf{Ordinary Differential Equation} (abbreviated ODE) is a DE involving an unknown function of only one variable.
\item An \textbf{Partial Differential Equation} (abbreviated PDE) is a DE involving an unknown function of more than one variable.
\end{itemize}

\vspace*{16pt}

The simplest ODE is of the form
	\begin{align*}
	y' = f(x) \quad \text{ or } \quad y^{(n)} = f(x)
	\end{align*}
where $f$ is a known function of $x$.

\vspace*{10pt}

\begin{example}
Find functions $y = y (x)$ satisfying
	\begin{enumerate}
	\item $y' = x^2$.
	\item $y'' = \cos (x)$.
	\end{enumerate}
\end{example}

\vfill

Our goal is to study general ODEs of the form
	\begin{align*}
	y^{(n)} = f(x, y, y' , \ldots , y^{(n-1)}) .
	\end{align*}
	
\newpage

\section{What Is a Solution to an ODE?}

A \textbf{solution} to the ODE 
	$$
	y^{(n)} (x) = f(x, y(x), y'(x), \ldots , y^{(n-1)}(x))
	$$
is a function $y = y(x)$ that verifies the ODE for any $x$ in some open interval $(a, b)$.

	\vspace*{10pt}
	\underline{Remark:} 
	
	\begin{itemize}
	\item Functions that satisfy an ODE at isolated points are not considered solutions.
	\end{itemize}
	
\vspace*{16pt}
	
\begin{example}\label{Ex:SolutionODE}
Verify that
	\begin{align*}
	y = \frac{x^2}{3} + \frac{1}{x}
	\end{align*}
is a solution of
	\begin{align*}
	xy' + y = x^2
	\end{align*}
on $(-\infty , 0)$ and $(0, \infty )$.
\end{example}

\newpage

\subsection{Solution and Integral Curves}

\begin{itemize}
\item The graph of a solution of an ODE is a \textbf{solution curve}.
\item More generally, a curve $C$ in the plane is said to be an \textbf{integral curve} of an ODE if every function $y = y(x)$ whose graph is a segment of $C$ is a solution of the ODE. 
\end{itemize}

\vspace*{16pt}

\begin{example}
Plot the solutions obtained in Example \ref{Ex:SolutionODE}. Are they solution curves of the ODE?
\end{example}

\vspace*{20pt}

\begin{example}
If $a$ is any positive constant, check that the circle
	\begin{align*}
	x^2 + y^2 = a^2
	\end{align*}
is an integral curve of $y' = -x/y$.
\end{example}

\newpage

\section{Initial Value Problems}

\begin{example}
Find a solution of
	\begin{align*}
	y' = x^3
	\end{align*}
satisfying the additional condition $y(1) = 2$.
\end{example}

\vspace*{6cm}

\begin{example}
All the solutions to
	\begin{align*}
	y'' - 2y' + 3y = 0
	\end{align*}
are the functions
	\begin{align*}
	y (x) = c_1 e^{x} + c_2 e^{-3x} 
	\end{align*}
where $c_1$, $c_2$ are arbitrary constants. Find the solution that satisfies $y(0) = 1$ and $y' (0) = 0$.
\end{example}

\newpage

An \textbf{Initial Value Problem} (abbreviated by IVP) is an ODE with additional \textbf{Initial conditions}. The general form of an IVP is
	\begin{align*}
	y^{(n)}(x) = f(x, y(x) , y'(x) , \ldots , y^{(n-1)}(x)), \quad y(x_0) = k_0 , \, y' (x_0) = k_1 , \ldots , \, y^{(n-1)}(x_0) = k_{n-1} .
	\end{align*}
	
\begin{itemize}
\item The largest open interval that contains $x_0$ on which $y(x)$ is defined and satisfies the ODE is called the \textbf{interval of validity} of $y$.
\end{itemize}

\vspace*{16pt}

\begin{example}
Find the interval of validity of the solution to
	\begin{align*}
	y' = x^3 , \, y(1) = 2 .
	\end{align*}
\end{example}

\newpage

\begin{example}
Find the interval of validity of the solution to the following IVPs:
	\begin{enumerate}
	\item $xy' + y = x^2$, $y (1) = 4/3$.
	\item $xy' + y = x^2$, $y (-1) = -2/3$.
	\end{enumerate}
\end{example}


\end{document}
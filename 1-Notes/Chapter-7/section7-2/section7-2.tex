\documentclass[12pt,a4paper]{article}
\usepackage[utf8]{inputenc}
\usepackage[english]{babel}

\usepackage{amsmath}
\usepackage{amsfonts}
\usepackage{amssymb}

\usepackage{graphicx}
\usepackage{lmodern}
\usepackage{tikz}
\usepackage{titlesec}
\usepackage{environ}
\usepackage{xcolor}
\usepackage{fancyhdr}
\usepackage[colorlinks = true, linkcolor = black]{hyperref}
\usepackage{xparse}
\usepackage{enumitem}
\usepackage{comment}
\usepackage{wrapfig}
\usepackage[capitalise]{cleveref}

\usepackage[left=2cm,right=2cm,top=2cm,bottom=2cm]{geometry}
\usepackage{multicol}
\usepackage[indent=0pt]{parskip}

\newcommand{\spaceP}{\vspace*{0.5cm}}
\newcommand{\Span}{\mathrm{Span}\,}
\newcommand{\range}{\mathrm{range}\,}
\newcommand{\ra}{\rightarrow}

%% Redefining sections
\newcommand{\sectionformat}[1]{%
    \begin{tikzpicture}[baseline=(title.base)]
        \node[rectangle, draw] (title) {#1};
    \end{tikzpicture}
    
    \noindent\hrulefill
}

% default values copied from titlesec documentation page 23
% parameters of \titleformat command are explained on page 4
\titleformat%
    {\section}% <command> is the sectioning command to be redefined, i. e., \part, \chapter, \section, \subsection, \subsubsection, \paragraph or \subparagraph.
    {\normalfont\large\scshape}% <format>
    {}% <label> the number
    {0em}% <sep> length. horizontal separation between label and title body
    {\centering\sectionformat}% code preceding the title body  (title body is taken as argument)

%% Set counters for sections to none
\setcounter{secnumdepth}{0}

%% Set the footer/headers
\pagestyle{fancy}
\fancyhf{}
\renewcommand{\headrulewidth}{0pt}
\renewcommand{\footrulewidth}{2pt}
\lfoot{P.-O. Paris{\'e}}
\cfoot{MATH 302}
\rfoot{Page \thepage}

%% Defining example environment
\newcounter{example}[section]
\NewEnviron{example}%
	{%
	\noindent\refstepcounter{example}\fcolorbox{gray!40}{gray!40}{\textsc{\textcolor{red}{Example~\theexample.}}}%
	%\fcolorbox{black}{white}%
		{  %\parbox{0.95\textwidth}%
			{
			\BODY
			}%
		}%
	}

% Theorem environment
\NewEnviron{theorem}%
	{%
	\noindent\refstepcounter{example}\fcolorbox{gray!40}{gray!40}{\textsc{\textcolor{blue}{Theorem~\theexample.}}}%
	%\fcolorbox{black}{white}%
		{  %\parbox{0.95\textwidth}%
			{
			\BODY
			}%
		}%
	}

\NewEnviron{notes}%
	{%
	\noindent \fcolorbox{gray!40}{gray!40}{\textsc{\textcolor{blue}{Solution.}}}%
	%\fcolorbox{black}{white}%
		{  %\parbox{0.95\textwidth}%
			{
			\textcolor{blue}{%
			\BODY
			}
			}%
		}%
	}
%%% Ignorer les notes
\excludecomment{notes}

%%%%
\begin{document}
\thispagestyle{empty}

\begin{center}
\vspace*{2.5cm}

{\Huge \textsc{Math 302}}

\vspace*{2cm}

{\LARGE \textsc{Chapter 7}} 

\vspace*{0.75cm}

\noindent\textsc{Section 7.2: Series Solutions Near an Ordinary Point}

\vspace*{0.75cm}

\tableofcontents

\vfill

\noindent \textsc{Created by: Pierre-Olivier Paris{\'e}} \\
\textsc{Fall 2022}
\end{center}

\newpage

\section{Setup}
Main goal:

	\begin{itemize}
	\item Solve a second order ODE
		\begin{align*}
		A(x) y'' + B(x) y' + C(x) y = 0
		\end{align*}
	where $A(x)$, $B(x)$, and $C(x)$ are polynomials.
	\item Use power series to obtain the solution $y(x)$. Such a solution is called a \textbf{power series solution} to the ODE.
	\end{itemize}
	
Recall from the previous section that
	\begin{itemize}
	\item $\displaystyle y(x) = \sum_{n = 0}^\infty a_n x^n$.
	\item $\displaystyle y'(x) = \sum_{n = 1}^\infty n a_n x^{n-1}$.
	\item $\displaystyle y''(x) = \sum_{n = 2}^\infty n (n - 1) a_n x^{n-2}$.
	\end{itemize}
	
\underline{Remark:}
	\begin{itemize}
	\item We denote the left-hand side by
		\begin{align*}
		L(y) := A(x) y'' + B(x) y' + C(x) y .
		\end{align*}
	\item The application $y \mapsto L(y)$ is called a \textbf{differential operator} in the litterature. 
	\end{itemize}

\vspace*{16pt}

\begin{example}
Find a power series solution to $y'' + y = 0$.
\end{example}

\newpage

\phantom{2}

\newpage

\phantom{2}

\vfill

\underline{\textbf{Recurrence Relation:}}
	
Solving ODE with power series involves a lot of recurrence relations. In the above problems, we encountered:
	\vspace*{2cm}

\newpage

\begin{example}\label{Ex:SecondEx}
Find a power series solution to $x^2 y'' + y = 0$.
\end{example}

\newpage

\phantom{2}

\newpage

\section{Ordinary and Singular Points}

\begin{itemize}
\item A number $x_0$ is called an \textbf{ordinary point} if $A(x_0) \neq 0$.
\item A number $x_0$ is called a \textbf{singular point} if $A (x_0) = 0$.
\end{itemize}

We will mainly focuss on power series solutions centered at ordinary points.

\vspace*{20pt}

\begin{example}
For each of the following ODEs, find the singular points.
	\begin{enumerate}[label=\textbf{(\alph*)}]
	\item $(1 - x^2) y'' + y = 0$.
	\item $(1 + 2x + x^2) y'' + y' + (2 + x) y = 0$.
	\item $(2x + 3x^2 + x^3) y'' + (x + 1) y' + (x^2 + 1) y = 0$.
	\end{enumerate}
\end{example}

\vfill

\underline{Remark:}
	\begin{itemize}
	\item A power series solution must be centered at an ordinary point, that is, if $x_0$ is an ordinary point, then the form of the solution is
		\begin{align*}
		y(x) = \sum_{n = 0}^\infty a_n (x - x_0)^n .
		\end{align*}
	\item In \cref{Ex:SecondEx}, we see why we can't solve: The power series used was centered at $x_0 = 0$, a singular point.
	\item In the case of a singular points, we need the Frobenius method. This is covered in a second class in ODE.
	\end{itemize}
	
	\newpage
	
	\begin{example}
	\begin{enumerate}[label=\textbf{(\alph*)}]
	\item Find a power series solution of
		\begin{align*}
		(x^2 - 4) y'' + 3xy + y = 0 .
		\end{align*}
	\item Find the solution to the IVP
		\begin{align*}
		(x^2 - 4)y'' + 3xy + y = 0 , \quad y(0) = 4 , \, y'(0) = 1 . 
		\end{align*}
	\end{enumerate}
	\end{example}
	
	\newpage
	
	\phantom{2}
	
	\newpage
	
	\phantom{2}
	
	\newpage
	
	\section{Translating to Success!}
	
	\begin{example}\label{Ex:TranslationExample}
	Find a power series solution to the following IVP:
		\begin{align*}
		(t^2 - 2t - 3) \frac{d^2 y}{dt^2} + 3 (t - 1) \frac{dy}{dt} + y = 0, \quad y(1) = 4 , \, y'(1) = -1 .
		\end{align*}
	\end{example}
	
	\newpage
	
	\phantom{2}
	
	\newpage
	
	\section{Radius of Convergence}
	It is important to know where our solution is valid.
		
		\begin{itemize}
		\item The \textbf{radius of convergence} of a power series $\sum_{n = 0}^\infty a_n (x - x_0)^n$ is the number $R$ such that
			\begin{itemize}
			\item $\sum_{n = 0}^\infty a_n (x - x_0)^n$ converges for any $x$ such that $|x - x_0| < R$.
			\item $\sum_{n = 0}^\infty a_n (x - x_0)^n$ diverges for all $x$ such that $|x - x_0| > R$.
			\end{itemize}
		\item If the limit
			\begin{align*}
			L := \lim_{n \ra \infty} \left| \frac{a_{n+1}}{a_n} \right|
			\end{align*}
		exists, then the radius of convergence of $\sum_{n = 0}^\infty a_n x^n$ is $R = \frac{1}{L}$.
		\end{itemize}
	
	\vspace*{16pt}
	
	\begin{example}
	Find the radius of convergence of
		\begin{enumerate}[label=\textbf{(\alph*)}]
		\item $f(x) = \sum_{n =0}^\infty x^n$.
		\item $g(x) = \sum_{n = 0}^\infty \frac{1}{n!} x^n$.
		\end{enumerate}
	\end{example}
	
	\newpage
	
	\begin{theorem}
	Suppose that $x_0$ is an ordinary point of the ODE
		\begin{align*}
		A(x) y'' + B(x) y' + C(x) y = 0. 
		\end{align*}
	Then the ODE has a general solution of the form
		\begin{align*}
		y(x) = \sum_{n = 0}^\infty a_n (x - x_0)^n .
		\end{align*}
	The radius of convergence of any such series solution is at least as large as the distance from $x_0$ to the nearest (real or complex) singular point of the ODE.
	\end{theorem}
	
	\vspace*{16pt}
	
	\begin{example}
	Determine the radius of convergence guaranteed by the last Theorem of a series solution of
		\begin{align*}
		(x^2 + 9) y'' + xy' + x^2 y = 0
		\end{align*}
		\begin{enumerate}[label=\textbf{(\alph*)}]
		\item in powers of $x$.
		\item in powers of $x - 4$.
		\end{enumerate}
	\end{example}
	
	\newpage
	
	\section{Taylor Polynomial}
	When we have a solution
		\begin{align*}
		y(x) = \sum_{n = 0}^\infty a_n x^n
		\end{align*}
	of an ODE
		\begin{align*}
		A(x) y'' + B(x) y' + C(x) y = 0,
		\end{align*}
	we can draw an approximation of the solution.
	
	\begin{itemize}
	\item The \textbf{Taylor polynomial} $T_N (x)$, where $N \geq 0$ is an integer, is given by the expression
		\begin{align*}
		T_N (x) = \sum_{n = 0}^N a_n (x - x_0)^n = a_0 + a_1 (x - x_0) + \cdots + a_N (x - x_0)^N .
		\end{align*}
	\item When the power series of $y(x)$ converges on a given interval $I$, we have
		\begin{align*}
		y(x) \approx T_N (x)
		\end{align*}
	for a sufficiently large integer $N$.
	\end{itemize}
	
	\vspace*{16pt}
	
	\begin{example}
	\begin{enumerate}[label=\textbf{(\alph*)}]
	\item Plot the graph of $T_{4} (x)$, $T_{10} (x)$, and $T_{20} (X)$ of the power series representation of $f(x) = \cos (x)$.
	\item Plot the graph of $T_{4}(x)$, $T_{10} (x)$, $T_{20} (x)$ for the power series solution of Example \ref{Ex:TranslationExample}. 
	\end{enumerate}
	\end{example}
	
	\newpage
	
	\phantom{2}
	
	\newpage

\end{document}
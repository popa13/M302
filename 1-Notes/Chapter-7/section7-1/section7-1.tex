\documentclass[12pt,a4paper]{article}
\usepackage[utf8]{inputenc}
\usepackage[english]{babel}

\usepackage{amsmath}
\usepackage{amsfonts}
\usepackage{amssymb}

\usepackage{graphicx}
\usepackage{lmodern}
\usepackage{tikz}
\usepackage{titlesec}
\usepackage{environ}
\usepackage{xcolor}
\usepackage{fancyhdr}
\usepackage[colorlinks = true, linkcolor = black]{hyperref}
\usepackage{xparse}
\usepackage{enumitem}
\usepackage{comment}
\usepackage{wrapfig}
\usepackage[capitalise]{cleveref}

\usepackage[left=2cm,right=2cm,top=2cm,bottom=2cm]{geometry}
\usepackage{multicol}
\usepackage[indent=0pt]{parskip}

\newcommand{\spaceP}{\vspace*{0.5cm}}
\newcommand{\Span}{\mathrm{Span}\,}
\newcommand{\range}{\mathrm{range}\,}
\newcommand{\ra}{\rightarrow}

%% Redefining sections
\newcommand{\sectionformat}[1]{%
    \begin{tikzpicture}[baseline=(title.base)]
        \node[rectangle, draw] (title) {#1};
    \end{tikzpicture}
    
    \noindent\hrulefill
}

% default values copied from titlesec documentation page 23
% parameters of \titleformat command are explained on page 4
\titleformat%
    {\section}% <command> is the sectioning command to be redefined, i. e., \part, \chapter, \section, \subsection, \subsubsection, \paragraph or \subparagraph.
    {\normalfont\large\scshape}% <format>
    {}% <label> the number
    {0em}% <sep> length. horizontal separation between label and title body
    {\centering\sectionformat}% code preceding the title body  (title body is taken as argument)

%% Set counters for sections to none
\setcounter{secnumdepth}{0}

%% Set the footer/headers
\pagestyle{fancy}
\fancyhf{}
\renewcommand{\headrulewidth}{0pt}
\renewcommand{\footrulewidth}{2pt}
\lfoot{P.-O. Paris{\'e}}
\cfoot{MATH 302}
\rfoot{Page \thepage}

%% Defining example environment
\newcounter{example}[section]
\NewEnviron{example}%
	{%
	\noindent\refstepcounter{example}\fcolorbox{gray!40}{gray!40}{\textsc{\textcolor{red}{Example~\theexample.}}}%
	%\fcolorbox{black}{white}%
		{  %\parbox{0.95\textwidth}%
			{
			\BODY
			}%
		}%
	}

% Theorem environment
\NewEnviron{theorem}%
	{%
	\noindent\refstepcounter{example}\fcolorbox{gray!40}{gray!40}{\textsc{\textcolor{blue}{Theorem~\theexample.}}}%
	%\fcolorbox{black}{white}%
		{  %\parbox{0.95\textwidth}%
			{
			\BODY
			}%
		}%
	}

\NewEnviron{notes}%
	{%
	\noindent \fcolorbox{gray!40}{gray!40}{\textsc{\textcolor{blue}{Solution.}}}%
	%\fcolorbox{black}{white}%
		{  %\parbox{0.95\textwidth}%
			{
			\textcolor{blue}{%
			\BODY
			}
			}%
		}%
	}
%%% Ignorer les notes
\excludecomment{notes}

%%%%
\begin{document}
\thispagestyle{empty}

\begin{center}
\vspace*{2.5cm}

{\Huge \textsc{Math 302}}

\vspace*{2cm}

{\LARGE \textsc{Chapter 7}} 

\vspace*{0.75cm}

\noindent\textsc{Section 7.1: Review of Power Series}

\vspace*{0.75cm}

\tableofcontents

\vfill

\noindent \textsc{Created by: Pierre-Olivier Paris{\'e}} \\
\textsc{Fall 2022}
\end{center}

\newpage

\section{Why Power Series?}

Most of the differential equation of order $2$ we have encountered are constant coefficients ODE. In most real-life application, the coefficients will be \textbf{variable coefficients} such as
	\begin{itemize}
	\item \textbf{Bessel's equation} of order $n$:
		\begin{align*}
		x^2 y'' + xy' + (x^2 - n^2) y = 0 .
		\end{align*}
	\item \textbf{Legendre's equation} of order $n$:
		\begin{align*}
		(1 - x^2 ) y'' - 2x y' + n (n + 1) y = 0 .
		\end{align*}
	\end{itemize}
	
The methods we used in chapter 5%\footnote{These lecture notes are now based, most of it, on the book \textit{Differential Equations and Boundary Value Problems: Computing and Modeling.}} 
won't be of use in those situations. This is why we need power series and the \textbf{power series method}.

%\begin{example}
%Find a general solution to $y' + y = 0$.
%\end{example}

\vspace*{2cm}

\section{Basic Definitions}

\begin{itemize}
\item A \textbf{Power series} centered at a number $a$ is an expression involving an infinite sum of powers of $(x - a)$:
	\begin{align*}
	\sum_{n = 0}^\infty a_n (x - a)^n .
	\end{align*}
\item If $a =0$, we simply write
	\begin{align*}
	\sum_{n = 0}^\infty a_n x^n .
	\end{align*}
\end{itemize}

We will confine ourselves to power series centered at $a = 0$.

\newpage

\begin{itemize}
\item A power series \textbf{converges} on an interval $I$ provided that for any $x$ in this interval $I$, the following limit exists
	\begin{align*}
	\lim_{N \ra \infty} \sum_{n = 0}^N a_n x^n .
	\end{align*}
\item If a function $f$ is expressed as a power series on $I$, we then write
	\begin{align*}
	f(x) = \sum_{n= 0}^\infty a_n x^n
	\end{align*}
and called this a \textbf{power series representation} of $f$.
\end{itemize}

\vspace*{20pt}

Some examples of power series representations of some famous\footnote{They made the coverpage of New York Times magazine several times for their influence on the world.} function
	\begin{itemize}
	\item $e^x = $
	\vspace*{16pt}
	\item $\cos x = $
	\vspace*{16pt}	
	\item $\sin x = $
	\vspace*{16pt}	
	\item $\cosh x = $
	\vspace*{16pt}
	\item $\sinh x = $
	\vspace*{16pt}
	\item $\ln (1 + x) = $
	\vspace*{16pt}
	\item $\displaystyle\frac{1}{1 - x} = $
	\vspace*{16pt}
	\item $(1 + x)^\alpha = $
	\end{itemize}
	
\vspace*{20pt}
	
\underline{Remark:} 
\begin{itemize}
\item The \textbf{Taylor series} of $f$ is
	\begin{align*}
	f(a) + f'(a) (x - a) + \frac{f''(a)}{2!} (x - a)^2 + \cdots = \sum_{n = 0}^\infty \frac{f^{(n)}(a)}{n!} (x - a)^n .
	\end{align*}
\item The \textbf{Maclaurin series} of $f$ is
	\begin{align*}
	f(0) + f'(0) x + \frac{f''(0)}{2!} x^2 + \cdots = \sum_{n = 0}^\infty \frac{f^{(n)} (0)}{n!} x^n .
	\end{align*}
\end{itemize}
	
\newpage
	
\section{Calculus Operations with Power Series}

\subsection{Differentiation}
If $f(x) = \sum_{n = 0}^\infty a_n x^n$, then
	\begin{align*}
	f'(x) &= \sum_{n = 1}^\infty n a_n x^{n-1} \\
	f''(x) &= \sum_{n = 2}^\infty n (n - 1) a_n x^{n-2} 
	\end{align*}
and in general
	\begin{align*}
	f^{(k)} (x) &= \sum_{n = k}^\infty n (n - 1) \cdots (n - k + 1) a_n x^{n- k} .
	\end{align*}
	
\begin{example}
Differentiate the power series representation of $\sin x$.
\end{example}


\newpage
	
\subsection{Identity Principle or Uniqueness of Power series}
If $f(x) = \sum_{n = 0}^\infty a_n x^n$ and $g(x) = \sum_{n = 0}^\infty b_n x^n$, then
	\begin{align*}
	f(x) = g(x) \quad \iff \quad a_n = b_n , \text{ for all } n \geq 0 .
	\end{align*}

\vspace*{16pt}
	
\underline{Consequence:}
	We have
		\begin{align*}
		\sum_{n = 0} a_n x^n = 0
		\end{align*}
	if, and only if, $a_n = 0$ for all $n \geq 0$.
	
\vspace*{16pt}

\begin{example}
Find $y(x)$ if
	\begin{align*}
	y' = \sum_{n = 1}^\infty x^n \quad \text{ and } \quad y(0) = 0 .
	\end{align*}
\end{example}

\newpage

\section{Algebraic Operations with Power Series}

\subsection{Sum, Difference and Multiplication by A Constant}
If $f(x) = \sum_{n = 0}^\infty a_n x^n$ and $g(x) = \sum_{n = 0}^\infty b_n x^n$ are two power series, then
	\begin{itemize}
	\item $ \displaystyle f(x) + g(x) = \sum_{n = 0}^\infty (a_n + b_n) x^n$.
	\item $\displaystyle f(x) - g(x) = \sum_{n =0}^\infty (a_n - b_n) x^n$.
	\item $\displaystyle c f(x) = \sum_{n = 0}^\infty (ca_n) x^n$.
	\end{itemize}
	
\vspace*{16pt}
	
\begin{example}
Use the definition of $\cosh (x)$ and $\sinh (x)$ to find its power series representation.
\end{example}

\newpage

\subsection{Product with Polynomials}
Let $f(x) = \sum_{n = 0}^\infty a_n x^n$. 
	\begin{itemize}
	\item $g(x) = cx$, then
		\begin{align*}
		f(x) g(x) &= \Big( \sum_{n = 0}^\infty a_n x^n \Big) cx \\
		&= 
		\end{align*}
	\item $g(x) = cx^2$, then
		\begin{align*}
		f(x) g(x) &= \Big( \sum_{n = 0}^\infty a_n x^n \Big) c x^2 \\
		& = 
		\end{align*}
	\item $g(x) = cx^3$, then
		\begin{align*}
		f(x) g(x) &= \Big( \sum_{n = 0}^\infty a_n x^n \Big) cx^3 \\
		&= 
		\end{align*}
	\end{itemize}

\begin{example}\label{Ex:PowersTimeSeries}
If $f(x) = \sum_{n = 0}^\infty a_n x^n$, find the expression of
	\begin{enumerate}[label=(\alph*)]
	\item $x f'$.
	\item $(2 - x) f''$.
	\end{enumerate}
\end{example}

\newpage

\phantom{2}

\newpage

\subsection{Shifting}

For any integer $k$, if
	\begin{align*}
	y(x) = \sum_{n = n_0}^\infty a_n x^{n-k}
	\end{align*}
then
	\begin{align*}
	y(x) = \sum_{n = n_0 - k}^\infty a_{n + k} x^{n}
	\end{align*}
	
\vspace*{16pt}

\begin{example}
Complete \cref{Ex:PowersTimeSeries}.
\end{example}

\newpage

\begin{example}
Express $2 y - xy''$ as a power series $\sum_{n =0}^\infty c_n x^n$.
\end{example}

\newpage

\phantom{2}


\end{document}
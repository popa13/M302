\documentclass[12pt,a4paper]{article}
\usepackage[utf8]{inputenc}
\usepackage[english]{babel}

\usepackage{amsmath}
\usepackage{amsfonts}
\usepackage{amssymb}

\usepackage{graphicx}
\usepackage{lmodern}
\usepackage{tikz}
\usepackage{titlesec}
\usepackage{environ}
\usepackage{xcolor}
\usepackage{fancyhdr}
\usepackage[colorlinks = true, linkcolor = black]{hyperref}
\usepackage{xparse}
\usepackage{enumitem}
\usepackage{comment}

\usepackage[left=2cm,right=2cm,top=2cm,bottom=2cm]{geometry}
\usepackage{multicol}
\usepackage[indent=0pt]{parskip}

\newcommand{\spaceP}{\vspace*{0.5cm}}
\newcommand{\Span}{\mathrm{Span}\,}
\newcommand{\range}{\mathrm{range}\,}
\newcommand{\ra}{\rightarrow}

%% Redefining sections
\newcommand{\sectionformat}[1]{%
    \begin{tikzpicture}[baseline=(title.base)]
        \node[rectangle, draw] (title) {#1};
    \end{tikzpicture}
    
    \noindent\hrulefill
}

% default values copied from titlesec documentation page 23
% parameters of \titleformat command are explained on page 4
\titleformat%
    {\section}% <command> is the sectioning command to be redefined, i. e., \part, \chapter, \section, \subsection, \subsubsection, \paragraph or \subparagraph.
    {\normalfont\large\scshape}% <format>
    {}% <label> the number
    {0em}% <sep> length. horizontal separation between label and title body
    {\centering\sectionformat}% code preceding the title body  (title body is taken as argument)

%% Set counters for sections to none
\setcounter{secnumdepth}{0}

%% Set the footer/headers
\pagestyle{fancy}
\fancyhf{}
\renewcommand{\headrulewidth}{0pt}
\renewcommand{\footrulewidth}{2pt}
\lfoot{P.-O. Paris{\'e}}
\cfoot{MATH 302}
\rfoot{Page \thepage}

%% Defining example environment
\newcounter{example}[section]
\NewEnviron{example}%
	{%
	\noindent\refstepcounter{example}\fcolorbox{gray!40}{gray!40}{\textsc{\textcolor{red}{Example~\theexample.}}}%
	%\fcolorbox{black}{white}%
		{  %\parbox{0.95\textwidth}%
			{
			\BODY
			}%
		}%
	}

% Theorem environment
\NewEnviron{theorem}%
	{%
	\noindent\refstepcounter{example}\fcolorbox{gray!40}{gray!40}{\textsc{\textcolor{blue}{Theorem~\theexample.}}}%
	%\fcolorbox{black}{white}%
		{  %\parbox{0.95\textwidth}%
			{
			\BODY
			}%
		}%
	}

\NewEnviron{notes}%
	{%
	\noindent \fcolorbox{gray!40}{gray!40}{\textsc{\textcolor{blue}{Solution.}}}%
	%\fcolorbox{black}{white}%
		{  %\parbox{0.95\textwidth}%
			{
			\textcolor{blue}{%
			\BODY
			}
			}%
		}%
	}
%%% Ignorer les notes
\excludecomment{notes}

%%%%
\begin{document}
\thispagestyle{empty}

\begin{center}
\vspace*{2.5cm}

{\Huge \textsc{Math 302}}

\vspace*{2cm}

{\LARGE \textsc{Chapter 5}} 

\vspace*{0.75cm}

\noindent\textsc{Section 5.7: Variation Of Parameters}

\vspace*{0.75cm}

\tableofcontents

\vfill

\noindent \textsc{Created by: Pierre-Olivier Paris{\'e}} \\
\textsc{Fall 2022}
\end{center}

\newpage

\section{What Is The Method Of Variation Of Parameters}

Our goal in this section is to find the solutions to 
	\begin{align*}
	P_0(x) y'' + P_1 (x) y' + P_2 (x) y = F(x)
	\end{align*}
using the method \textbf{variation of parameters}. Our assumption is
	\begin{itemize}
	\item We know at least two solutions to the complementary equation $P_0 (x) y'' + P_1 (x) y' + P_2 (x) y = 0$.
	\end{itemize}

\begin{example}
Find the general solution to
	\begin{align*}
	x^2 y'' - 2x y' + 2y = x^{9/2}
	\end{align*}
given that $y_1 (x) = x$ and $y_2 (x) = x^2$ are solutions to the complementary equation.
\end{example}

\newpage

\begin{notes}
\noindent\underline{Key idea:} Use variation of parameters on the two constants.

We know that the general solution to the complementary equation is
	\begin{align*}
	y (x) = c_1 x + c_2 x^2 .
	\end{align*}
We will change $c_1$ and $c_2$ by two functions $u_1$ and $u_2$ to find a particular solution to the ODE. So we let 
	\begin{align*}
	y_{par} (x) = y(x) = u_1 (x) x + u_2 (x) x^2 .
	\end{align*}
We can then compute $y'$ and $y''$. We have
	\begin{align*}
	y' = u_1' x + u_1 + u_2' x^2 + 2u_2 x .
	\end{align*}
We will impose one condition on $u_1, u_2$ to make things easier. We will assume that $u_1$ and $u_2$ are chosen so that
	\begin{align}
	u_1'x + u_2' x^2 = 0 .
	\end{align}
Therefore, the derivative of $y$ is simply given by
	\begin{align*}
	y' = u_1 + 2u_2 x .
	\end{align*}
In other words, we have taken the derivative of $x$ and $x^2$ in the expression of $y$ and leaving $u_1$ and $u_2$ untouched. 

With this expression of $y'$, we can obtain the second derivative
	\begin{align*}
	y'' = u_1' + 2u_2' x + 2u_2 .
	\end{align*}
We can then replace $y$, $y'$ and $y''$ in the ODE and after simplifying we get
	\begin{align*}
	x^2 u_1' + 2x^3 u_2' = x^{9/2} .
	\end{align*}

We then have to find the functions $u_1, u_2$ satisfying the system of differential equations:
	\begin{align*}
	\left\{ \begin{matrix}
	u_1' x + u_2' x^2 = 0 \\
	x^2 u_1' + 2x^3 u_2' = x^{9/2} .
	\end{matrix} \right.
	\end{align*}
From the first equation, we see that $u_1' = -u_2'x$. Replacing this in the second equation, we obtain
	\begin{align*}
	-x^3 u_2' + 2x^3 u_2' = x^{9/2}
	\end{align*}
and therefore $u_2' = -x^{3/2}$. This also means that $u_1' = x^{5/2}$. Integrating with respect to $x$, we obtain
	\begin{align*}
	u_1 (x) = -\frac{2}{7} x^{7/2} \quad \text{ and } \quad u_2 (x) = \frac{2}{5} x^{5/2} .
	\end{align*}
Therefore, replacing in $y_{par}$, we obtain
	\begin{align*}
	y_{par} (x) = -\frac{2}{5} x^{9/2} + \frac{2}{7} x^{9/2} = \frac{4}{35} x^{9/2} .
	\end{align*}
	
Finally, the general solution to the original ODE is
	\begin{align*}
	y (x) = y_{par} (x) + c_1 x + c_2 x^2 = \frac{4}{35} x^{9/2} + c_1 x + c_2 x^2 .
	\end{align*}
	
\end{notes}

\phantom{1}

\newpage

\subsection{General Procedure}
To find a particular solution to
	\begin{align*}
	P_0 (x) y'' + P_1 (x) y' + P_0 (x) y = F(x)
	\end{align*}
knowing two solutions $y_1 (x)$ and $y_2 (x)$ to the complementary equation, we follow these steps:
	\begin{itemize}
	\item Write $y_{par} (x) = u_1(x) y_1 (x) + u_2 (x) y_2 (x)$.
	\item Write the system
		\begin{align*}
		u_1' y_1 + u_2' y_2 &= 0 \\
		u_1' y_1' + u_2' y_2' &= \frac{F}{P_0} .
		\end{align*}
	\item Solve the system for $u_1'$ and $u_2'$:
		\begin{align*}
		u_1' = -\frac{Fy_2}{P_0 (y_1 y_2' - y_1' y_2)} \quad \text{ and } \quad u_2' = \frac{Fy_1 }{P_0 (y_1 y_2' - y_1' y_2)} .
		\end{align*}
	\item Obtain $u_1$ and $u_2$ by integrating $u_1'$ and $u_2'$ respectively.
	\item Substitute $u_1$ and $u_2$ in $y_{par}(x)$ to obtain the particular solution.
	\end{itemize}
	
\vspace*{16pt}
	
\begin{example}
Find a particular solution to 
	\begin{align*}
	y'' + 3y' + 2y = \frac{1}{1 + e^x} .
	\end{align*}
\end{example}
\begin{notes}
The general solution to the complementary equation $y'' + 3y' + 2y = 0$ is
	\begin{align*}
	y (x) = c_1 e^{-2 x} + c_2 e^{-x} .
	\end{align*}
So we set
	\begin{align*}
	y_{par} (x) = y(x) = u_1 (x) e^{-2x} + u_2 (x) e^{-x} .
	\end{align*}
We add the restriction 
	\begin{align*}
	u_1' e^{-2x} + u_2' e^{-x} = 0 \iff u_1' + u_2' e^{x} = 0.
	\end{align*}
Therefore, the derivative of $y$ is
	\begin{align*}
	y' = -2 u_1 e^{-2x} - u_2 e^{-x} 
	\end{align*}
and the second derivative is
	\begin{align*}
	y'' = -2u_1' e^{-2x} + 4 u_1 e^{-2x} - u_2' e^{-x} + u_2 e^{-x} .
	\end{align*}
Replacing this in the initial ODE and simplifying, we get
	\begin{align*}
	-2u_1' - u_2' e^x = \frac{e^{2x}}{1 + e^x} .
	\end{align*}
Therefore, $u_1$ and $u_2$ must satisfy the following system of ODEs:
	\begin{align*}
	\left\{ \begin{matrix}
	u_1' + u_2' e^{x} = 0 \\
	-2u_1' - u_2' e^x = \frac{e^{2x}}{1 + e^x} .
	\end{matrix} \right.
	\end{align*}
From the first equation, we see that $u_1' = -u_2' e^x$. Replacing this into the second equation, we see that
	\begin{align*}
	2u_2' e^x - u_2' e^x = \frac{e^{2x}}{1 + e^x} \iff u_2' = \frac{e^x}{1 + e^x} .
	\end{align*}
	Plugging this in the expression of $u_1'$, we see that 
		\begin{align*}
		u_1' = -\frac{e^{2x}}{1 + e^x} .
		\end{align*}
	Integrating $u_2' = \frac{e^x}{1 + e^x}$ with the change of variable $v = 1 + e^x$, we see that
		\begin{align*}
		u_2 (x) = \ln (1 + e^x )
		\end{align*}
	where the absolute value was removed because $1 + e^x$ is always positive for any $x$. To integrate $u_1' = -\frac{e^{2x}}{1 + e^x}$, we make the change of variable $v = e^x$ and then integrate $v/(1 + v) = v + 1$. Therefore, we see that
		\begin{align*}
		u_1 (x) = e^{2x} + e^x .
		\end{align*}
	Plugging into $y_{par}$, we obtain
		\begin{align*}
		y_{par} (x) = e^{-2x} (e^{2x} + e^x) + e^{-x} \ln (1 + e^x) = 1 + e^{-x} + e^{-x} \ln (1 + e^x ) .
		\end{align*}
		
		The general solution to the ODE is then
			\begin{align*}
			y (x) = y_{par} (x) + c_1 e^{-2x} + c_2 e^{-x} = 1 + e^{-x} + e^{-x} \ln (1 + e^x ) + c_1 e^{-2x} + c_2 e^{-x} .
			\end{align*}
\end{notes}

\newpage

\phantom{2}

\end{document}
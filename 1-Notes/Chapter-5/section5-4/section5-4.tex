\documentclass[12pt,a4paper]{article}
\usepackage[utf8]{inputenc}
\usepackage[english]{babel}

\usepackage{amsmath}
\usepackage{amsfonts}
\usepackage{amssymb}

\usepackage{graphicx}
\usepackage{lmodern}
\usepackage{tikz}
\usepackage{titlesec}
\usepackage{environ}
\usepackage{xcolor}
\usepackage{fancyhdr}
\usepackage[colorlinks = true, linkcolor = black]{hyperref}
\usepackage{xparse}
\usepackage{enumitem}

\usepackage[left=2cm,right=2cm,top=2cm,bottom=2cm]{geometry}
\usepackage{multicol}
\usepackage[indent=0pt]{parskip}

\newcommand{\spaceP}{\vspace*{0.5cm}}
\newcommand{\Span}{\mathrm{Span}\,}
\newcommand{\range}{\mathrm{range}\,}
\newcommand{\ra}{\rightarrow}

%% Redefining sections
\newcommand{\sectionformat}[1]{%
    \begin{tikzpicture}[baseline=(title.base)]
        \node[rectangle, draw] (title) {#1};
    \end{tikzpicture}
    
    \noindent\hrulefill
}

% default values copied from titlesec documentation page 23
% parameters of \titleformat command are explained on page 4
\titleformat%
    {\section}% <command> is the sectioning command to be redefined, i. e., \part, \chapter, \section, \subsection, \subsubsection, \paragraph or \subparagraph.
    {\normalfont\large\scshape}% <format>
    {}% <label> the number
    {0em}% <sep> length. horizontal separation between label and title body
    {\centering\sectionformat}% code preceding the title body  (title body is taken as argument)

%% Set counters for sections to none
\setcounter{secnumdepth}{0}

%% Set the footer/headers
\pagestyle{fancy}
\fancyhf{}
\renewcommand{\headrulewidth}{0pt}
\renewcommand{\footrulewidth}{2pt}
\lfoot{P.-O. Paris{\'e}}
\cfoot{MATH 302}
\rfoot{Page \thepage}

%% Defining example environment
\newcounter{example}[section]
\NewEnviron{example}%
	{%
	\noindent\refstepcounter{example}\fcolorbox{gray!40}{gray!40}{\textsc{\textcolor{red}{Example~\theexample.}}}%
	%\fcolorbox{black}{white}%
		{  %\parbox{0.95\textwidth}%
			{
			\BODY
			}%
		}%
	}

% Theorem environment
\NewEnviron{theorem}%
	{%
	\noindent\refstepcounter{example}\fcolorbox{gray!40}{gray!40}{\textsc{\textcolor{blue}{Theorem~\theexample.}}}%
	%\fcolorbox{black}{white}%
		{  %\parbox{0.95\textwidth}%
			{
			\BODY
			}%
		}%
	}
	

%%%%
\begin{document}
\thispagestyle{empty}

\begin{center}
\vspace*{2.5cm}

{\Huge \textsc{Math 302}}

\vspace*{2cm}

{\LARGE \textsc{Chapter 5}} 

\vspace*{0.75cm}

\noindent\textsc{Section 5.4: The Method of Undetermined Coefficient I}

\vspace*{0.75cm}

\tableofcontents

\vfill

\noindent \textsc{Created by: Pierre-Olivier Paris{\'e}} \\
\textsc{Fall 2022}
\end{center}

\newpage

\section{When The Force Function Is An Exponential}

We consider the following basic case:
	\begin{align*}
	a y'' + by' + cy = k e^{\alpha x}
	\end{align*}
where $a$, $b$, $c$, $\alpha$, and $k$ are fixed real numbers.

\subsection{Case I}

When $e^{\alpha x}$ is not a solution to the complementary equation $ay'' + by' + cy = 0$.

\begin{example}
Find the general solution of
	\begin{align*}
	y'' - 7y' + 12 y = 4e^{2x} .
	\end{align*}
\end{example}

\newpage

\subsection{Case II}

When $e^{\alpha x}$ is a solution to the complementary equation.

\begin{example}
Find the general solution of
	\begin{align*}
	y'' - 7y' + 12 y = 5 e^{4x} .
	\end{align*}
\end{example}

\newpage

\phantom{2}

\newpage

\subsection{Case III}
When $e^{\alpha x}$, and $xe^{\alpha x}$ are solutions to the complementary equation.

\begin{example}
Find the general solution of
	\begin{align*}
	y'' - 8 y' + 16y = 2e^{4x} .
	\end{align*}
\end{example}

\newpage

\phantom{2}

\newpage

\subsection{Recap}
To find a particular solution to
	\begin{align*}
	ay'' + by' + cy = k e^{\alpha x}
	\end{align*}
where $k$ is a fixed real number, we follow the following tips:
	\begin{itemize}
	\item If $e^{\alpha x}$ is not a solution of the complementary equation, then we take $y_{par} (x) = Ae^{\alpha x}$, where $A$ is a constant.
	\item If $e^{\alpha x}$ is a solution of the complementary equation, then we take $y_{par} (x) = xAe^{\alpha x}$, where $A$ is a constant.
	\item If both $e^{\alpha x}$ and $x e^{\alpha x}$ are solutions of the complementary equation, then we take $y_{par} (x) = Ax^2 e^{\alpha x}$, where $A$ is a constant.
	\end{itemize}

\newpage

\section{When The Force Function Is Exponential Times Polynomial}

We now consider a more general case:
	\begin{align*}
	ay'' + by' + cy = e^{\alpha x} G(x)
	\end{align*}
where $a$, $b$, $c$, $\alpha$ are fixed real numbers and $G(x)$ is a polynomial.

\subsection{Case I}

When $e^{\alpha x}$ is not a solution to the complementary equation $ay'' + by' + cy = 0$.

\begin{example}
Find the general solution to
	\begin{align*}
	y'' - 3y' + 2y = e^{3x} (x^2 + 2x - 1) .
	\end{align*}
\end{example}

\newpage

\phantom{2}

\newpage

\subsection{Case II}

When $e^{\alpha x}$ is a solution to the complementary equation.

\begin{example}
Find the general solution to
	\begin{align*}
	y'' - 4y' + 3y = e^{3x} (12x^2 + 8x + 6) .
	\end{align*}
\end{example}

\newpage

\phantom{1}

\newpage

\subsection{Case III}

When $e^{\alpha x}$ and $xe^{\alpha x}$ are solutions to the complementary equation.

\begin{example}
Find the general solution to
	\begin{align*}
	4y'' + 4y' + y = e^{-x/2} (144x^2 + 48 x - 8) .
	\end{align*}
\end{example}

\newpage

\subsection{Recap}
To find a particular solution to
	\begin{align*}
	ay'' + by' + cy = k e^{\alpha x} G(x)
	\end{align*}
where $k$ is a fixed real number and $G(x)$ is a polynomial, we follow the following tips:
	\begin{itemize}
	\item If $e^{\alpha x}$ is not a solution of the complementary equation, then we take $y_{par} (x) = A e^{\alpha x} Q(x)$, where $A$ is a constant and $Q(x)$ is a polynomial of the same degree as $G(x)$.
	\item If $e^{\alpha x}$ is a solution of the complementary equation, then we take $y_{par} (x) = Axe^{\alpha x} Q(x)$, where $A$ is a constant and $Q(x)$ is a polynomial of the same degree as $G(x)$.
	\item If $e^{\alpha x}$ and $xe^{\alpha x}$ are solutions to the complementary equation, then we take $y_{par} (x) = Ax^2 e^{\alpha x} Q(x)$, where $A$ is a constant and $Q(x)$ is a polynomial of the same degree as $G(x)$.
	\end{itemize}

\end{document}
\documentclass[12pt,a4paper]{article}
\usepackage[utf8]{inputenc}
\usepackage[english]{babel}

\usepackage{amsmath}
\usepackage{amsfonts}
\usepackage{amssymb}

\usepackage{graphicx}
\usepackage{lmodern}
\usepackage{tikz}
\usepackage{titlesec}
\usepackage{environ}
\usepackage{xcolor}
\usepackage{fancyhdr}
\usepackage[colorlinks = true, linkcolor = black]{hyperref}
\usepackage{xparse}
\usepackage{enumitem}

\usepackage[left=2cm,right=2cm,top=2cm,bottom=2cm]{geometry}
\usepackage{multicol}
\usepackage[indent=0pt]{parskip}

\newcommand{\spaceP}{\vspace*{0.5cm}}
\newcommand{\Span}{\mathrm{Span}\,}
\newcommand{\range}{\mathrm{range}\,}
\newcommand{\ra}{\rightarrow}

%% Redefining sections
\newcommand{\sectionformat}[1]{%
    \begin{tikzpicture}[baseline=(title.base)]
        \node[rectangle, draw] (title) {#1};
    \end{tikzpicture}
    
    \noindent\hrulefill
}

% default values copied from titlesec documentation page 23
% parameters of \titleformat command are explained on page 4
\titleformat%
    {\section}% <command> is the sectioning command to be redefined, i. e., \part, \chapter, \section, \subsection, \subsubsection, \paragraph or \subparagraph.
    {\normalfont\large\scshape}% <format>
    {}% <label> the number
    {0em}% <sep> length. horizontal separation between label and title body
    {\centering\sectionformat}% code preceding the title body  (title body is taken as argument)

%% Set counters for sections to none
\setcounter{secnumdepth}{0}

%% Set the footer/headers
\pagestyle{fancy}
\fancyhf{}
\renewcommand{\headrulewidth}{0pt}
\renewcommand{\footrulewidth}{2pt}
\lfoot{P.-O. Paris{\'e}}
\cfoot{MATH 302}
\rfoot{Page \thepage}

%% Defining example environment
\newcounter{example}[section]
\NewEnviron{example}%
	{%
	\noindent\refstepcounter{example}\fcolorbox{gray!40}{gray!40}{\textsc{\textcolor{red}{Example~\theexample.}}}%
	%\fcolorbox{black}{white}%
		{  %\parbox{0.95\textwidth}%
			{
			\BODY
			}%
		}%
	}

% Theorem environment
\NewEnviron{theorem}%
	{%
	\noindent\refstepcounter{example}\fcolorbox{gray!40}{gray!40}{\textsc{\textcolor{blue}{Theorem~\theexample.}}}%
	%\fcolorbox{black}{white}%
		{  %\parbox{0.95\textwidth}%
			{
			\BODY
			}%
		}%
	}
	

%%%%
\begin{document}
\thispagestyle{empty}

\begin{center}
\vspace*{2.5cm}

{\Huge \textsc{Math 302}}

\vspace*{2cm}

{\LARGE \textsc{Chapter 5}} 

\vspace*{0.75cm}

\noindent\textsc{Section 5.2: Constant Coefficient Homogeneous Equations}

\vspace*{0.75cm}

\tableofcontents

\vfill

\noindent \textsc{Created by: Pierre-Olivier Paris{\'e}} \\
\textsc{Fall 2022}
\end{center}

\newpage

\section{What is a Constant Coefficient Homogeneous ODE?}

We restrict even further the second order ODE. A \textbf{second order constant coefficient ODE} is an ODE of the form
	\begin{align}
	a y'' + by' + c y = f(x) 
	\end{align}
where $a$, $b$, $c$ are fixed numbers and $f$ is a continuous function. 

\underline{Goal:} 

Find the solutions to 
	\begin{align*}
	ay'' + by' + cy = 0 .
	\end{align*}
We call this the \textbf{constant coefficient homogeneous ODE}.

\underline{Trick}:

\newpage

\section{Distinct Real Roots: $\sqrt{b^2 - 4ac} > 0$}

\begin{example}
Find the general solution of
	\begin{align*}
	y'' + 6y' + 5y = 0 .
	\end{align*}
\end{example}

\vfill

\underline{General Fact:}
	\begin{itemize}
	\item If the roots of the characteristic polynomial are $r_1$ and $r_2$, then $y_1 (x) = e^{r_1 x}$ and $y_2 = e^{r_2 x}$ are solutions to the ODE.
	\item The general solutions is given by
		\begin{align*}
		y (x) = c_1 e^{r_1 x} + c_2 e^{r_2 x} .
		\end{align*}
	\end{itemize}
	
\newpage

\section{Repeated Roots: $\sqrt{b^2 - 4ac} = 0$}

\begin{example}
\begin{enumerate}[label=\alph*)]
\item Find the general solution of
	\begin{align*}
	y'' + 6y' + 9y = 0 .
	\end{align*}
\item Solve the following IVP:
	\begin{align*}
	y'' + 6y' + 9y = 0 , \quad y(0) = 3 , \, y' (0) = -1 .
	\end{align*}
\end{enumerate}
\end{example}

\newpage

\phantom{2}

\vfill

\underline{General Facts:}
	\begin{itemize}
	\item If the root of the characteristic polynomial is $r_1$, then $y_1 (x) = e^{r_1 x}$ and $y_2 (x) = x e^{r_1 x}$ are solutions to the ODE.
	\item The general solution is given by
		\begin{align*}
		y(x) = e^{r_1 x} (c_1 + c_2 x ) .
		\end{align*}
	\end{itemize}
	
\newpage

\section{Complex Roots: $\sqrt{b^2 - 4ac} < 0$}

	\begin{example}
	\begin{enumerate}[label=\alph*)]
	\item Find the general solution of
		\begin{align*}
		y'' + 4y' + 13y = 0 .
		\end{align*}
	\item Solve the following IVP:
		\begin{align*}
		y'' + 4y + 13y = 0 , \quad y (0) = 2 , \, y' (0) = -3 .
		\end{align*}
	\end{enumerate}
	\end{example}
	
	\vfill

	\subsection{Complex Numbers}
	A complex number is an expression of the form
		\begin{align*}
		z = \alpha + i \beta
		\end{align*}
	where $\alpha$, $\beta$ are real numbers and $i^2 = -1$ ($i = \sqrt{-1}$). 
	
	Consider $z = \alpha + i \beta$ and $w = \gamma + i \mu$.
	\begin{multicols}{2}
		\begin{itemize}
		\item $z = w$ if and only if $\alpha = \gamma$ and $\beta = \mu$.
		\item $z + w = (\alpha + \gamma ) + i (\beta + \mu )$.
		\item $zw = (\alpha \gamma - \beta \mu ) + i (\alpha \mu + \beta \gamma )$.
		\item $z/w = \frac{(\alpha + i \beta) (\gamma - i \mu )}{(\gamma + i \mu) (\gamma - i \mu )}$, if $w \neq 0$.
		\end{itemize}
	\end{multicols}
	
	\begin{example}
	If $z = 1 + i$ and $w = 1 - i$, find
	\begin{multicols}{3}
		\begin{enumerate}[label=\alph*)]
		\item $z + w$.
		\item $zw$.
		\item $z/w$.
		\end{enumerate}
	\end{multicols}
	\end{example}
	
	\vfill
	
\newpage

\begin{example}
Complete the previous example.
\end{example}

\newpage

\phantom{2}

\vfill

\underline{General Facts:}
	\begin{itemize}
	\item If $r_1 = \alpha + \beta i$ and $r_2 = \alpha - \beta i$ are the roots of the characteristic polynomial, then $y_1 (x) = e^{\alpha x} \cos (\beta x )$ and $y_2 (x) = e^{\alpha x} \sin (\beta x )$ are solutions to the ODE.
	\item The general solution has the form
		\begin{align*}
		y(x) = e^{\alpha x} (c_1 \cos (\beta x) + c_2 \sin (\beta x ) ) .
		\end{align*}
	\end{itemize}
	

\end{document}
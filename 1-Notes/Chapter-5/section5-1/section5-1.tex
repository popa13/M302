\documentclass[12pt,a4paper]{article}
\usepackage[utf8]{inputenc}
\usepackage[english]{babel}

\usepackage{amsmath}
\usepackage{amsfonts}
\usepackage{amssymb}

\usepackage{graphicx}
\usepackage{lmodern}
\usepackage{tikz}
\usepackage{titlesec}
\usepackage{environ}
\usepackage{xcolor}
\usepackage{fancyhdr}
\usepackage[colorlinks = true, linkcolor = black]{hyperref}
\usepackage{xparse}
\usepackage{enumitem}

\usepackage[left=2cm,right=2cm,top=2cm,bottom=2cm]{geometry}
\usepackage{multicol}
\usepackage[indent=0pt]{parskip}

\newcommand{\spaceP}{\vspace*{0.5cm}}
\newcommand{\Span}{\mathrm{Span}\,}
\newcommand{\range}{\mathrm{range}\,}
\newcommand{\ra}{\rightarrow}

%% Redefining sections
\newcommand{\sectionformat}[1]{%
    \begin{tikzpicture}[baseline=(title.base)]
        \node[rectangle, draw] (title) {#1};
    \end{tikzpicture}
    
    \noindent\hrulefill
}

% default values copied from titlesec documentation page 23
% parameters of \titleformat command are explained on page 4
\titleformat%
    {\section}% <command> is the sectioning command to be redefined, i. e., \part, \chapter, \section, \subsection, \subsubsection, \paragraph or \subparagraph.
    {\normalfont\large\scshape}% <format>
    {}% <label> the number
    {0em}% <sep> length. horizontal separation between label and title body
    {\centering\sectionformat}% code preceding the title body  (title body is taken as argument)

%% Set counters for sections to none
\setcounter{secnumdepth}{0}

%% Set the footer/headers
\pagestyle{fancy}
\fancyhf{}
\renewcommand{\headrulewidth}{0pt}
\renewcommand{\footrulewidth}{2pt}
\lfoot{P.-O. Paris{\'e}}
\cfoot{MATH 302}
\rfoot{Page \thepage}

%% Defining example environment
\newcounter{example}[section]
\NewEnviron{example}%
	{%
	\noindent\refstepcounter{example}\fcolorbox{gray!40}{gray!40}{\textsc{\textcolor{red}{Example~\theexample.}}}%
	%\fcolorbox{black}{white}%
		{  %\parbox{0.95\textwidth}%
			{
			\BODY
			}%
		}%
	}

% Theorem environment
\NewEnviron{theorem}%
	{%
	\noindent\refstepcounter{example}\fcolorbox{gray!40}{gray!40}{\textsc{\textcolor{blue}{Theorem~\theexample.}}}%
	%\fcolorbox{black}{white}%
		{  %\parbox{0.95\textwidth}%
			{
			\BODY
			}%
		}%
	}
	

%%%%
\begin{document}
\thispagestyle{empty}

\begin{center}
\vspace*{2.5cm}

{\Huge \textsc{Math 302}}

\vspace*{2cm}

{\LARGE \textsc{Chapter 5}} 

\vspace*{0.75cm}

\noindent\textsc{Section 5.1: Homogeneous Linear Equations}

\vspace*{0.75cm}

\tableofcontents

\vfill

\noindent \textsc{Created by: Pierre-Olivier Paris{\'e}} \\
\textsc{Fall 2022}
\end{center}

\newpage

\section{What Is A Second Order Linear ODE?}

We will be mainly interested in the following specific ODEs:
	\begin{align}
	y'' + p(x) y' + q(x) y = f(x) \label{Eq:SecondOrderLinearODE}
	\end{align}
where $p$, $q$, and $f$ are continuous functions of the variable $x$.

	\begin{itemize}
	\item When $f(x) = 0$ for any $x$, the ODEs is called \textbf{homogeneous}.
	\item When $f(x) \neq 0$, the ODEs is called \textbf{non-homogeneous}.
	\item The function $f$ is called the \textbf{forcing function} .
	\item The IVP associated to a second order ODE of the form \eqref{Eq:SecondOrderLinearODE} is
		\begin{align*}
		y'' + p(x) y' + q(x) y = f(x) , \quad y(x_0) = k_0 , \, y'(x_0) = k_1
		\end{align*}
	for some point $x_0$ in an interval $(a, b)$ and $k_0$, $k_1$ are arbitrary numbers.
	\end{itemize}
	
\underline{Goal:} To solve the homogeneous equation
	\begin{align*}
	y'' + p(x) y' + q(x) y = 0 .
	\end{align*}
	
\begin{example}\label{Examp:FirstOne}
Consider the ODE
	\begin{align*}
	y'' - y = 0 .
	\end{align*}
\begin{enumerate}[label=\alph*)]
\item Identify the functions $p$ and $q$.
\item Verify that $y_1 (x) = e^x$ and $y_2 (x) = e^{-x}$ are solutions of the ODE on $(-\infty , \infty )$.
\item Verify that if $c_1$ and $c_2$ are arbitrary constants, then $y (x) = c_1 e^{x} + c_2 e^{-x}$ is a solution to the ODE on $(-\infty , \infty )$.
\item Solve the initial value problem
	\begin{align*}
	y'' - y = 0 , \quad y(0) = 1 , \, y' (0) = 3 .
	\end{align*}
\end{enumerate}
\end{example}

\newpage

\phantom{1}

\newpage

\begin{example}\label{Examp:SecondOne}
Let $\omega$ be a positive number. Consider
	\begin{align*}
	y'' + \omega^2 y = 0 .
	\end{align*}
\begin{enumerate}[label=\alph*)]
\item Identify the functions $p(x)$ and $q(x)$.
\item Verify that $y_1 (x) = \cos (\omega x )$ and $y_2 (x) = \sin (\omega x)$ are solutions to the ODE.
\item Verify that $y(x) = c_1 \cos (\omega x ) + c_2 \sin (\omega x)$ is a solution to the ODE.
\end{enumerate}
\end{example}

\newpage

Sometimes, the ODE will be given in the following form:
	\begin{align*}
	P_0 (x) y'' + P_1 (x) y' + P_2 (x) y = 0
	\end{align*}
where $P_0$, $P_1$, and $P_2$ are continuous functions. 

\vspace*{16pt}

\begin{example}\label{Examp:ThirdOne}
Consider the equation
	\begin{align*}
	x^2 y'' + x y' - 4 y = 0 .
	\end{align*}
	\begin{enumerate}[label=\alph*)]
	\item Identify the functions $p(x)$ and $q(x)$.
	\item Verify that $y_1 (x) = x^2$ and $y_2 (x) = 1/x^2$ are solutions to the ODE.
	\item Verify that if $c_1$ and $c_2$ are arbitrary numbers, then $y(x) = c_1 x^2 + c_2 /x^2$ is a solution of the ODE.
	\item Solve the IVP
		\begin{align*}
		x^2 y'' + x y' - 4y = 0, \quad y(1) = 2 , \, y'(1) = 0 .
		\end{align*}
	\end{enumerate}
\end{example}

\newpage

\section{General Solutions}

\subsection{Linear combinations}

If $y_1$ and $y_2$ are functions, we say that the function
	\begin{align*}
	y(x) = c_1 y_1 (x) + c_2 y_2 (x) ,
	\end{align*}
where $c_1$ and $c_2$ are numbers, is a \textbf{linear combination} of $y_1$ and $y_2$.

	\underline{Fact:}
	\begin{itemize}
	\item If $y_1$ and $y_2$ are solutions to \eqref{Eq:SecondOrderLinearODE}, then any linear combinations of $y_1$ and $y_2$ is a solution to \eqref{Eq:SecondOrderLinearODE}.
	\end{itemize}
	
\subsection{Fundamental Set of Solutions}

We say that $\{ y_1 , y_2 \}$ is a \textbf{fundamental set of solutions} for \eqref{Eq:SecondOrderLinearODE} if every solutions of the ODE is a linear combination of $y_1$ and $y_2$. 
	
	\underline{Facts}:
		\begin{itemize}
		\item $\{ y_1 , y_2 \}$ is a fundamental set of solutions for \eqref{Eq:SecondOrderLinearODE} if and only if neither $y_2/y_1$ or $y_1/y_2$ is a constant.
		\end{itemize}
		
	\begin{example}
	Show that
		\begin{itemize}
		\item The functions $\{ y_1 , y_2 \}$ where $y_1$, $y_2$ are as in Example \ref{Examp:FirstOne} is a foundamental set of solutions.
		\item Same question for $y_1$, $y_2$ from Example \ref{Examp:SecondOne}.
		\item Same question for $y_1$, $y_2$ from Example \ref{Examp:ThirdOne}.
		\end{itemize}
	\end{example}
	
	\vfill
	
\subsection{General Solutions}
If $\{ y_1 , y_2 \}$ is a fundamental set of solutions for \eqref{Eq:SecondOrderLinearODE}, then we call the linear combination $y (x) = c_1 y_1 + c_2 y_2$ the \textbf{general solution} to \eqref{Eq:SecondOrderLinearODE}.
	
\newpage

\section{Existence and Uniqueness of Solutions}

It is always clever to verify if an ODE has solutions. Here are some important facts about existence and uniqueness of solutions to an ODE of the form \eqref{Eq:SecondOrderLinearODE}.

	\subsection{Existence}
	Assume that $p$ and $q$ are continuous on an open interval $(a, b)$. Then the ODE
		\begin{align*}
		y'' + p(x) y' + q(x) y = 0
		\end{align*}
	has at least one solution on the interval $(a, b)$.
	
	\subsection{Uniqueness}
	Assume again that $p$ and $q$ are continuous on an open interval $(a, b)$ and let $x_0$ be any point in $(a, b)$. Then the IVP
		\begin{align*}
		y'' + p(x) y' + q(x) y = 0 , \quad y(x_0) = k_0 , \, y'(x_0) = k_1
		\end{align*}
	has a unique solution on $(a, b)$.

\end{document}
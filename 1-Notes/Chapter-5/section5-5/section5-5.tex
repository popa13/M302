\documentclass[12pt,a4paper]{article}
\usepackage[utf8]{inputenc}
\usepackage[english]{babel}

\usepackage{amsmath}
\usepackage{amsfonts}
\usepackage{amssymb}

\usepackage{graphicx}
\usepackage{lmodern}
\usepackage{tikz}
\usepackage{titlesec}
\usepackage{environ}
\usepackage{xcolor}
\usepackage{fancyhdr}
\usepackage[colorlinks = true, linkcolor = black]{hyperref}
\usepackage{xparse}
\usepackage{enumitem}

\usepackage[left=2cm,right=2cm,top=2cm,bottom=2cm]{geometry}
\usepackage{multicol}
\usepackage[indent=0pt]{parskip}

\newcommand{\spaceP}{\vspace*{0.5cm}}
\newcommand{\Span}{\mathrm{Span}\,}
\newcommand{\range}{\mathrm{range}\,}
\newcommand{\ra}{\rightarrow}

%% Redefining sections
\newcommand{\sectionformat}[1]{%
    \begin{tikzpicture}[baseline=(title.base)]
        \node[rectangle, draw] (title) {#1};
    \end{tikzpicture}
    
    \noindent\hrulefill
}

% default values copied from titlesec documentation page 23
% parameters of \titleformat command are explained on page 4
\titleformat%
    {\section}% <command> is the sectioning command to be redefined, i. e., \part, \chapter, \section, \subsection, \subsubsection, \paragraph or \subparagraph.
    {\normalfont\large\scshape}% <format>
    {}% <label> the number
    {0em}% <sep> length. horizontal separation between label and title body
    {\centering\sectionformat}% code preceding the title body  (title body is taken as argument)

%% Set counters for sections to none
\setcounter{secnumdepth}{0}

%% Set the footer/headers
\pagestyle{fancy}
\fancyhf{}
\renewcommand{\headrulewidth}{0pt}
\renewcommand{\footrulewidth}{2pt}
\lfoot{P.-O. Paris{\'e}}
\cfoot{MATH 302}
\rfoot{Page \thepage}

%% Defining example environment
\newcounter{example}[section]
\NewEnviron{example}%
	{%
	\noindent\refstepcounter{example}\fcolorbox{gray!40}{gray!40}{\textsc{\textcolor{red}{Example~\theexample.}}}%
	%\fcolorbox{black}{white}%
		{  %\parbox{0.95\textwidth}%
			{
			\BODY
			}%
		}%
	}

% Theorem environment
\NewEnviron{theorem}%
	{%
	\noindent\refstepcounter{example}\fcolorbox{gray!40}{gray!40}{\textsc{\textcolor{blue}{Theorem~\theexample.}}}%
	%\fcolorbox{black}{white}%
		{  %\parbox{0.95\textwidth}%
			{
			\BODY
			}%
		}%
	}
	

%%%%
\begin{document}
\thispagestyle{empty}

\begin{center}
\vspace*{2.5cm}

{\Huge \textsc{Math 302}}

\vspace*{2cm}

{\LARGE \textsc{Chapter 5}} 

\vspace*{0.75cm}

\noindent\textsc{Section 5.5: The Method of Undetermined Coefficient II}

\vspace*{0.75cm}

\tableofcontents

\vfill

\noindent \textsc{Created by: Pierre-Olivier Paris{\'e}} \\
\textsc{Fall 2022}
\end{center}

\newpage

\section{When The Force Function Is A Trig. Function}

We consider the following first basic case:
	\begin{align*}
	ay'' + by' + cy = F \cos \omega x + G \sin \omega x 
	\end{align*}
where $F$, $G$ and $\alpha$ are fixed real numbers.
	
\subsection{Case I}
When $\cos \omega x$ and $\sin \omega x$ are not solution to the complementary equation $ay'' + by' + cy = 0$.

\begin{example}
Find the general solution to
	\begin{align*}
	y'' - 2y' + y = 5\cos 2x + 10 \sin 2x .
	\end{align*}
\end{example}

\newpage

\phantom{1}

\newpage

\subsection{Case II}
When $\cos \omega x$ or $\sin \omega x$ are solutions to the complementary equation.

\begin{example}
Find the general solution to
	\begin{align*}
	y'' + 4y = 8\cos 2x + 12 \sin 2x .
	\end{align*}
\end{example}

\newpage

\phantom{1}

\newpage

\section{When The Force Function Is Polynomial Times Trig. Function}

We consider the following second basic case:
	\begin{align*}
	ay'' + by' + cy = F(x) \cos \omega x + G(x) \sin \omega x
	\end{align*}
where $\omega$ is a fixed real number and $F$, $G$ are two polynomials.

There are still two cases: weither $\cos \omega x$ and $\sin \omega x$ are or are not solutions to the complementary equation.

\begin{example}
Find the general solution to
	\begin{align*}
	y'' + 3y' + 2y = (16 + 20x) \cos x + 10 \sin x .
	\end{align*}
\end{example}

\newpage

\phantom{2}

\newpage

\section{When The Force Function Is Poly., Expo., Trig. Functions}

We now consider the more general case
	\begin{align*}
	ay'' + by' + c = e^{\alpha x} \left( F(x) \cos \omega x + G(x) \sin \omega x \right)
	\end{align*}
where $\alpha$, $\omega$ are real numbers with $\omega \neq 0$ and $F$, $G$ are polynomials.

There are also two cases: weither $e^{\alpha x}\cos \omega x$ and/or $e^{\alpha x} \sin \omega x$ are or are not solutions to the complementary equation.

\begin{example}
Find the general solution of
	\begin{align*}
	y'' + 2y' + 5y = e^{-x} \left( (6 - 16x) \cos 2x - (8 + 8x) \sin 2x \right) .
	\end{align*}
\end{example}

\newpage

\phantom{2}

\newpage

\subsection{Recap}
A particular solution of
	\begin{align*}
	ay'' + by' + cy = e^{\alpha x} \left( F(x) \cos \omega x + G(x) \sin \omega x \right)
	\end{align*}
where $\omega \neq 0$ has the form
	\begin{itemize}
	\item \underline{when $e^{\alpha x} \cos \omega x$ and $e^{\alpha x} \sin \omega x$ are not solutions to the complementary equation}, 
		\begin{align*}
		y_{par} (x) = e^{\alpha x} \left( A(x) \cos \omega x + B(x) \sin \omega x \right),
		\end{align*}
	with $A(x)$ and $B(x)$ are polynomials of the same degree as the biggest degree between $F(x)$ and $G(x)$
	\item \underline{When $e^{\alpha x} \cos \omega x$ and $e^{\alpha x} \sin \omega x$ are solutions to the complementary equation},
		\begin{align*}
		y_{par} (x) = x e^{\alpha x} \left( A(x) \cos \omega x + B(x) \sin \omega x \right),
		\end{align*}
	with $A(x)$ and $B(x)$ are polynomials of the same degree as the highest degree between the polynomials $F(x)$ and $G(x)$.
	\end{itemize}

\end{document}
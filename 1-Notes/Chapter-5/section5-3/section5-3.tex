\documentclass[12pt,a4paper]{article}
\usepackage[utf8]{inputenc}
\usepackage[english]{babel}

\usepackage{amsmath}
\usepackage{amsfonts}
\usepackage{amssymb}

\usepackage{graphicx}
\usepackage{lmodern}
\usepackage{tikz}
\usepackage{titlesec}
\usepackage{environ}
\usepackage{xcolor}
\usepackage{fancyhdr}
\usepackage[colorlinks = true, linkcolor = black]{hyperref}
\usepackage{xparse}
\usepackage{enumitem}

\usepackage[left=2cm,right=2cm,top=2cm,bottom=2cm]{geometry}
\usepackage{multicol}
\usepackage[indent=0pt]{parskip}

\newcommand{\spaceP}{\vspace*{0.5cm}}
\newcommand{\Span}{\mathrm{Span}\,}
\newcommand{\range}{\mathrm{range}\,}
\newcommand{\ra}{\rightarrow}

%% Redefining sections
\newcommand{\sectionformat}[1]{%
    \begin{tikzpicture}[baseline=(title.base)]
        \node[rectangle, draw] (title) {#1};
    \end{tikzpicture}
    
    \noindent\hrulefill
}

% default values copied from titlesec documentation page 23
% parameters of \titleformat command are explained on page 4
\titleformat%
    {\section}% <command> is the sectioning command to be redefined, i. e., \part, \chapter, \section, \subsection, \subsubsection, \paragraph or \subparagraph.
    {\normalfont\large\scshape}% <format>
    {}% <label> the number
    {0em}% <sep> length. horizontal separation between label and title body
    {\centering\sectionformat}% code preceding the title body  (title body is taken as argument)

%% Set counters for sections to none
\setcounter{secnumdepth}{0}

%% Set the footer/headers
\pagestyle{fancy}
\fancyhf{}
\renewcommand{\headrulewidth}{0pt}
\renewcommand{\footrulewidth}{2pt}
\lfoot{P.-O. Paris{\'e}}
\cfoot{MATH 302}
\rfoot{Page \thepage}

%% Defining example environment
\newcounter{example}[section]
\NewEnviron{example}%
	{%
	\noindent\refstepcounter{example}\fcolorbox{gray!40}{gray!40}{\textsc{\textcolor{red}{Example~\theexample.}}}%
	%\fcolorbox{black}{white}%
		{  %\parbox{0.95\textwidth}%
			{
			\BODY
			}%
		}%
	}

% Theorem environment
\NewEnviron{theorem}%
	{%
	\noindent\refstepcounter{example}\fcolorbox{gray!40}{gray!40}{\textsc{\textcolor{blue}{Theorem~\theexample.}}}%
	%\fcolorbox{black}{white}%
		{  %\parbox{0.95\textwidth}%
			{
			\BODY
			}%
		}%
	}
	

%%%%
\begin{document}
\thispagestyle{empty}

\begin{center}
\vspace*{2.5cm}

{\Huge \textsc{Math 302}}

\vspace*{2cm}

{\LARGE \textsc{Chapter 5}} 

\vspace*{0.75cm}

\noindent\textsc{Section 5.3: Nonhomogeneous Linear Equations}

\vspace*{0.75cm}

\tableofcontents

\vfill

\noindent \textsc{Created by: Pierre-Olivier Paris{\'e}} \\
\textsc{Fall 2022}
\end{center}

\newpage

\section{Particular Solutions}

Our goal is to find the solutions to
	\begin{align}
	y'' + p(x)y' + q(x) y = f(x) . \label{Eq:SecondODE}
	\end{align}
	
\underline{Nomenclature:}
	\begin{itemize}
	\item the equation $y'' + p(x) y' + q(x) y = 0$ is the \textbf{complementary equation} for \eqref{Eq:SecondODE}.
	\item a \textbf{particular solution} is a solution $y_{par}$ of \eqref{Eq:SecondODE}.
	\end{itemize}
	
\vspace*{16pt}

\begin{example}
Find a particular solution to the following ODE:
	\begin{align*}
	y'' - 2y' + y = 4x .
	\end{align*}
\end{example}

\newpage

\underline{Assumptions:}

	\begin{itemize}
	\item[1)] Suppose $\{ y_1 , y_2 \}$ is a fundamental set of solutions to
		\begin{align*}
		y'' + p(x) y' + q(x) y = 0 .
		\end{align*}
	\item[2)] Suppose $y_{par}$ is a particular solution to
		\begin{align*}
		y'' + p(x) y' + q(x) y = f(x) .
		\end{align*}
	\end{itemize}

\underline{Conclusion:}
	\begin{itemize}
	\item Then the $y = y_{par} + c_1 y_1 + c_2 y_2$ is the general solution of
		\begin{align*}
		y'' + p(x) y' + q(x) y = f(x) .
		\end{align*}
	\end{itemize}
	
	\vspace*{16pt}

	\begin{example}
	\begin{enumerate}[label=\alph*)]
	\item Find the general solution of
		\begin{align*}
		y'' - 2y' + y = -3 - x + x^2 .
		\end{align*}
	\item Solve the following IVP:
		\begin{align*}
		y'' - 2y' + y = -3 - x + x^2 , \quad y(0) = -2 , \, y'(0) = 1 .
		\end{align*}
	\end{enumerate}
	\end{example}
	
	\newpage
	
	\phantom{2}
	
	\newpage
	
	\section{The Principle of Superposition}
		
	\begin{example}
	Suppose that we know that $y_1 (x) = x^4/15$ is a particular solution to
		\begin{align*}
		x^2 y'' + 4xy' + 2y = 2x^4
		\end{align*}
	and that $y_2 (x) = x^2/3$ is a particular solution to
		\begin{align*}
		x^2 y'' + 4xy' + 2y = 4x^2 .
		\end{align*}
	Find a particular solution to
		\begin{align*}
		x^2 y'' + 4xy' + 2y = 2x^4 + 4x^2 .
		\end{align*}
	\end{example}
	
	\newpage
	
	\underline{General Fact:}
	If $y_1$ is a particular solution to
		\begin{align*}
		y'' + p (x) y' + q (x) y = f_1 (x)
		\end{align*}
	and $y_2$ is a particular solution to
		\begin{align*}
		y'' + p (x) y' + q (x) y = f _2(x)
		\end{align*}
	then $y_{par} = y_1 + y_2$ is a particular solution to
		\begin{align*}
		y'' + p(x) y' + q(x) y = f_1 (x) + f_2 (x) .
		\end{align*}

\end{document}
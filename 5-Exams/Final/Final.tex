\documentclass[addpoints, 12pt]{exam}%, answers]
\usepackage[utf8]{inputenc}
\usepackage[T1]{fontenc}

\usepackage{lmodern}
\usepackage{arydshln}
\usepackage[margin=2cm]{geometry}

\usepackage{enumitem}

\usepackage{amsmath, amsthm, amsfonts, amssymb}
\usepackage{graphicx}
\usepackage{tikz}
\usetikzlibrary{arrows,calc,patterns}
\usepackage{pgfplots}
\pgfplotsset{compat=newest}
\usepackage{url}
\usepackage{multicol}
\usepackage{thmtools}
\usepackage{wrapfig}

\usepackage{caption}
\usepackage{subcaption}

\usepackage{pifont}

% MATH commands
\newcommand{\bC}{\mathbb{C}}
\newcommand{\bR}{\mathbb{R}}
\newcommand{\bN}{\mathbb{N}}
\newcommand{\bZ}{\mathbb{Z}}
\newcommand{\bT}{\mathbb{T}}
\newcommand{\bD}{\mathbb{D}}

\newcommand{\cL}{\mathcal{L}}
\newcommand{\cM}{\mathcal{M}}
\newcommand{\cP}{\mathcal{P}}
\newcommand{\cH}{\mathcal{H}}
\newcommand{\cB}{\mathcal{B}}
\newcommand{\cK}{\mathcal{K}}
\newcommand{\cJ}{\mathcal{J}}
\newcommand{\cU}{\mathcal{U}}
\newcommand{\cO}{\mathcal{O}}
\newcommand{\cA}{\mathcal{A}}
\newcommand{\cC}{\mathcal{C}}
\newcommand{\cF}{\mathcal{F}}

\newcommand{\fK}{\mathfrak{K}}
\newcommand{\fM}{\mathfrak{M}}

\newcommand{\ga}{\left\langle}
\newcommand{\da}{\right\rangle}
\newcommand{\oa}{\left\lbrace}
\newcommand{\fa}{\right\rbrace}
\newcommand{\oc}{\left[}
\newcommand{\fc}{\right]}
\newcommand{\op}{\left(}
\newcommand{\fp}{\right)}

\newcommand{\ra}{\rightarrow}
\newcommand{\Ra}{\Rightarrow}

\renewcommand{\Re}{\mathrm{Re}\,}
\renewcommand{\Im}{\mathrm{Im}\,}
\newcommand{\Arg}{\mathrm{Arg}\,}
\newcommand{\Arctan}{\mathrm{Arctan}\,}
\newcommand{\sech}{\mathrm{sech}\,}
\newcommand{\csch}{\mathrm{csch}\,}
\newcommand{\Log}{\mathrm{Log}\,}
\newcommand{\cis}{\mathrm{cis}\,}

\newcommand{\ran}{\mathrm{ran}\,}
\newcommand{\bi}{\mathbf{i}}
\newcommand{\Sp}{\mathrm{span}\,}
\newcommand{\Inv}{\mathrm{Inv}\,}
\newcommand\smallO{
  \mathchoice
    {{\scriptstyle\mathcal{O}}}% \displaystyle
    {{\scriptstyle\mathcal{O}}}% \textstyle
    {{\scriptscriptstyle\mathcal{O}}}% \scriptstyle
    {\scalebox{.7}{$\scriptscriptstyle\mathcal{O}$}}%\scriptscriptstyle
  }
\newcommand{\HOL}{\mathrm{Hol}}
\newcommand{\cl}{\mathrm{clos}}
\newcommand{\ve}{\varepsilon}

\DeclareMathOperator{\dom}{dom}

%%%%%% Définitions Theorems and al.
%\declaretheoremstyle[preheadhook = {\vskip0.2cm}, mdframed = {linewidth = 2pt, backgroundcolor = yellow}]{myThmstyle}
%\declaretheoremstyle[preheadhook = {\vskip0.2cm}, postfoothook = {\vskip0.2cm}, mdframed = {linewidth = 1.5pt, backgroundcolor=green}]{myDefstyle}
%\declaretheoremstyle[bodyfont = \normalfont , spaceabove = 0.1cm , spacebelow = 0.25cm, qed = $\blacktriangle$]{myRemstyle}

%\declaretheorem[ style = myThmstyle, name=Th\'eor\`eme]{theorem}
%\declaretheorem[style =myThmstyle, name=Proposition]{proposition}
%\declaretheorem[style = myThmstyle, name = Corollaire]{corollary}
%\declaretheorem[style = myThmstyle, name = Lemme]{lemma}
%\declaretheorem[style = myThmstyle, name = Conjecture]{conjecture}

%\declaretheorem[style = myDefstyle, name = D\'efinition]{definition}

%\declaretheorem[style = myRemstyle, name = Remarque]{remark}
%\declaretheorem[style = myRemstyle, name = Remarques]{remarks}

\newtheorem{theorem}{Théorème}
\newtheorem{corollary}{Corollaire}
\newtheorem{lemma}{Lemme}
\newtheorem{proposition}{Proposition}
\newtheorem{conjecture}{Conjecture}

\theoremstyle{definition}

\newtheorem{definition}{Définition}[section]
\newtheorem{example}{Exemple}[section]
\newtheorem{remark}{\textcolor{red}{Remarque}}[section]
\newtheorem{exer}{\textbf{Exercice}}[section]


\tikzstyle{myboxT} = [draw=black, fill=black!0,line width = 1pt,
    rectangle, rounded corners = 0pt, inner sep=8pt, inner ysep=8pt]

\begin{document}
	\noindent \hrulefill \\
	\noindent MATH-302 \hfill Created by Pierre-O. Paris{\'e}\\
	Final\hfill 2022/15/12, Fall 2022\\\vspace*{-0.7cm}

\noindent\hrulefill
	
\vspace*{1cm}

\noindent\makebox[\textwidth]{\textbf{Last name:}\enspace \hrulefill}
\makebox[\textwidth]{\textbf{First name:}\enspace\hrulefill}

\vspace*{1cm}

\vspace*{1cm}

\noindent\textbf{Instructions:} 

\begin{itemize}
\item Make sure to write your complete name on your copy. 
\item You must answer all the questions below and write your answers directly on the questionnaire.
\item You have 120 minutes (2 hours) to complete the exam.
\item When you are done (or at the end of the 120min period), return your copy. 
\item No devices such as a smart phone, cell phone, laptop, or tablet can be used during the exam. 
\item \textbf{Turn off your cellphone during the exam}.
\item You may use a digital calculator (no graphical calculators or symbolic calculators will be allowed).
\item You are not allowed to use the lecture notes or the textbook.
\item You may bring one 2-sided cheat sheet of handwriting notes. 
\item You must show ALL your work to have full credit. An answer without justification is worth no point.
\end{itemize}

\vspace{0.5cm}

\noindent\textbf{Your Signature:} \hrulefill

\vspace*{2cm}
\noindent May the Force be with you! \hfill Pierre-Olivier Parisé

\vspace*{1cm}

\begin{center}
\begin{minipage}{0.29\textwidth}
\begin{Huge}
\textsc{University of Hawai'i}
\end{Huge}
\end{minipage}
\begin{minipage}{0.12\textwidth}
\includegraphics[scale=0.05]{../../../../manoaseal_transparent.png}
\end{minipage}
\end{center}

\qformat{\rule{0.3\textwidth}{.4pt} \begin{large}{\textsc{Question}} \thequestion \end{large} \hspace*{0.2cm} \hrulefill \hspace*{0.1cm} \textbf{(\totalpoints\hspace*{0.1cm} pts)}}

\vspace*{0.5cm}

\newpage

\begin{questions}

\question[20]
Find the solution of the following ODE using the power series method.
	\begin{align*}
	(1 + x^2)y'' + x y' + y = 0 , \quad y(0) = 2 , \, y'(0) = -1 .
	\end{align*}
Give only the first five coefficients of the power series solution.
	
\newpage

\phantom{2}

\newpage
	
\question
Answer the following questions.
	\begin{parts}
	\part[10]
	Find the Laplace transform of $f(t) = t e^t \cos (2t)$.
	
	\vfill
	
	\part[10]
	Find the inverse Laplace transform of $F(s) = \displaystyle \frac{1}{(s - 2) (s + 3)}$.
	\vfill
	
	\end{parts}
	
\newpage

\question
Answer the following questions.
	\begin{parts}
	\part[10]
	Find the Laplace transform of the function
		\begin{align*}
		f(t) = \left\{ \begin{matrix}
		t - 1 & 0 \leq t < 1 \\
		t + 1 & 1 \leq t .
		\end{matrix} \right.
		\end{align*}
		
	\vfill
		
	\part[10]
	Find the inverse Laplace transform of the function $F(s) = \displaystyle \frac{e^{-s}}{(s + 1)^2}$.
	
	\vfill
	\end{parts}

\newpage

\question[20]
Find the solution to the following IVP using the Laplace transform:
	\begin{align*}
	y'' -4 y' - 5y = 0 , \quad y(0) = 1 , \, y'(0) = 0 .
	\end{align*}
	

\newpage

\question

	\begin{parts}
	\part[5]
	Denote by $F(s)$ the Laplace transform of $f(t)$. Show that if $h(t) = \displaystyle \int_0^t x f(x) \, dx$, then $L(h(t)) = -\displaystyle\frac{F'(s)}{s}$.
	
	\vfill
	
	\part[5]
	%Find the solution to the following IVP
	%	\begin{align*}
	%	(2t^2 + 4t + 3) \frac{d^2y}{dt^2} + 10 (t + 1) \frac{dy}{dt} + 8y(t) = 0 , \quad y(-1) = 2, \, y'(-1) = -3 
	%	\end{align*}			
	%given that the solution of
	%	\begin{align*}
	%	(1 + 2x^2) y'' + 10 xy' + 8y = 0, \quad y(0) = 2 , \, y'(0) = -3
	%	\end{align*}
	%is $y(x) = 2 - 3x - 8x^2 + 9x^3 + \frac{64}{3} x^4 - \frac{45}{2} x^5 - \frac{256}{5} x^6 + \frac{105}{2} x^7 + \cdots$
	Find the solution of the following integral equation:
		\begin{align*}
		y(t) = 1 + \int_0^t y(x) \, dx .
		\end{align*}
	\vfill
	
	\end{parts}
	
\newpage

\question
Answer the following statements with \textbf{True} or \textbf{False}. Write your answer on the horizontal line at the end of each statement. Justify your answer in the white space underneath each statement.

	\begin{parts}
	\pointformat{(\hspace*{0.5cm}/ \thepoints)}
	\pointname{}
	%\pointsinrightmargin
	
	\part[2]
	The radius of convergence of the power series solution $\sum_{n = 0}^\infty a_n (x - 3)^n$ of the ODE $(16 + x^2)y'' + xy' + y = 0$ is $5$.
	\begin{solution}[\stretch{1}]
	
	\end{solution}
	\answerline[True]
	
	\part[2]
	If $f(t) = t$ and $g(t) = t^2$, then $L (f(t)g(t)) = \displaystyle \frac{2}{s^5}$.
	\begin{solution}[\stretch{1}]
	
	\end{solution}
	\answerline[True]
	
	\part[2]
	If $f(t) = 0$ for $t < 2$, $f(t) = 2$ for $2 \leq t < 3$ and $f(t) = t$ for $t \geq 3$, then $f(t) = 2u (t-2) + (t - 2) u(t-3)$.
	\begin{solution}[\stretch{1}]
	
	\end{solution}
	\answerline[True]
	
	\part[2]
	If $f(t) = t^2$ and $g(t) = t^2$, then $f(t) \ast g(t) = \displaystyle \frac{t^5}{30}$.
	\begin{solution}[\stretch{1}]
	
	\end{solution}
	\answerline[True]
	
	\part[2]
	The number $x = 0$ is a singular point of the ODE $(x^2 + x) y'' + x y' + y = 0$.
	\begin{solution}[\stretch{1}]
	
	\end{solution}
	\answerline[True]
	
	\end{parts}
	
	\newpage
	
	\begin{Large}
	\textsc{Do not write on this page.}
	\end{Large}
	
	\vfill
	
	\textit{For officials use only:}
	\begin{center}
	\gradetable[h][questions]
	\end{center}
	

\end{questions}


\end{document}
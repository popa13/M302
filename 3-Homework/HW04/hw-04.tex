\documentclass[12pt]{article}
\usepackage[utf8]{inputenc}

\usepackage{lmodern}

\usepackage{enumitem}
\usepackage[margin=2cm]{geometry}

\usepackage{amsmath, amsfonts, amssymb}
\usepackage{graphicx}
\usepackage{subfigure}
\usepackage{tikz}
\usepackage{pgfplots}
\usepackage{multicol}

\usepackage{comment}
\usepackage{url}
\usepackage{calc}
%\usepackage{subcaption}
\usepackage[indent=0pt]{parskip}

\usepackage{array}
\usepackage{blkarray,booktabs, bigstrut}

\pgfplotsset{compat=1.16}

% MATH commands
\newcommand{\ga}{\left\langle}
\newcommand{\da}{\right\rangle}
\newcommand{\oa}{\left\lbrace}
\newcommand{\fa}{\right\rbrace}
\newcommand{\oc}{\left[}
\newcommand{\fc}{\right]}
\newcommand{\op}{\left(}
\newcommand{\fp}{\right)}

\newcommand{\bi}{\mathbf{i}}
\newcommand{\bj}{\mathbf{j}}
\newcommand{\bk}{\mathbf{k}}
\newcommand{\bF}{\mathbf{F}}

\newcommand{\mR}{\mathbb{R}}

\newcommand{\ra}{\rightarrow}
\newcommand{\Ra}{\Rightarrow}

\newcommand{\sech}{\mathrm{sech}\,}
\newcommand{\csch}{\mathrm{csch}\,}
\newcommand{\curl}{\mathrm{curl}\,}
\newcommand{\dive}{\mathrm{div}\,}

\newcommand{\ve}{\varepsilon}
\newcommand{\spc}{\vspace*{0.5cm}}

\DeclareMathOperator{\Ran}{Ran}
\DeclareMathOperator{\Dom}{Dom}

\newcommand{\exo}[3]{\noindent\textcolor{red}{\fbox{\textbf{Section {#1} | Problem {#2} | {#3} points}}}\\}

\begin{document}
	\noindent \hrulefill \\
	MATH-302 \hfill Pierre-Olivier Paris{\'e}\\
	Homework 4 Solutions \hfill Fall 2022\\\vspace*{-1cm}
	
	\noindent\hrulefill
	
	\spc
	
	\exo{4.2}{5}{25}
	\\
	\begin{enumerate}[label=(\alph*)]
	\item (20 points) We have $T_m = 35$ and $T_0 = 150$. Threfore, the expression of $T(t)$ is
		\begin{align*}
		T(t) = 35 + 115 e^{-k t} .
		\end{align*}
	Let $t_0$ be the time taking for the object to drop to the temperature of $120$ and let $t_1$ be the time for the object to drop to the temperature of $90$. Then, using the formula of $T$, we have
		\begin{align*}
		120 &= T(t_0) = 35 + 115 e^{-k t_0} , \\
		90 &= T(t_1) = 35 + 115 e^{-k t_1} .
		\end{align*}
	Therefore, after substracting $35$, dividing by $115$ and taking the $\ln$ on each side, we obtain
		\begin{align*}
		\ln (17 / 23) = -k t_0 , \\
		\ln (12/23) = -k t_1 .
		\end{align*}
	Taking the difference of the first equation with the second equation, we get
		\begin{align*}
		k(t_1 - t_0) = \ln (17/23) - \ln (12/23) = \ln (17/12) .
		\end{align*}
	However, we know that $t_1 - t_0 = 5$ because $t_0$ and $t_1$ represents the time between the beginning and 12:15\textsc{pm} and 12:20\textsc{pm} respectively. Therefore, we obtain
		\begin{align*}
		k = \frac{\ln (17 / 12)}{5} .
		\end{align*}
	Plugging this in the expression of $T$, we obtain
		\begin{align*}
		T(t) = 35 + 115 e^{-\frac{\ln(17/12)}{5} t} .
		\end{align*}
	Now that we have the constant $k$, we can now find one of the values of $t_0$ or $t_1$. We can then substract $t_0$ from the time 12:15\textsc{pm} to obtain the time the object was moved outside. From the equation $\ln (17/23) = -k t_0$, we can find $t_0$:
		\begin{align*}
		t_0 = - \frac{5\ln (17/23)}{\ln (17/12)} \approx 4.33929 \approx 4 \tfrac{1}{3} .
		\end{align*}
	The units of $t_0$ are $\mathrm{min}$ and therefore the time that the object was moved outside is approximately 12:10:40\textsc{pm}.
	\item (5 points) We have to solve
		\begin{align*}
		40 = 35 + 115 e^{- \frac{\ln (17/12)}{5} t} .
		\end{align*}
	We find $t \approx 45$ and therefore at 12:55:40\textsc{pm}.
	\end{enumerate}
	
	\newpage
	
	\exo{4.2}{11}{25}
	\\
	The initial volume of water is $V_0 = 100$ gallons and it contains initially $Q_0 = 20\text{lb}$ of salt. Let $Q(t)$ be the quantity of salt in the tank and let $V(t)$ be the volume of water in the tank. 
	
	There are $4$ gallons of product per minute coming in the tank and there are $2$ gallons of product per minute coming out the tank. So, the differential equation modeling the volume of stuff in the tank is
		\begin{align*}
		\frac{dV}{dt} = 4 - 2 = 2 .
		\end{align*}
	Therefore, we have $V(t) = 2t + c$ for some constant $c$. Initially, we have $V_0 = 100$ and $c = 100$. Therefore, we obtain 
		\begin{align*}
		V (t) = 2t + 100 .
		\end{align*}
	
	The concentration of salt coming in the tank is $(1/4) \cdot 4 = 1 \, \mathrm{lb}/\mathrm{min}$. The concentration of salt coming out of the tank is $\frac{Q(t)}{V(t)} \cdot 2 = 2 Q(t)/(2t + 100) \,  \mathrm{lb}/\mathrm{min}$. The differential equation modeling $Q(t)$ is then
		\begin{align*}
		\frac{dQ}{dt} = 1 - 2\frac{Q}{2t + 100} \iff \frac{dQ}{dt} + 2 \frac{Q}{2t + 100} = 1 .
		\end{align*}
	
	To solve this differential equation, we first solve the complementary equation $Q' + 2 Q/(2t+ 100) = 0$. We separate the variables:
		\begin{align*}
		\frac{Q'}{Q} = -\frac{2}{2t + 100} \quad \Ra \quad \ln Q = -\ln (2t + 100) + K .
		\end{align*}
	Therefore, $Q (t) = c/(2t + 100)$ where $c = e^K$. We didn't use the absolute value because $Q > 0$ and $t \geq 0$ so that $2t + 100 \geq 100$. 
	
	We use variation of parameter to solve the initial EDO. Let $Q(t) = u(t) / (2t + 100)$. Therefore, 
		\begin{align*}
		Q' = \frac{u'}{2t + 100} - \frac{2u}{(2t + 100)^2} .
		\end{align*}	
	Replacing $Q$ and $Q'$ in the initial differential equation, we get
		\begin{align*}
		& \frac{u'}{2t + 100} - \frac{2u}{(2t + 100)^2} + \frac{2u}{(2t + 100)^2} = 1 \\
		\iff & u' = 2t + 100 \\
		\iff & u (t) = t^2 + 100 t + c .
		\end{align*}		
	Therefore, we find that
		\begin{align*}
		Q(t) = \frac{t^2 + 100 t}{2t + 100} + \frac{c}{2t + 100} .
		\end{align*}	
	
	At $t = 0$, we know that $Q_0 = 20$. Therefore, we get
		\begin{align*}
		20 = \frac{c}{100} \quad \Ra \quad 2000 = c .
		\end{align*}	
	So the quantity of salt at time $t$ is
		\begin{align*}
		Q(t) = \frac{t^2 + 100t}{2t + 100} + \frac{2000}{2t + 100} .
		\end{align*}
	
	Using the function for the volume, we know that the maximum it can reach is $200\mathrm{gal}$, we find $200 = 2t + 100$ which is $t = 50$. After $50\mathrm{min}$, the tank will overflow and so we replace this value of $t$ in the expression of $Q(t)$ to get
		\begin{align*}
		Q(50) = 47.5\, \mathrm{lb} .
		\end{align*}							 
	
\end{document}
\documentclass[12pt]{article}
\usepackage[utf8]{inputenc}

\usepackage{lmodern}

\usepackage{enumitem}
\usepackage[margin=2cm]{geometry}

\usepackage{amsmath, amsfonts, amssymb}
\usepackage{graphicx}
\usepackage{subfigure}
\usepackage{tikz}
\usepackage{pgfplots}
\usepackage{multicol}

\usepackage{comment}
\usepackage{url}
\usepackage{calc}
%\usepackage{subcaption}
\usepackage[indent=0pt]{parskip}

\usepackage{array}
\usepackage{blkarray,booktabs, bigstrut}

\pgfplotsset{compat=1.16}

% MATH commands
\newcommand{\ga}{\left\langle}
\newcommand{\da}{\right\rangle}
\newcommand{\oa}{\left\lbrace}
\newcommand{\fa}{\right\rbrace}
\newcommand{\oc}{\left[}
\newcommand{\fc}{\right]}
\newcommand{\op}{\left(}
\newcommand{\fp}{\right)}

\newcommand{\bi}{\mathbf{i}}
\newcommand{\bj}{\mathbf{j}}
\newcommand{\bk}{\mathbf{k}}
\newcommand{\bF}{\mathbf{F}}

\newcommand{\mR}{\mathbb{R}}

\newcommand{\ra}{\rightarrow}
\newcommand{\Ra}{\Rightarrow}

\newcommand{\sech}{\mathrm{sech}\,}
\newcommand{\csch}{\mathrm{csch}\,}
\newcommand{\curl}{\mathrm{curl}\,}
\newcommand{\dive}{\mathrm{div}\,}

\newcommand{\ve}{\varepsilon}
\newcommand{\spc}{\vspace*{0.5cm}}

\DeclareMathOperator{\Ran}{Ran}
\DeclareMathOperator{\Dom}{Dom}

\newcommand{\exo}[3]{\noindent\textcolor{red}{\fbox{\textbf{Section {#1} | Problem {#2} | {#3} points}}}\\}
\newcommand{\qu}[2]{\noindent\textcolor{red}{\fbox{\textbf{Section {#1} | Problem {#2}}}}\\}

\begin{document}
	\noindent \hrulefill \\
	MATH-302 \hfill Pierre-Olivier Paris{\'e}\\
	Homework 10 Solutions \hfill Fall 2022\\\vspace*{-1cm}
	
	\noindent\hrulefill
	
	\spc
	
	\exo{7.2}{1}{25}
	\\
	We let $y (x) = \sum_{n = 0}^\infty a_n x^n$. We have
		\begin{align*}
		y'(x) = \sum_{n = 1}^\infty n a_n x^{n-1} \quad \text{and} \quad y'' (x) = \sum_{n = 2}^\infty n (n - 1) a_n x^{n-2} .
		\end{align*}
	Therefore, we can rewrite
		\begin{align*}
		(1 + x^2) y'' = \sum_{n= 2}^\infty n (n - 1) a_n x^{n - 2} + x^2 \sum_{n = 2}^\infty n (n - 1) x^{n-2} = \sum_{n = 2}^\infty n (n - 1) a_n x^{n-2} + \sum_{n=2}^\infty n (n - 1) a_n x^n
		\end{align*}
	and
		\begin{align*}
		6x y' = 6x \sum_{n = 1}^\infty n a_n x^{n-1} = \sum_{n=1}^\infty 6n a_n x^n .
		\end{align*}
	The expression of the left-hand side of the ODE is then
		\begin{align*}
		\sum_{n = 2}^\infty n (n - 1) a_n x^{n-2} + \sum_{n = 2}^\infty n (n - 1) a_n x^n + \sum_{n = 1}^\infty 6n a_n x^n + \sum_{n = 0}^\infty 6 a_n x^n .
		\end{align*}
	Shifting the first summation, we get
		\begin{align*}
		\sum_{n = 0}^\infty (n + 2) (n + 1) a_{n + 2} x^{n} + \sum_{n = 2}^\infty n (n - 1) a_n x^n + \sum_{n = 1}^\infty 6n a_n x^n + \sum_{n = 0}^\infty 6 a_n x^n .
		\end{align*}
	Combining similar powers together, we obtain
		\begin{align*}
		2 a_2 + 6 a_3x + 6a_1x + 6a_0 + 6a_1x + \sum_{n = 2}^\infty \big[ (n + 2) (n + 1) a_{n+2} + ( n (n - 1) + 6n + 6) a_n \big] x^n
		\end{align*}
	and therefore
		\begin{align*}
		(6a_0 + 2a_2) + (12 a_1 + 6a_3) x + \sum_{n = 2}^\infty \big[ (n + 2) (n + 1) a_{n + 2} + ( n (n - 1) + 6n + 6) a_n \big]  x^n = 0 = \sum_{n = 0}^\infty 0 x^n .
		\end{align*}
	Equating each coefficients with the same powers, we obtain
		\begin{align*}
		a_2 = -3a_0 , \quad a_3 = -2a_1 \quad \text{ and } \quad a_{n + 2} = - \frac{n^2 + 5n + 6}{n^2 + 3n + 2} a_n
		\end{align*}
	We can get a sample of the list of $a_n$ for even indexed coefficients:
		\begin{itemize}
		\item $a_0$ arbitrary;
		\item $a_2 = -3 a_0$;
		\item $a_4 = - \frac{2^2 + 5 \cdot 2 + 6}{2^2 + 3 \cdot 2 + 2} a_2 = - \frac{20}{12} (-3 a_0) = 5 a_0$;
		\item $a_6 = - \frac{4^2 + 5 \cdot 4 + 6}{4^2 + 3 \cdot 4 + 2} a_4 = - \frac{42}{30} 5 a_0 = - 7 a_0$;
		\item $a_8 = - \frac{6^2 + 5 \cdot 6 + 6}{6^2 + 3 \cdot 6 + 2} a_6 = - \frac{72}{56} (-7 a_0) = 9 a_0$.
		\end{itemize}
	We see a pattern. The general pattern is $a_{2n} = (-1)^n (2n + 1) a_0$, for $n \geq 1$. We can do the same for the coefficients indexed by odd integers:
		\begin{itemize}
		\item $a_1$ arbitrary;
		\item $a_3 = -2a_1$;
		\item $a_5 = - \frac{3^2 + 5 \cdot 3 + 6}{3^2 + 3 \cdot 3 + 2} a_3 = - \frac{30}{20} (-2 a_1) = 3 a_1$;
		\item $a_7 = - \frac{5^2 + 5 \cdot 5 + 6}{5^2 + 3 \times 5 + 2} a_5 = - \frac{56}{42} 3 a_1 = -4 a_1$;
		\item $a_9 = - \frac{7^2 + 5 \cdot 7 + 6}{7^2 + 3 \cdot 7 + 2} a_7 = - \frac{90}{72} (-4 a_1) = 5 a_1$.
		\end{itemize}
	We see a pattern. The general pattern is $a_{2n + 1} = (-1)^n (n + 1) a_1$. Therefore, our final answer is
		\begin{align*}
		y(x) &= a_0 + a_1 x + \sum_{n = 1}^\infty (-1)^n (2n + 1) a_0 x^{2n} + \sum_{n = 1}^\infty (-1)^n (n + 1) a_1 x^{2n + 1} \\
		&= a_0 \sum_{n = 0}^\infty (-1)^n (2n + 1) x^{2n} + a_1 \sum_{n = 0}^\infty (-1)^n (n + 1) x^{2n + 1} .
		\end{align*}
		
	\underline{Remark:} For those interested, we can find an explicit expression of the power series solution. First, notice that integrating term-by-term the first series in front of $a_0$ gives
		\begin{align*}
		\int_0^x \sum_{n = 0}^\infty (-1)^n (2n + 1) t^{2n} \, dt = \sum_{n = 0}^\infty (-1)^n (2n + 1) \left. \frac{t^{2n + 1}}{2n + 1} \right|_{t = 0}^{t = x} = \sum_{n = 0}^\infty (-1)^n x^{2n + 1} .
		\end{align*}
	Rewrite the last series as followed:
		\begin{align*}
		\sum_{n = 0}^\infty (-1)^n x^{2n + 1} = x \sum_{n = 0}^\infty (-x^2)^n
		\end{align*}
	and then use the power series of $(1 - x)^{-1}$ with $-x^2$ in place of $x$ to get
		\begin{align*}
		x \sum_{n = 0}^\infty (-x^2)^n = \frac{x}{1 + x^2} .
		\end{align*}
	This is valid only when $-1 < x < 1$. To obtain the original series, we use the Fundamental Theorem of Calculus and take the derivative of the last expression we obtained:
		\begin{align*}
		\sum_{n = 0}^\infty (-1)^n (2n + 1) x^{2n} = \frac{d}{dx} \Big( \frac{x}{1 + x^2} \Big) = \frac{1 - x^2}{(1 + x^2)^2} .
		\end{align*}
		
	For the power series in front of $a_1$, we will manipulate algebraically the expression. A simple algebra trick leads to
		\begin{align*}
		\sum_{n = 0}^\infty (-1)^n (n + 1) x^{2n + 1} = \frac{1}{2} \sum_{n =0}^\infty (-1)^n (2n + 2) x^{2n + 1} = \frac{1}{2} \sum_{n = 0}^\infty (-1)^n (2n + 1) x^{2n + 1} + \frac{1}{2} \sum_{n = 0}^\infty (-1)^n x^{2n + 1} .
		\end{align*}
	Now, we see that
		\begin{align*}
		\sum_{n =0}^\infty (-1)^n (2n + 1) x^{2n + 1} = x \sum_{n = 0}^\infty (2n + 1) x^{2n} = x \Big( \frac{1 - x^2}{(1 + x^2)^2} \Big) = \frac{x - x^3}{(1 + x^2)^2} .
		\end{align*}
	We also see, using the power series of $(1 - x)^{-1}$ with $-x^2$ in place of $x$ that
		\begin{align*}
		\sum_{n = 0}^\infty (-1)^n x^{2n + 1} = x \sum_{n = 0}^\infty (-x^2)^n = \frac{x}{1 + x^2} .
		\end{align*}
	Combining everything together, we get
		\begin{align*}
		\sum_{n = 0}^\infty (-1)^n (n + 1) x^{2n + 1} &= \frac{1}{2} \Big( \frac{x - x^3}{(1 + x^2)^2} \Big) + \frac{1}{2} \Big( \frac{x}{1 + x^2} \Big) \\
		&= \frac{1}{2} \Big( \frac{x - x^3 + x (1 + x^2)}{ (1 + x^2)^2} \Big) \\
		&= \frac{x}{(1+ x^2)^2} .
		\end{align*}
		
	Finally, we can rewrite the power series solution (valid for $-1 < x < 1$) as
		\begin{align*}
		y(x) = a_0 \frac{1 - x^2}{(1 + x^2)^2} + a_1  \frac{x}{(1 + x^2)^2} .
		\end{align*}
	You can check that $y_1 (x) := \frac{1 - x^2}{(1 + x^2)^2}$ is a solution to the ODE of the problem and $y_2 (x) := \frac{x}{(1 + x^2)^2}$ is also a solution to the ODE. We see that
		\begin{align*}
		\frac{y_1}{y_2} = \frac{1- x^2}{x} = \frac{1}{x} - x
		\end{align*}
	which is not constant. Therefore, $\{ y_1 , y_2 \}$ is a fundamental set of solutions for the ODE! Isn't it beautiful ;)
	
	\newpage
	
	\exo{7.2}{7}{25}
	\\
	Write $y(x) = \sum_{n = 0}^\infty a_n x^n$. Using the rules to differentiate power series, we get
		\begin{align*}
		y' (x) = \sum_{n = 1}^\infty n a_n x^{n-1} \quad \text{ and } \quad y'' (x) = \sum_{n = 2}^\infty n (n - 1) a_n x^{n - 2} .
		\end{align*}
	Therefore, we get
		\begin{align*}
		(1 - x^2) y'' (x) = \sum_{n = 2}^\infty n (n - 1) a_n x^{n - 2} - \sum_{n = 2}^\infty n (n - 1) a_n x^n
		\end{align*}
	and
		\begin{align*}
		5x y' (x) = \sum_{n = 1}^\infty 5n a_n x^n .
		\end{align*}
	Putting that into the left-hand side of the ODE, we get
		\begin{align*}
		& \sum_{n = 2}^\infty n (n - 1) a_n x^{n - 2} - \sum_{n = 2}^\infty n (n - 1) a_n x^n - \sum_{n = 1}^\infty 5n a_n x^n  - \sum_{n = 0}^\infty 4a_n x^n \\
		& = \sum_{n = 0}^\infty (n + 2) (n + 1) a_{n + 2} x^n - \sum_{n = 2}^\infty n (n - 1) a_n x^n - \sum_{n = 1}^\infty 5n a_n x^n - \sum_{n = 0}^\infty 4a_n x^n \\
		& = a_2 + a_3 x - 5a_1 x - 4a_0 - 4a_1 x + \sum_{n = 2}^\infty \Big( (n + 2) (n + 1) a_{n + 2} - \big[ n ( n -1) + 5n + 4 \big] a_n \Big) x^n \\
		& = (a_2 - 4a_0) + (a_3 - 9 a_1)x + \sum_{n = 2}^\infty \Big( (n + 2) (n + 1) a_{n + 2} - \big[ n ( n -1) + 5n + 4 \big] a_n \Big) x^n .
		\end{align*}
	Equating the left-hand side with the right-hand side, we get
		\begin{align*}
		a_2 = 4 a_0 , \quad a_3 = 9 a_1 \quad \text{ and } \quad a_{n + 2} = \frac{n^2 + 4n + 4}{(n + 2)(n + 1)} a_{n} .
		\end{align*}
	Since $n^2 + 4n + 4 = (n + 2)^2$, we get
		\begin{align*}
		a_2 = 4 a_0 , \quad a_3 = 9 a_1 \quad \text{ and } \quad a_{n + 2} = \frac{n + 2}{n + 1} a_{n} .
		\end{align*}
	For even integers, we see that
		\begin{itemize}
		\item $a_4 = \frac{4}{3} a_2 = \frac{4}{3} 4 a_0 = \frac{4 \cdot 2}{3 \cdot 1} 2 a_0$;
		\item $a_6 = \frac{6}{5} a_4 = \frac{6 \cdot 4 \cdot 2}{5 \cdot 3 \cdot 1} 2 a_0$;
		\item $a_8 = \frac{8}{7} a_6 = \frac{8 \cdot 6 \cdot 4 \cdot 2}{7 \cdot 5 \cdot 3 \cdot 1} 2 a_0$;
		\item In general $a_{2n + 2} = \frac{(2n + 2)!!}{(2n + 1)!!} (2a_0)$ where $k!!$ is the double factorial of an integer $k$ (look it up on Google ;) ).
		\end{itemize}
	It is more appropriate to write the general rule as $a_{2n} = \frac{(2n)!!}{(2n - 1)!!} (2a_0)$. For odd indexes, we have
		\begin{itemize}
		\item $a_5 = \frac{5}{4} a_3 = \frac{5}{4} 9a_1 = \frac{5 \cdot 3}{4 \cdot 2} (6 a_1)$;
		\item $a_7 = \frac{7}{6} a_5 = \frac{7 \cdot 5 \cdot 3}{6 \cdot 4 \cdot 2} (6 a_1)$;
		\item $a_9 = \frac{9}{8} a_7 = \frac{9 \cdot 7 \cdot 5 \cdot 3}{8 \cdot 6 \cdot 4 \cdot 2} (6a_1)$;
		\item In general $a_{2n + 3} = \frac{(2n + 3)!!}{(2n + 2)!!} (6a_1)$.
		\end{itemize}
	It is more appropriate to write the general rule as $a_{2n + 1} = \frac{(2n + 1)!!}{(2n)!!} (6a_1)$.
	
	Therefore, the general solution looks like
		\begin{align*}
		y(x) &= a_0 + a_1 x + 2 a_0 \sum_{n = 1}^\infty \frac{(2n)!!}{(2n - 1)!!} x^{2n} + 6a_1 \sum_{n = 1}^\infty \frac{(2n + 1)!!}{(2n)!!} x^{2n + 1} \\
		&= a_0 \Big( 1 + 2 \sum_{n = 1}^\infty \frac{(2n)!!}{(2n-1)!!} x^{2n} \Big) + a_1 \Big( x + 6 \sum_{n = 1}^{\infty} \frac{(2n + 1)!!}{(2n)!!} x^{2n + 1} \Big) .
		\end{align*}
	
	
	\underline{More details}: We can rewrite the second series as a closed expression. By integrating terms-by-terms, we get
		\begin{align*}
		\sum_{n = 1}^\infty \frac{(2n + 1)!!}{(2n)!!} \frac{x^{2n + 2}}{2n + 2} = \sum_{n = 1}^\infty \frac{(2n + 1)!!}{(2n + 2)!!} x^{2n + 2} = \sum_{n = 2}^\infty \frac{(2n - 1)!!}{(2n)!!} x^{2n} .
		\end{align*}
	The power series representation of $(1 - x^2)^{-1/2}$ (for $-1 < x < 1$) is
		\begin{align*}
		1 + \frac{1}{2}x^2 + \frac{1 \cdot 3}{2 \cdot 4} x^4 + \frac{1 \cdot 3 \cdot 5}{2 \cdot 4 \cdot 6} x^6 + \cdots = 1 + \frac{1}{2} x^2 + \sum_{n = 2}^\infty \frac{(2n - 1)!!}{(2n)!!} x^{2n}
		\end{align*}
	and therefore, we obtain
		\begin{align*}
		\sum_{n = 2}^\infty \frac{(2n - 1)!!}{(2n)!!} x^{2n} = (1 - x^2)^{-1/2} - 1 - \frac{1}{2} x^2 .
		\end{align*}
	Since we took the integral, undoing this process is differentiation. So, after differentiating, we obtain
		\begin{align*}
		\sum_{n = 1}^\infty  \frac{(2n + 1)!!}{(2n)!!} x^{2n + 1} = \frac{2x}{(1 - x^2)^{3/2}} - x .
		\end{align*}
	Therefore, setting $y_2 (x) = x + 6 \sum_{n= 1}^\infty \frac{(2n + 1)!!}{(2n)!!} x^{2n + 1}$, we get
		\begin{align*}
		y_2(x) = \frac{12x}{(1 - x^2)^{3/2}} - 5x .
		\end{align*}
	
	For the power series in front for $a_0$, we will use another strategy: Variation of parameters. We know one solution of the ODE. So, let 
		\begin{align*}
		y(x) = u(x) y_2 (x) .
		\end{align*}
	Therefore, we get 
		\begin{align*}
		y' (x) = u' (x) y_2 (x) + u(x) y_2'(x) \quad \text{ and } \quad y'' (x) = u'' (x) y_2 + 2 u'(x) y_2 (x) + u(x) y_2''(x) .
		\end{align*}
	Replacing in the ODE, we see that
		\begin{align*}
		(1 - x^2) (u'' y_2 + 2 u'y_2+ u y_2'') - 5x (u'y_2 + u y_2') - 4 uy_2
		\end{align*}
	which can be simplified to
		\begin{align*}
		(1 - x^2) (u'' y_2 + 2u' y_2) - 5x u' y_2 + u \Big( (1 - x^2) y_2'' - 5x y_2' - 4y_2) \Big) .
		\end{align*}
	Because $y_2$ is a solution, the last expression can be simplified to
		\begin{align*}
		(1 - x^2) u'' y_2+ ((1 - x^2) 2y_2 - 5x y_2) u' .
		\end{align*}
	Letting $z = u'$, then the ODE becomes
		\begin{align*}
		(1 - x^2) y_2 z' + ( (1 - x^2) 2y_2 - 5x y_2) z = 0
		\end{align*}
	and solving for $z$, we obtain
		\begin{align*}
		\frac{z'}{z} = -\frac{ (1 - x^2) 2y_2 - 5x y_2}{(1 - x^2)y_2} = - \frac{2 (1 - x^2) - 5x}{1 - x^2} = -2 + \frac{5x}{1 - x^2} .
		\end{align*}
	We can then integrate to get
		\begin{align*}
		\ln |z| = -2x + \int \frac{5x}{1 - x^2} \, dx = -2x - \frac{5}{2} \ln |1 - x^2| + k_1 .
		\end{align*}
	Taking the exponential and changing the name of the constant, we get
		\begin{align*}
		z = c_1 \frac{e^{-2x}}{(1 - x^2)^{5/2}} .
		\end{align*}
	Since $z = u'$, we get
		\begin{align*}
		u (x) = c_1 \int \frac{e^{-2x}}{(1 - x^2)^{5/2}} \, dx + c_2 .
		\end{align*}
	Therefore, we obtain the general solution as
		\begin{align*}
		y(x) &= \Big( c_1 \int \frac{e^{-2x}}{(1 - x^2)^{5/2}} \, dx + c_2 \Big) y_2 (x) \\
		&= c_1 y_2 (x) \int \frac{e^{-2x}}{(1 - x^2)^{5/2}} \, dx + c_2 y_2 (x) .
		\end{align*}
	From these last calculations, we can conclude that
		\begin{align*}
		1 + 2 \sum_{n = 1}^\infty \frac{(2n)!!}{(2n - 1)!!} x^{2n} = \Big( \frac{12x}{(1 - x^2)^{3/2}} - 5x \Big) \int \frac{e^{2x}}{(1 - x^2)^{5/2}} \, dx .
		\end{align*}
	Isn't beautiful? ;)
	
	\vfill
	
	\hfill \textcolor{red}{\textsc{Total (Points): 50.}}
	
	
	
\end{document}
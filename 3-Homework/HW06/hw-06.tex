\documentclass[12pt]{article}
\usepackage[utf8]{inputenc}

\usepackage{lmodern}

\usepackage{enumitem}
\usepackage[margin=2cm]{geometry}

\usepackage{amsmath, amsfonts, amssymb}
\usepackage{graphicx}
\usepackage{subfigure}
\usepackage{tikz}
\usepackage{pgfplots}
\usepackage{multicol}

\usepackage{comment}
\usepackage{url}
\usepackage{calc}
%\usepackage{subcaption}
\usepackage[indent=0pt]{parskip}

\usepackage{array}
\usepackage{blkarray,booktabs, bigstrut}

\pgfplotsset{compat=1.16}

% MATH commands
\newcommand{\ga}{\left\langle}
\newcommand{\da}{\right\rangle}
\newcommand{\oa}{\left\lbrace}
\newcommand{\fa}{\right\rbrace}
\newcommand{\oc}{\left[}
\newcommand{\fc}{\right]}
\newcommand{\op}{\left(}
\newcommand{\fp}{\right)}

\newcommand{\bi}{\mathbf{i}}
\newcommand{\bj}{\mathbf{j}}
\newcommand{\bk}{\mathbf{k}}
\newcommand{\bF}{\mathbf{F}}

\newcommand{\mR}{\mathbb{R}}

\newcommand{\ra}{\rightarrow}
\newcommand{\Ra}{\Rightarrow}

\newcommand{\sech}{\mathrm{sech}\,}
\newcommand{\csch}{\mathrm{csch}\,}
\newcommand{\curl}{\mathrm{curl}\,}
\newcommand{\dive}{\mathrm{div}\,}

\newcommand{\ve}{\varepsilon}
\newcommand{\spc}{\vspace*{0.5cm}}

\DeclareMathOperator{\Ran}{Ran}
\DeclareMathOperator{\Dom}{Dom}

\newcommand{\exo}[3]{\noindent\textcolor{red}{\fbox{\textbf{Section {#1} | Problem {#2} | {#3} points}}}\\}

\begin{document}
	\noindent \hrulefill \\
	MATH-302 \hfill Pierre-Olivier Paris{\'e}\\
	Homework 6 Solutions \hfill Fall 2022\\\vspace*{-1cm}
	
	\noindent\hrulefill
	
	\spc
	
	\exo{5.3}{1}{10}
	\\
	\underline{\textbf{Find the general solutions to the complementary equation.}}\\
	The complementary equation is
		\begin{align*}
		y'' + 5y' - 6y = 0 .
		\end{align*}
	The characteristic equation is $r^2 + 5r - 6 = 0$. The roots are $r = 1$ and $r= 5$. So the general solution of the complementary equation is
		\begin{align*}
		y_h (x) = c_1 e^{x} + c_2e^{5x} .
		\end{align*}
	
	\underline{\textbf{Find a particular solution.}}\\
	We have a degree 2 polynomial on the right-hand side of the ODE. We therefore suggest
		\begin{align*}
		y_{par}(x) = Ax^2 + Bx + C .
		\end{align*}
	We have $y' = 2Ax + B$ and $y'' = 2A$. Therefore, after pluging in the ODE:
		\begin{align*}
		2A + 5(2Ax + B) + Ax^2 + Bx + C = 22 + 18x - 18x^2 .
		\end{align*}
	Gathering similar terms, we obtain the equation
		\begin{align*}
		2A + 5B + C + (10A + B)x + Ax^2 = 22 + 18x - 18x^2 .
		\end{align*}
	We therefore get $A = -18$, $10A + B = 18$, and $2A + 5B + C = -18$. So 
		\begin{align*}
		B = 18 + 180 = 198 .
		\end{align*}
	Finally, $C = 36 -990 = -954$. Therefore, the particular solution we were seeking for is
		\begin{align*}
		y_{par} (x) = -18x^2 + 198 x - 954 .
		\end{align*}
		
	\underline{\textbf{General solution.}}\\
	Combining $y_h$ and $y_{par}$, we get
		\begin{align*}
		y (x) = y_h (x) + y_{par} (x) = c_1 e^x + c_2 e^{5x} -18x^2 + 198x - 954 .
		\end{align*}
		
	\newpage
		
	\exo{5.3}{3}{10}
	\\
	\underline{\textbf{Find the general solution to the complementary equation.}}\\
	The complementary equation is 
		\begin{align*}
		y'' + 8y' + 7y = 0 .
		\end{align*}
	The characteristic equation is $r^2 + 8r + 7 = 0$. The roots are $r = -1$ and $r = -7$. Therefore, the general solution to the complementary equation is
		\begin{align*}
		y_h (x) = c_1 e^{-x} + c_2 e^{-7x} .
		\end{align*}
		
	\underline{\textbf{Find a particular solution.}}\\
	We have a degree three polynomial on the right-hand side of the polynomial. We therefore suggest
		\begin{align*}
		y_{par} (x) = Ax^3 + Bx^2 + Cx + D .
		\end{align*}
	We have $y' (x) = 3Ax^2 + 2Bx + C$ and $y'' (x) = 6Ax + 2B$. After plugging in the ODE, we get
		\begin{align*}
		6Ax + 2B + 8(3Ax^2 + 2Bx + C) + 7(Ax^3 + Bx^2 + Cx + D) = -8 - x + 24x^2 + 7x^3 .
		\end{align*}
	Collecting the terms with the same power of $x$, we get
		\begin{align*}
		2B + 8C + 7D + (6A + 16B + 7C)x + (24A + 7B)x^2 + 24Ax^3 = -8 - x + 24x^2 + 7x^3 .
		\end{align*}
	Therefore, we see that $24A = 7$, $24A + 7B = 24$, $6A + 16B + 7C = -1$, and $2B + 8C + 7D = -8$. After the dominoes effect, we find that
		\begin{align*}
		A = 7/24 , \, B = 17/7, \, C = -1165/196, \, D = 1700/343 .
		\end{align*}
	A particular solution is
		\begin{align*}
		y_{par} (x) = (7/24)x^3 + (17/7)x^2 - (1165/196)x + (1700/343) .
		\end{align*}
		
	\underline{\textbf{General solution.}}\\
	The general solution is therefore
		\begin{align*}
		y(x)= y_h (x) + y_{par} (x) = c_1 e^{-x} + c_2 e^{-7x} + (7/24)x^3 + (17/7)x^2 - (1165/196)x + (1700/343) .
		\end{align*}
		
	\newpage
	
	\exo{5.3}{7}{5}
	\\
	Suppose that we could find a particular solution of the form $y_{par} (x) = A + Bx + Cx^2$. Replacing in the ODE $y'$ and $y''$, we find
		\begin{align*}
		2C + B + 2Cx = 1 + 2x + x^2 \iff (2C + B) + (2C)x + 0x^2 = 1 +2x + x^2 .
		\end{align*}
	But, $0$ in front of the $x^2$ on the left-hand side can't be equal to the $1$ in front of the $x^2$ on the right-hand side. Therefore, the particular solution can't be of the form $A + Bx + Cx^2$.
	
	\newpage
	
	\exo{5.3}{15}{5}
	\\
	Suppose that $y_{par} (x) = Ax^{\alpha}$, where $A$ is a non-zero constant. To be a solution, the function $y_{par}$ should satisfy the ODE. We have
		\begin{align*}
		y' = A\alpha x^{\alpha - 1} \quad \text{ and } \quad y'' = A\alpha (\alpha - 1) x^{\alpha - 2} .
		\end{align*}
	Substituting in the ODE, we get
		\begin{align*}
		ax^2 (A \alpha (\alpha - 1)x^{\alpha - 2}) + bx (A\alpha x^{\alpha - 1}) + c Ax^\alpha = M x^\alpha .
		\end{align*}
	After simplifying, we obtain
		\begin{align*}
		aA \alpha (\alpha - 1) x^{\alpha} + b A \alpha x^{\alpha} + cA x^\alpha = M x^\alpha .
		\end{align*}
	Dividing through $Ax^{\alpha}$, we get
		\begin{align*}
		a\alpha (\alpha - 1)  + b \alpha + c = M/A .
		\end{align*}
	Since $M/A \neq 0$, then $a \alpha (\alpha - 1) + b\alpha + c$ can't be zero. This was the claim made.
	
	In the other direction, if $a \alpha (\alpha - 1) + b \alpha + c \neq 0$, then there is some constant $M\neq 0$ such that
		\begin{align*}
		a \alpha (\alpha - 1) + b \alpha + c = M .
		\end{align*}
	Multiplying by $x^{\alpha}$, we obtain
		\begin{align*}
		a \alpha (\alpha - 1) x^{\alpha} + b \alpha x^{\alpha} + c x^{\alpha} = M x^\alpha
		\end{align*}
	which can be rewritten as
		\begin{align*}
		a x^2 \alpha (\alpha - 1) x^{\alpha - 2} + b x \alpha x^{\alpha} + c x^{\alpha} = M x^{\alpha} .
		\end{align*}
	Letting $y (x) = x^{\alpha}$, we therefore see that
		\begin{align*}
		ax^2 y'' + bx x' + c y = Mx^{\alpha} .
		\end{align*}
	Therefore, $y = x^{\alpha}$ is a particular solution (here, with $A = 1$). 
	
	\newpage
	
	\exo{5.4}{15}{10}
	\\
	\underline{\textbf{Find the general solution to the complementary equation.}}\\
	The complementary equation is
		\begin{align*}
		y'' - 3y' + 2y .
		\end{align*}
	The characteristic equation is $r^2 - 3r + 2 = 0$. Therefore, the roots are $r = 1$ and $r = 2$. So, the general solution to the complementary equation is
		\begin{align*}
		y_h (x) = c_1 e^{x} + c_2 e^{2x} .
		\end{align*}
		
	\underline{\textbf{Find a particular solution.}}\\
	The right-hand side if of the form exponential times a polynomial. Also, one of the root does not appear in the exponential. We therefore suggest
		\begin{align*}
		y_{par} (x) = A e^{3x} + Bx e^{3x} .
		\end{align*}
	We have
		\begin{align*}
		y' = 3Ae^{3x} + Be^{3x} + 3Bxe^{3x} \quad \text{ and } \quad y'' = 9Ae^{3x} + 3Be^{3x} + 9Bxe^{3x} + 3Be^{3x} .
		\end{align*}
	Replacing this in the ODE, we find
		\begin{align*}
		9A e^{3x} + 3Be^{3x} + 9Bxe^{3x} + 3B e^{3x} - 3(3A e^{3x} + Be^{3x} + 3Bxe^{3x}) + 2(Ae^{3x} + Bxe^{3x}) = e^{3x} + xe^{3x} .
		\end{align*}
	Collecting similar terms together, we get
		\begin{align*}
		(2A + 3B)e^{3x} + (2B)xe^{3x} = e^{3x} + xe^{3x}
		\end{align*}
	We should have the same number of $e^{3x}$ and $xe^{3x}$ on both sides. Therefore, we find that
		\begin{align*}
		2B = 1 \quad \text{ and } 2A + 3B = 1 .
		\end{align*}
	We find that $B = 1/2$ and $A = -1/4$. The particular solution is therefore
		\begin{align*}
		y_{par} (x) = -\frac{e^{3x}}{4} + \frac{xe^{3x}}{2} .
		\end{align*}
	
	\underline{\textbf{General solution.}}\\
	The general solution is therefore
		\begin{align*}
		y (x) = y_{h} (x) + y_{par} (x) = c_1 e^x + c_2e^{2x} - \frac{e^{3x}}{4} + \frac{xe^{3x}}{2} .
		\end{align*}
	
	\newpage
	
	\exo{5.4}{19}{10}
	\\
	\underline{\textbf{Find the general solution to the complementary equation.}}\\
	The characteristic equation is $r^2 - 2r + 1 = 0$. There is only one root, $r = 1$. Therefore, the solution is
		\begin{align*}
		y_h (x) = c_1 e^{x} + c_2 x e^{x} .
		\end{align*}
		
	\underline{\textbf{Find a particular solution.}}\\
	We have that both $e^{x}$ and $xe^x$ are solutions to the complementary equation. Therefore, based on the lecture notes, we suggest
		\begin{align*}
		y_{par} (x) = x^2 e^x (Ax + B) = Ax^3 e^x + Bx^2 e^x .
		\end{align*}
	We have 
		\begin{align*}
		y' &= 3Ax^2 e^x + Ax^3 e^x + 2Bxe^x + Bx^2 e^x \\ 
		y'' &= Ax^3 e^x + (6A + B)x^2 e^x + (6A + 4B) xe^x + 2B e^x
		\end{align*}
	We plug this in the ODE:
		\begin{align*}
		Ax^3 e^x + (6A + B)x^2 e^x + (6A + 4B) xe^x + 2B e^x - 2(3Ax^2 e^x + Ax^3 e^x + 2Bxe^x + Bx^2 e^x)\\
		+ Ax^3 e^x + Bx^2 e^x = 2e^x - 12xe^x .
		\end{align*}
	Collecting similar terms, we obtain
		\begin{align*}
		6A xe^x + 2B e^x = 2e^x - 12xe^x .
		\end{align*}
	We must have $6A = 2$ and $2B = -12$. Therefore, we conclude that $A = 1/3$ and $B = -6$. A particular solution to the ODE is
		\begin{align*}
		y_{par} (x) = \frac{1}{3} x^3 e^x -6 x^2 e^x .
		\end{align*}
	
	\underline{\textbf{General solution.}}\\
	The general solution is
		\begin{align*}
		y (x) = y_h (x) + y_{par} (x) = c_1 e^x + c_2 xe^x + \frac{x^3 e^x}{3} - 6 x^2 e^x .
		\end{align*}
	
	\vfill
	
	\hfill \textcolor{red}{\textsc{Total (Points): 50.}}
	
	
	
\end{document}
\documentclass[12pt]{article}
\usepackage[utf8]{inputenc}

\usepackage{lmodern}

\usepackage{enumitem}
\usepackage[margin=2cm]{geometry}

\usepackage{amsmath, amsfonts, amssymb}
\usepackage{graphicx}
\usepackage{subfigure}
\usepackage{tikz}
\usepackage{pgfplots}
\usepackage{multicol}

\usepackage{comment}
\usepackage{url}
\usepackage{calc}
%\usepackage{subcaption}
\usepackage[indent=0pt]{parskip}

\usepackage{array}
\usepackage{blkarray,booktabs, bigstrut}
\usepackage{bigints}

\pgfplotsset{compat=1.16}

% MATH commands
\newcommand{\ga}{\left\langle}
\newcommand{\da}{\right\rangle}
\newcommand{\oa}{\left\lbrace}
\newcommand{\fa}{\right\rbrace}
\newcommand{\oc}{\left[}
\newcommand{\fc}{\right]}
\newcommand{\op}{\left(}
\newcommand{\fp}{\right)}

\newcommand{\bi}{\mathbf{i}}
\newcommand{\bj}{\mathbf{j}}
\newcommand{\bk}{\mathbf{k}}
\newcommand{\bF}{\mathbf{F}}

\newcommand{\mR}{\mathbb{R}}

\newcommand{\ra}{\rightarrow}
\newcommand{\Ra}{\Rightarrow}

\newcommand{\sech}{\mathrm{sech}\,}
\newcommand{\csch}{\mathrm{csch}\,}
\newcommand{\curl}{\mathrm{curl}\,}
\newcommand{\dive}{\mathrm{div}\,}

\newcommand{\ve}{\varepsilon}
\newcommand{\spc}{\vspace*{0.5cm}}

\DeclareMathOperator{\Ran}{Ran}
\DeclareMathOperator{\Dom}{Dom}

\newcommand{\exo}[3]{\noindent\textcolor{red}{\fbox{\textbf{Section {#1} | Problem {#2} | {#3} points}}}\\}
\newcommand{\qu}[4]{\noindent\textcolor{#4}{\fbox{\textbf{Section {#1} | Problem {#2}}} \hrulefill{{\fbox{\textbf{{#3} Points}}}}\\}}

\begin{document}
	\noindent \hrulefill \\
	MATH-302 \hfill Pierre-Olivier Paris{\'e}\\
	Homework 11 Problems \hfill Fall 2022\\\vspace*{-1cm}
	
	\noindent\hrulefill
	
	\spc
		
	\qu{7.2}{A}{15}{blue}
	\\
	Solve the following IVP using power series:
		\begin{align*}
		(x^2 - 4)y'' - xy' - 3y = 0 , \quad y(0) = -1 , \, y' (0) = 2 .
		\end{align*}
		
	%\textcolor{blue}{\hrulefill}
	
	\spc
	
	\qu{7.2}{B}{10}{blue}
	\\
	Solve the following IVP using power series:
		\begin{align*}
		y'' + (x - 3)y' + 3y = 0, \quad y(3) = -2, \, y' (3) = 3 ,
		\end{align*}
	given that the general solution to
		\begin{align*}
		y''(t) + t y' + 3y(t) = 0
		\end{align*}
	is 
		\begin{align*}
		y(t) = a_0 + a_1 t - \frac{3}{2} a_0 t^2 - \frac{2}{3} a_1 t^3 + \frac{5}{8} a_0 t^4 + \frac{3}{10} a_1 t^5 - \frac{7}{48} a_0 t^6 - \frac{2}{35} a_1 t^7 + \cdots .
		\end{align*}
		
	%\textcolor{blue}{\hrulefill}
		
	\spc
		
	\qu{8.1}{C}{25}{blue}
	\\
	Find the Laplace transform of the following functions (you can use the table)
	\begin{multicols}{2}
		\begin{enumerate}[label=\textcolor{blue}{\arabic*)}]
		\item $\cosh (t) \sin (t)$.
		\item $\cosh^2 (t)$.
		\item $t \sinh (2t)$.
		\item $\sin (2t) + \cos (4t)$.
		\item $\sin (2t) \cos (3t )$.
		\end{enumerate}
	\end{multicols}
		
	\textcolor{blue}{\hrulefill}
	
	\hfill \textcolor{blue}{\textsc{Total (Points): 50.}}
	
	
\begin{comment}

	\newpage
	
	\begin{center}
	\large
	\textcolor{red}{\textbf{Complete Solutions}}
	\end{center}
	
	\noindent\textcolor{red}{\hrulefill}
	
	\spc
	
	\qu{7.2}{A}{15}{red}
	\\
	The initial condition are given at $x =0$ and $x = 0$ is not a singular point of the ODE. So, we let $y(x) = \sum_{n = 0}^\infty a_n x^n$ and therefore
		\begin{align*}
		(x^2- 4) y'' &= \sum_{n = 2}^\infty n (n - 1) a_n x^n - \sum_{n = 2}^\infty n (n - 1) 4a_n x^{n - 2} \\
		x y' &= \sum_{n = 1}^\infty n a_n x^n \\
		3y &= \sum_{n = 0}^\infty 3a_n x^n .
		\end{align*}
	The left-hand side then becomes
		\begin{align*}
		LHS &= \sum_{n = 2}^\infty n (n - 1) a_n x^n - \sum_{n = 0}^\infty (n + 2) (n + 1) 4 a_{n + 2} x^n - \sum_{n = 1}^\infty n a_n x^n - \sum_{n = 0}^\infty 3 a_n x^n \\
		&= (-8 a_2 - 3a_0) + (-24 a_3 - 4a_1)x + \sum_{n = 2}^\infty \big[ n (n - 1) a_n - (n + 3) a_n - (n + 2) (n + 1) 4 a_{n + 2} \big] x^n  \\
		&= (-8a_2 - 3a_0) + (-24 a_3 - 4a_1) x + \sum_{n = 2}^\infty \big[ (n^2 - 2n - 3) a_n - 4(n^2 + 3n + 2)a_{n + 2} \big] x^n
		\end{align*}
	Setting $LHS = 0$, we obtain the following relations:
		\begin{align*}
		a_2 = -\frac{3}{8} a_0, \quad a_3 = -\frac{1}{6}a_1 \quad \text{ and } \quad a_{n+2} = \frac{(n-3)(n+1)}{4 (n + 2)(n + 1)} a_n = \frac{n-3}{4(n + 2)} a_n.
		\end{align*}
	Let's try to find an explicit expression for $a_n$.
		\begin{itemize}
		\item $a_4 = a_{2 + 2} = \frac{2-3}{4(2 + 2)} a_2 = \frac{-1}{16} \frac{-3}{8} a_0 = \frac{3}{128} a_0 = \frac{3}{2^7}a_0$.
		\item $a_5 = a_{3 + 2} = \frac{3-3}{4 (5)} a_3 = 0$.
		\item $a_6 = a_{4 + 2} = \frac{4 - 3}{4 (4 + 2)} a_4 = \frac{1}{24} \frac{3}{128} a_0 = \frac{1}{1024} a_0 = \frac{1}{2^{10}} a_0$.
		\item $a_7 = a_{5 + 2} = \frac{5 - 3}{4 (5 + 2)} a_5 = 0$.
		\item $a_8 = a_{6 + 2} = \frac{6 - 3}{4 (6 + 2)} a_6 = \frac{3}{32} \frac{1}{1024} a_0 = \frac{3}{2^{15}} a_0$.
		\item $a_9 = 0$.
		\item $a_{10} = a_{8 + 2} = \frac{8 - 3}{4 (8 + 2)} a_8 = \frac{5}{4 \cdot 10} \frac{3}{2^{15}} a_0 = \frac{3}{2^{18}} a_0$. 
		\item $a_{11} = 0$.
		\item $a_{12} = a_{10 + 2} = \frac{10 - 3}{4 (10 + 2)} a_{10} = \frac{7}{48} \frac{3}{2^{18}} a_0 = \frac{7}{2^{22}} a_0$.
		\end{itemize}
	There is no easy pattern. Therefore, we will only give a finite number of terms of the power series solution:
		\begin{align*}
		y(x) = a_0 + a_1 x - \frac{3}{8} a_0 x^2 - \frac{1}{6} a_1 x^3 + \frac{3}{2^7} a_0 x^4 + \frac{1}{2^{10}} a_0 x^6 + \frac{3}{2^{15}} a_0 x^8 + \frac{3}{2^{18}} a_0 x^{10} + \frac{7}{2^{22}} a_0 x^{12} + \cdots .
		\end{align*}
		
	We know that $y(0) = -1$. So $a_0 = -1$. Also, we know that $y'(0) = 2$, so $a_1 = 2$. Replacing this in the general solution, we obtain
		\begin{align*}
		y(x) = -1 + 2x + \frac{3}{8} x^2 - \frac{1}{3} x^3 - \frac{3}{2^7} x^4 - \frac{1}{2^{10}} x^6 - \frac{3}{2^{15}} x^8 - \frac{3}{2^{18}} x^{10} - \frac{7}{2^{22}} x^{12} - \cdots .
		\end{align*}
		
	\underline{Remark:} Notice that there is a finite number of powers of $x$ multiplying the arbitrary constant $a_1$. In other words, the function $a_1 (x - \frac{x^3}{6})$ is a solution to the ODE (you can check this). So we can use the method of variation of parameters to find a second solution.
	
	Let $y(x) = u(x) y_1 (x)$. Then, we have
		\begin{align*}
		y' (x) &= u' y_1 + u y_1' \\
		y'' (x) &= u'' y_1 + 2 u' y_1 + u y_1'' 
		\end{align*}
	and replacing this in the ODE:
		\begin{align*}
		(x^2 - 4) (u'' y_1 + 2u' y_1 + uy_1'') - x (u'y_1 + u y_1') - 3uy_1 
		= (x^2 - 4) (u'' y_1 + 2u' y_1) - x u' y_1
		\end{align*}
	and therefore
		\begin{align*}
		(x^2 - 4)y_1 u'' + \big( 2 (x^2 - 4)y_1 - x y_1 \big) u' = 0 .
		\end{align*}
	After dividing by $(x^2 - 4) y_1$, we get
		\begin{align*}
		u'' + \Big( 2 - \frac{x}{x^2 - 4} \Big) u' = 0 .
		\end{align*}
	Letting $z = u'$, we have to solve the following first order ODE:
		\begin{align*}
		z' + \Big( 2 - \frac{x}{x^2 - 4} \Big) z = 0 .
		\end{align*}
	The solution is given by
		\begin{align*}
		z (x) = a_0 e^{-2x} \sqrt{|x^2 - 4|} .
		\end{align*}
	Therefore, integrating leads to
		\begin{align*}
		u(x) = a_0 \int e^{-2x} \sqrt{|x^2 - 4|} \, dx + a_1
		\end{align*}
	Finally, replacing the expression of $u(x)$ in $y(x)$, we get
		\begin{align*}
		y(x) = a_0 \big( x - \frac{x^3}{6} \big) \int e^{-2x} \sqrt{|x^2 - 4|} \, dx + a_1 \big( x - \frac{x^3}{6} \big) .
		\end{align*}
		
	\newpage
	
	\qu{7.2}{B}{10}{red}
	\\
	We translate the problem, so that the initial condition are given at $0$. We let $t= x - 3$ and set $y(t) = y(x - 3)$. Therefore, we see that
		\begin{align*}
		y'' (t) = y'' , \, y' (t) = y' .
		\end{align*}
	Therefore, the ODE becomes
		\begin{align*}
		y''(t) + t y' (t) + 3 y(t) = 0 .
		\end{align*}
	From the assumptions of the problem, we have that
		\begin{align*}
		y(t) = a_0 + a_1 t - \frac{3}{2} a_0 t^2 - \frac{2}{3} a_1 t^3 + \frac{5}{8} a_0 t^4 + \frac{3}{10} a_1 t^5 - \frac{7}{48} a_0 t^6 - \frac{2}{35} a_1 t^7 + \cdots .
		\end{align*}
	and changing $t$ for $x - 3$ gives
		\begin{align*}
		y(x) &= a_0 + a_1 (x - 3) - \frac{3}{2} a_0 (x - 3)^2 - \frac{2}{3} a_1 (x - 3)^3 + \frac{5}{8} a_0 (x - 3)^4 + \frac{3}{10} a_1 (x - 3)^5 \\
		& \qquad - \frac{7}{48} a_0 (x - 3)^6 - \frac{2}{35} a_1 (x - 3)^7 + \cdots .
		\end{align*}
	From there, we see that $y(3) = a_0$ and $y'(3) = a_1$. From the initial conditions, we get $a_0 = -2$ and $a_1 = 3$. The solution to the IVP is therefore
		\begin{align*}
		y(x) &= -2 + 3 (x - 3) + 3 (x - 3)^2 - 2 (x - 3)^3 - \frac{5}{4} (x - 3)^4 + \frac{9}{10} (x - 5)^5 + \frac{7}{24} (x - 3)^6 \\
		& \qquad \qquad - \frac{6}{35} (x - 3)^7 + \cdots .
		\end{align*}
	
	\newpage
	
	\qu{8.1}{C}{25}{red}
	
	\begin{enumerate}[label=\textcolor{red}{\arabic*)}]
	\item We have $\cosh(t) = \frac{e^t + e^{-t}}{2}$ and therefore
		\begin{align*}
		\cosh (t) \sin (t) = \frac{1}{2} e^t \sin (t) + \frac{1}{2} e^{-t} \sin (t) .
		\end{align*}
	Using the linearity, we see that
		\begin{align*}
		L \big( \cosh (t) \sin (t) \big) = \frac{1}{2} L (e^t \sin (t)) + \frac{1}{2} L (e^{-t} \sin (t) ) .
		\end{align*}
	From the table and the property that $L (e^{at} f(t)) = F(s - a)$, we obtain
		\begin{align*}
		L (e^t \sin t) = \frac{1}{(s - 1)^2 + 1} \quad \text{ and } \quad L (e^{-t} \sin (t)) = \frac{1}{(s + 1)^2 + 1} .
		\end{align*}
	Therefore, we obtain
		\begin{align*}
		L \big( \cosh (t) \sin (t)\big) =  \frac{1/2}{ (s - 1)^2 + 1} + \frac{1/2}{(s + 1)^2 + 1} .
		\end{align*}
	\item We have
		\begin{align*}
		\cosh^2 (t) = \big( \frac{e^t + e^{-t}}{2} \big)^2 = \frac{e^{2t} + 2 + e^{-2t}}{4} .
		\end{align*}
	Using the linearity, we obtain
		\begin{align*}
		L \big( \cosh^2 (t) \big) = \frac{1}{4} L (e^{2t}) + \frac{1}{2} L (1) + \frac{1}{4} L (e^{-2t}) .
		\end{align*}
	From the table, we have
		\begin{align*}
		L (e^{2t}) = \frac{1}{s- 2} , \quad L (1) = \frac{1}{s} \quad \text{ and } \quad L (e^{-2t}) = \frac{1}{s + 2} .
		\end{align*}
	Therefore, we get
		\begin{align*}
		L \big( \cosh^2 (t) \big) = \frac{1}{4 (s - 2)} + \frac{1}{2s} + \frac{1}{4 (s + 1)} .
		\end{align*}
	\item We have that
		\begin{align*}
		L (\sinh (2t)) = \frac{2}{s^2 + 4} .
		\end{align*}
	From the fact that $L (t f(t)) = -\frac{d}{ds} F(s)$, we get that
		\begin{align*}
		L \big( t \sinh (2t) \big) = - \frac{d}{ds} \big( \frac{2}{s^2 + 4} \big) = \frac{4s}{(s^2 + 4)^2} .
		\end{align*}
	\item Using the linearity, we have
		\begin{align*}
		L \big( \sin (2t) + \cos (4t) \big) = L \big( \sin (2t) \big) + L \big( \cos (4t) \big) .
		\end{align*}
	From the table, we have
		\begin{align*}
		L \big( \sin (2t) \big) = \frac{2}{s^2 + 4} \quad \text{ and } \quad L \big( \cos (4t) \big) = \frac{s}{s^2 + 16} .
		\end{align*}
	Therefore, we obtain
		\begin{align*}
		L \big( \sin (2t) + \cos (4t) \big) = \frac{2}{s^2 + 4} + \frac{s}{s^2 + 16} .
		\end{align*}
	\item Using a trigonometric identity, we have
		\begin{align*}
		\sin (2t) \cos (3t ) = \frac{1}{2} \sin (-t) + \frac{1}{2} \sin (5t) = \frac{1}{2} \sin (5t) - \frac{1}{2} \sin (t) .
		\end{align*}
	Using the linearity, we see that
		\begin{align*}
		L \big( \sin (2t) \cos (3t ) \big) = \frac{1}{2} L \big( \sin (5t) \big) - \frac{1}{2} L \big( \sin (t) \big) .
		\end{align*}
	Using the table, we obtain
		\begin{align*}
		L \big( \sin (2t) \cos (3t ) \big) = \frac{5}{2 (s^2 + 25)} - \frac{1}{2(s^2 + 1)} .
		\end{align*}
	\end{enumerate}
	
	\vfill
	
	\hfill \textcolor{blue}{\textsc{Total (Points): 50.}}
	
	\end{comment}
	
\end{document}
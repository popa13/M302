\documentclass[12pt]{article}
\usepackage[utf8]{inputenc}

\usepackage{lmodern}

\usepackage{enumitem}
\usepackage[margin=2cm]{geometry}

\usepackage{amsmath, amsfonts, amssymb}
\usepackage{graphicx}
\usepackage{subfigure}
\usepackage{tikz}
\usepackage{pgfplots}
\usepackage{multicol}

\usepackage{comment}
\usepackage{url}
\usepackage{calc}
%\usepackage{subcaption}
\usepackage[indent=0pt]{parskip}

\usepackage{array}
\usepackage{blkarray,booktabs, bigstrut}
\usepackage{bigints}

\pgfplotsset{compat=1.16}

% MATH commands
\newcommand{\ga}{\left\langle}
\newcommand{\da}{\right\rangle}
\newcommand{\oa}{\left\lbrace}
\newcommand{\fa}{\right\rbrace}
\newcommand{\oc}{\left[}
\newcommand{\fc}{\right]}
\newcommand{\op}{\left(}
\newcommand{\fp}{\right)}

\newcommand{\bi}{\mathbf{i}}
\newcommand{\bj}{\mathbf{j}}
\newcommand{\bk}{\mathbf{k}}
\newcommand{\bF}{\mathbf{F}}

\newcommand{\mR}{\mathbb{R}}

\newcommand{\ra}{\rightarrow}
\newcommand{\Ra}{\Rightarrow}

\newcommand{\sech}{\mathrm{sech}\,}
\newcommand{\csch}{\mathrm{csch}\,}
\newcommand{\curl}{\mathrm{curl}\,}
\newcommand{\dive}{\mathrm{div}\,}

\newcommand{\ve}{\varepsilon}
\newcommand{\spc}{\vspace*{0.5cm}}

\DeclareMathOperator{\Ran}{Ran}
\DeclareMathOperator{\Dom}{Dom}

\newcommand{\exo}[3]{\noindent\textcolor{red}{\fbox{\textbf{Section {#1} | Problem {#2} | {#3} points}}}\\}
\newcommand{\qu}[4]{\noindent\textcolor{#4}{\fbox{\textbf{Section {#1} | Problem {#2}}} \hrulefill{{\fbox{\textbf{{#3} Points}}}}\\}}

\begin{document}
	\noindent \hrulefill \\
	MATH-302 \hfill Pierre-Olivier Paris{\'e}\\
	Homework 13 Problems \hfill Fall 2022\\\vspace*{-1cm}
	
	\noindent\hrulefill
	
	\spc
		
	\qu{8.2}{A}{20}{black}
	\\
	Solve the following IVP using the Laplace transform:
		\begin{align*}
		2y'' - 3y' - 2y = 4e^t , \quad y(0) = 1, \, y' (0) = -2 .
		\end{align*}
		
	%\textcolor{blue}{\hrulefill}
	
	\spc
	
	\qu{8.3}{B}{10}{black}
	\\
	Express the given function $f$ in terms of the unit step functions. 
	\begin{enumerate}[label=\arabic*)]
	\item $\displaystyle f(t) = \left\{ \begin{matrix}
	t & \text{, } 0 \leq t < 1 \\
	1 & \text{, } t \geq 1 .
	\end{matrix} \right.$
	\item $\displaystyle f(t) = \left\{ \begin{matrix}
	t^2 & \text{, } 0 \leq t < 1 \\
	\sin (t) & \text{, } t \geq 1 .
	\end{matrix} \right.$
	\end{enumerate}
	
	\spc
	
	\qu{8.3}{C}{10}{black}
	\\
	Find the Laplace tranform of the given function.
	\begin{enumerate}[label=\arabic*)]
	\item $\displaystyle f(t) = \left\{ \begin{matrix}
	te^t & \text{, } 0 \leq t < 1 \\
	e^t & \text{, } t \geq 1 .
	\end{matrix} \right.$
	\item $\displaystyle f(t) = \left\{ \begin{matrix}
	3 & \text{, } 0 \leq t < 2 \\
	3t + 2 & \text{, } 2 \leq t < 4 \\
	4t & \text{, } t \geq 4 .
	\end{matrix} \right.$
	\end{enumerate}
	
	\spc
	
	\qu{8.3}{D}{10}{black}
	\\
	Find the inverse Laplace transform of the given function.
	\begin{enumerate}[label=\arabic*)]
	\item $\displaystyle H(s) = \frac{e^{-s}}{s^3} + \frac{e^{-2s}}{s^2}$.
	\item $\displaystyle H(s) = \frac{5}{s} - \frac{1}{s^2} + e^{-3s} \big( \frac{6}{s} + \frac{7}{s^2} \big) + \frac{3e^{-6s}}{s^3}$.
	\end{enumerate}
		
	\textcolor{black}{\hrulefill}
	
	\hfill \textcolor{black}{\textsc{Total (Points): 50.}}
	
	
\begin{comment}
	\newpage
	
	\begin{center}
	\large
	\textcolor{red}{\underline{\textbf{Complete Solutions}}}
	\end{center}
	
	%\noindent\textcolor{red}{\hrulefill}
	
	\spc
	
	\qu{8.2}{A}{20}{red}
	\\
	Apply the Laplace transform to the ODE to get
		\begin{align*}
		2 (s^2 Y - s y(0) - y'(0)) - 3 (s Y - y(0)) - 2Y = \frac{4}{s - 1}.
		\end{align*}
	Using the initial condition and collecting the terms, we obtain
		\begin{align*}
		(2s^2 - 3s - 2) Y = \frac{4}{s - 1} + 2s - 7
		\end{align*}
	Since $2s^2 - 3s - 2 = (2s + 1) (s - 2)$, we obtain
		\begin{align*}
		Y(s) = \frac{4}{(2s + 1)( s- 2)(s - 1)} + \frac{2s - 7}{(2s + 1)(s - 2)} .
		\end{align*}
	We can rewrite each term in the RHS using the partial fraction decomposition:
		\begin{align*}
		\frac{4}{(2s + 1)( s- 2)(s - 1)} = -\frac{4/3}{s-1} + \frac{16/15}{2s + 1} + \frac{4/5}{s - 2}
		\end{align*}
	and
		\begin{align*}
		\frac{2s - 7}{(2s + 1)(s - 2)} = \frac{16/5}{2s + 1} - \frac{3/5}{s - 2} .
		\end{align*}
	Therefore, the expression of $Y(s)$ becomes
		\begin{align*}
		Y (s) = -\frac{4/3}{s-1} + \frac{64/15}{2s + 1} + \frac{1/5}{s - 2} = -\frac{4/3}{s-1} + \frac{32/15}{s + 1/2} + \frac{1/5}{s - 2} .
		\end{align*}
	Taking the inverse Laplace transform, we obtain
		\begin{align*}
		y(t) = -\frac{4}{3} e^t + \frac{32}{15} e^{-t/2} + \frac{1}{5} e^{2t} .
		\end{align*}
		
	\newpage
	
	\qu{8.3}{B}{10}{red}
	\begin{enumerate}[label=\textcolor{red}{\arabic*)}]
	\item To deal with the first part, we use the function $t u(t)$. To deal the second part, we use $u (t - 1) - tu(t - 1)$. The expression $t u(t-1)$ is present because we want to cancel out the first $tu(t)$. Therefore, the expression of the function is
		\begin{align*}
		f(t) = t u(t) + u(t-1) - t u(t-1) = t (u(t) - u(t-1)) + u(t - 1) .
		\end{align*}	
	\item To deal with the first part, we use the function $t^2 u(t)$. To deal with the second part, we use $u(t- 1) \sin (t)$ and to cancel out the term $t^2 u(t)$, we substract $t^2 u(t - 1)$. Therefore, the expression of the function is
		\begin{align*}
		f(t) = t^2 u(t) + \sin (t) u (t - 1) - t^2 u (t - 1) = t^2 (u(t) - u(t-1)) + \sin (t) u (t - 1) .
		\end{align*}
	\end{enumerate}
	
	\newpage
	
	\qu{8.3}{C}{10}{red}
	\begin{enumerate}[label=\textcolor{red}{\arabic*)}]
	\item We rewrite the function with the unit step function:
		\begin{align*}
		f(t) = te^t u(t) + e^t u(t-1) - te^t u(t - 1) .
		\end{align*}
	We will use the first formula:
		\begin{align*}
		L (u(t - a) f(t)) = e^{-sa} L (f(t + a)) .
		\end{align*}
	We first have that
		\begin{align*}
		L (u(t) te^t) = e^{-s (0)} L ((t + 0) e^{t + 0}) = L (te^t) = \frac{1}{(s - 1)^2} .
		\end{align*}
	Secondly, we have
		\begin{align*}
		L (e^t u(t - 1)) = e^{-s} L (e^{t + 1}) .
		\end{align*}
	But $e^{t + 1} = e e^t$ and therefore
		\begin{align*}
		L (e^t u(t - 1)) = e^{1 - s} L (e^t) = \frac{e^{1 - s}}{s - 1} .
		\end{align*}
	Thirdly, we have
		\begin{align*}
		L (t e^t u (t - 1)) = e^{-s} L ((t + 1) e^{t + 1}) .
		\end{align*}
	But $(t +1)e^{t + 1} = e (te^t + e^t)$ and therefore
		\begin{align*}
		L (t e^t u (t - 1)) =  e^{1 - s} \big( L (te^t) + L (e^t) \big) = \frac{e^{1 - s}}{(s - 1)^2} + \frac{e^{1 - s}}{s - 1} .
		\end{align*}
	
	Combining everything, we obtain
		\begin{align*}
		L (f(t)) &= \frac{1}{(s - 1)^2} + \frac{e^{1 - s}}{s - 1} - \frac{e^{1 -s}}{(s - 1)^2} - \frac{e^{1 - s}}{s - 1} \\
		&= \frac{1}{(s - 1)^2} - \frac{e^{1 - s}}{(s - 1)^2} .
		\end{align*}
	\item The function can be rewritten as followed:
		\begin{align*}
		f(t) &= 3 u (t) + (3t + 2) u(t - 2) -3 u (t - 2) + 4t u(t - 4) - (3t + 2) u(t - 4)\\
		&= 3u(t) + (3t - 1) u (t - 2) + (t - 2) u (t - 4) .
		\end{align*}
		We will use the second formula:
		\begin{align*}
		L (u(t - a) f(t- a)) = e^{-sa} F(s) .
		\end{align*}
		To apply this formula, we rewrite the function $f(t)$ in the following way:
		\begin{align*}
		f(t) &= 3u (t) + (3t - 6 + 5) u (t - 2) + (t - 4 + 2) u (t - 4) \\
		&= 3u (t) + 3 (t - 2)u(t-2) + 5 u (t -2) + (t - 4) u (t - 4) + 2 u (t - 4) .
		\end{align*}
	Therefore, we have
		\begin{align*}
		L (f (t)) &= 3L (u(t)) + 3 L ((t - 2) u(t - 2)) + 5 L (u(t-2)) + L ((t - 4) u(t - 4)) + 2 L (u (t - 4)) \\
		&= \frac{3}{s} + 3 e^{-2s} L (t) + \frac{5e^{-2s}}{s} + e^{-4s} L(t) + \frac{2e^{-4s}}{s}  \\
		&= \frac{3}{s} + \frac{3e^{-2s}}{s^2} + \frac{5e^{-2s}}{s} + \frac{e^{-4s}}{s^2} + \frac{2e^{-4s}}{s} .
		\end{align*}
	
	\end{enumerate}
	
	\newpage
	
	\qu{8.3}{D}{10}{red}
	\begin{enumerate}[label=\textcolor{red}{\arabic*)}]
	\item From the formula
		\begin{align*}
		L (u(t - a)f(t-a)) = e^{-sa} F(s), 
		\end{align*}
	we extract the following information from the first term:
		\begin{align*}
		F(s) = \frac{1}{s^3} \quad \text{ and } \quad a = 1
		\end{align*}
	and therefore
		\begin{align*}
		L^{-1} \big( \frac{e^{-s}}{s^3} \big) &= u (t - 1) \big( \frac{1}{2} (t - 1)^2 \big)  \\
		&= \frac{1}{2} (t - 1)^2 u (t - 1) .
		\end{align*}
	From the same formula, we extract the following information from the second term:
		\begin{align*}
		F(s) = \frac{1}{s^2} \quad \text{ and } \quad a = 2
		\end{align*}
	and therefore
		\begin{align*}
		L^{-1} \big( \frac{e^{-2s}}{s^2} \big) = u (t - 2) (t - 2) .
		\end{align*}
	Therefore, the final answer is
		\begin{align*}
		L^{-1} (H(s)) = \frac{1}{2} (t - 1)^2 u (t - 1) + (t - 2) u (t - 2) .
		\end{align*}
	\item We have
		\begin{align*}
		L^{-1} \big( \frac{5}{s} \big) = 5 \quad \text{ and } \quad L^{-1} \big( \frac{1}{s^2} \big) = t .
		\end{align*}
	Using the formula
		\begin{align*}
		L (u (t - a) f(t - a)) = e^{-sa} F(s) ,
		\end{align*}
	we see that
		\begin{align*}
		L^{-1} \big( \frac{e^{-3s}}{s} \big) = u (t - 3)\text{, } \quad L^{-1} \big( \frac{e^{-3s}}{s^2} \big) = (t - 3) u (t - 3)
		\end{align*}
	and
		\begin{align*}
		L^{-1} \big( \frac{e^{-6s}}{s^3} \big) = \frac{1}{2} (t - 6)^2 u (t - 6) .
		\end{align*}
	
	The final answer is then
		\begin{align*}
		L^{-1} (H(s)) &= 5 L^{-1} \big( \frac{1}{s} \big) - L^{-1} \big( \frac{1}{s^2} \big) + 6L^{-1} \big( \frac{e^{-3s}}{s} \big) + 7 L^{-1} \big( \frac{e^{-3s}}{s^2} \big) + 3 L^{-1} \big( \frac{e^{-6x}}{s^3} \big) \\
		&= 5 - t + 6 u (t - 3) + 6 u (t - 3) + 7 (t - 3) u (t - 3) + \frac{3}{2} (t - 6)^2 u ( t- 6) . 
		\end{align*}
	\end{enumerate}
	


	\vfill
	
	\hfill \textcolor{red}{\textsc{Total (Points): 50.}}
	
\end{comment}
	
\end{document}
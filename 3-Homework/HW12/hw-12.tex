\documentclass[12pt]{article}
\usepackage[utf8]{inputenc}

\usepackage{lmodern}

\usepackage{enumitem}
\usepackage[margin=2cm]{geometry}

\usepackage{amsmath, amsfonts, amssymb}
\usepackage{graphicx}
\usepackage{subfigure}
\usepackage{tikz}
\usepackage{pgfplots}
\usepackage{multicol}

\usepackage{comment}
\usepackage{url}
\usepackage{calc}
%\usepackage{subcaption}
\usepackage[indent=0pt]{parskip}

\usepackage{array}
\usepackage{blkarray,booktabs, bigstrut}
\usepackage{bigints}

\pgfplotsset{compat=1.16}

% MATH commands
\newcommand{\ga}{\left\langle}
\newcommand{\da}{\right\rangle}
\newcommand{\oa}{\left\lbrace}
\newcommand{\fa}{\right\rbrace}
\newcommand{\oc}{\left[}
\newcommand{\fc}{\right]}
\newcommand{\op}{\left(}
\newcommand{\fp}{\right)}

\newcommand{\bi}{\mathbf{i}}
\newcommand{\bj}{\mathbf{j}}
\newcommand{\bk}{\mathbf{k}}
\newcommand{\bF}{\mathbf{F}}

\newcommand{\mR}{\mathbb{R}}

\newcommand{\ra}{\rightarrow}
\newcommand{\Ra}{\Rightarrow}

\newcommand{\sech}{\mathrm{sech}\,}
\newcommand{\csch}{\mathrm{csch}\,}
\newcommand{\curl}{\mathrm{curl}\,}
\newcommand{\dive}{\mathrm{div}\,}

\newcommand{\ve}{\varepsilon}
\newcommand{\spc}{\vspace*{0.5cm}}

\DeclareMathOperator{\Ran}{Ran}
\DeclareMathOperator{\Dom}{Dom}

\newcommand{\exo}[3]{\noindent\textcolor{red}{\fbox{\textbf{Section {#1} | Problem {#2} | {#3} points}}}\\}
\newcommand{\qu}[4]{\noindent\textcolor{#4}{\fbox{\textbf{Section {#1} | Problem {#2}}} \hrulefill{{\fbox{\textbf{{#3} Points}}}}\\}}

\begin{document}
	\noindent \hrulefill \\
	MATH-302 \hfill Pierre-Olivier Paris{\'e}\\
	Homework 12 Problems \hfill Fall 2022\\\vspace*{-1cm}
	
	\noindent\hrulefill
	
	\spc
		
	\qu{8.2}{A}{25}{black}
	\\
	Solve the following IVP using the Laplace transform:
		\begin{align*}
		y'' + 3y' + 2y = 2e^t , \quad y(0) = 0, \, y'(0) = -1 .
		\end{align*}
		
	%\textcolor{blue}{\hrulefill}
	
	\spc
	
	\qu{8.2}{B}{25}{black}
	\\
	Solve the following IVP using the Laplace transform:
		\begin{align*}
		y'' - 2y' + y =  t^2 e^t, \quad y(0) = 1, \, y'(0) = 2 .
		\end{align*}
		
	\textcolor{black}{\hrulefill}
	
	\hfill \textcolor{black}{\textsc{Total (Points): 50.}}
	
	
\begin{comment}
	\newpage
	
	\begin{center}
	\large
	\textcolor{red}{\underline{\textbf{Complete Solutions}}}
	\end{center}
	
	%\noindent\textcolor{red}{\hrulefill}
	
	\spc
	
	\qu{8.2}{A}{25}{red}
	\\
	Apply the Laplace transform to the ODE to get
		\begin{align*}
		s^2 Y - sy(0) - y'(0) + 3 (sY - f(0)) + 2Y = \frac{2}{s - 1} .
		\end{align*}
	Using the initial condition and collecting the terms, we obtain
		\begin{align*}
		(s^2 + 3s + 2) Y = \frac{2}{s - 1} - 1
		\end{align*}
	Using the fact that $s^2 + 3s + 2 = (s + 1)(s + 2)$, we obtain
		\begin{align*}
		Y(s) = \frac{2}{(s - 1) (s + 1) (s + 2)} - \frac{1}{(s + 1)(s + 2)}
		\end{align*}
	We can rewrite each term in the RHS using the partial fraction decomposition:
		\begin{align*}
		\frac{2}{(s - 1) (s + 1) (s + 2)} = \frac{1/3}{s - 1} + \frac{-1}{s + 1} + \frac{2/3}{s + 2}
		\end{align*}
	and
		\begin{align*}
		\frac{1}{(s + 1)(s + 2)} = \frac{1}{s + 1} + \frac{-1}{s + 2} .
		\end{align*}
	Therefore, the expression of $Y(s)$ becomes
		\begin{align*}
		Y (s) = \frac{1/3}{s - 1} - \frac{1/3}{s + 2} .
		\end{align*}
	Taking the inverse Laplace transform, we obtain
		\begin{align*}
		y(t) = \frac{1}{3} e^{t} - \frac{1}{3} e^{-2t} .
		\end{align*}
		
	\newpage
	
	\qu{8.2}{B}{25}{red}
	\\
	After applying the Laplace transform, we get
		\begin{align*}
		s^2Y - sy(0) - y'(0) - 2sY + 2y(0) + Y = \frac{2}{(s - 1)^3}
		\end{align*}
	Using the initial condition, we then get
		\begin{align*}
		(s^2 - 2s + 1)Y -s  = \frac{2}{(s - 1)^3}.		
		\end{align*}
	Therefore, we obtain
		\begin{align*}
		(s^2 - 2s + 1)Y = s + \frac{2}{(s - 1)^3} .
		\end{align*}
	Using the fact that $s^2 - 2s + 1 = (s - 1)^2$, we get
		\begin{align*}
		Y = \frac{s}{(s - 1)^2} + \frac{2}{(s - 1)^5}.
		\end{align*}
	We notice that
		\begin{align*}
		\frac{s}{(s - 1)^2} = s F(s) .
		\end{align*}
	The inverse of $F(s)$ is $te^t$ and recall that
		\begin{align*}
		L (f'(t)) = sF - f(0) .
		\end{align*}
	Since at $t = 0$, the function $te^t$ is $0$, we get
		\begin{align*}
		\frac{s}{(s-1)^2} \longrightarrow (te^t)' = e^t + te^t .
		\end{align*}
	Also, since
		\begin{align*}
		L (t^4e^t) = \frac{4!}{(s - 1)^5},
		\end{align*}
	we have
		\begin{align*}
		\frac{1}{(s - 1)^5} \longrightarrow \frac{1}{4!} t^4 e^t .
		\end{align*}
	Taking the inverse Laplace transform, we obtain
		\begin{align*}
		y(t) =  e^t + te^t + \frac{1}{12} t^4 e^t .
		\end{align*}


	\vfill
	
	\hfill \textcolor{red}{\textsc{Total (Points): 50.}}
	
\end{comment}
	
\end{document}
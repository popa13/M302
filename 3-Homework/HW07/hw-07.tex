\documentclass[12pt]{article}
\usepackage[utf8]{inputenc}

\usepackage{lmodern}

\usepackage{enumitem}
\usepackage[margin=2cm]{geometry}

\usepackage{amsmath, amsfonts, amssymb}
\usepackage{graphicx}
\usepackage{subfigure}
\usepackage{tikz}
\usepackage{pgfplots}
\usepackage{multicol}

\usepackage{comment}
\usepackage{url}
\usepackage{calc}
%\usepackage{subcaption}
\usepackage[indent=0pt]{parskip}

\usepackage{array}
\usepackage{blkarray,booktabs, bigstrut}

\pgfplotsset{compat=1.16}

% MATH commands
\newcommand{\ga}{\left\langle}
\newcommand{\da}{\right\rangle}
\newcommand{\oa}{\left\lbrace}
\newcommand{\fa}{\right\rbrace}
\newcommand{\oc}{\left[}
\newcommand{\fc}{\right]}
\newcommand{\op}{\left(}
\newcommand{\fp}{\right)}

\newcommand{\bi}{\mathbf{i}}
\newcommand{\bj}{\mathbf{j}}
\newcommand{\bk}{\mathbf{k}}
\newcommand{\bF}{\mathbf{F}}

\newcommand{\mR}{\mathbb{R}}

\newcommand{\ra}{\rightarrow}
\newcommand{\Ra}{\Rightarrow}

\newcommand{\sech}{\mathrm{sech}\,}
\newcommand{\csch}{\mathrm{csch}\,}
\newcommand{\curl}{\mathrm{curl}\,}
\newcommand{\dive}{\mathrm{div}\,}

\newcommand{\ve}{\varepsilon}
\newcommand{\spc}{\vspace*{0.5cm}}

\DeclareMathOperator{\Ran}{Ran}
\DeclareMathOperator{\Dom}{Dom}

\newcommand{\exo}[3]{\noindent\textcolor{red}{\fbox{\textbf{Section {#1} | Problem {#2} | {#3} points}}}\\}

\begin{document}
	\noindent \hrulefill \\
	MATH-302 \hfill Pierre-Olivier Paris{\'e}\\
	Homework 7 Solutions \hfill Fall 2022\\\vspace*{-1cm}
	
	\noindent\hrulefill
	
	\spc
	
	\exo{5.4}{5}{10}
	\\
	\underline{\textbf{Find the general solutions to the complementary equation.}}\\
	The complementary equation is
		\begin{align*}
		y'' + 4y = 0 .
		\end{align*}
	The characteristic equation associated to the complementary equation is $r^2 + 4 = 0$. Therefore, the roots are $r_1 = 2i$ and $r_2 = -2i$. The general solution is
		\begin{align*}
		y_h (x) = c_1 \cos (2x) + c_2 \sin (2x) .
		\end{align*}
	
	\underline{\textbf{Find a particular solution.}}\\
	We have an exponential times a polynomial of degree two. Also, the number $\alpha = -1$ is not a root of the characteristic polynomial. Therefore, we suggest
		\begin{align*}
		y_p (x) = e^{-x} (Ax^2 + Bx + C) .
		\end{align*}
	The respective derivatives are
		\begin{align*}
		y' (x) &= -e^{-x} (Ax^2 + Bx + C) + e^{-x} (2Ax + B) \\
		y''(x) &= e^{-x} (Ax^2 + Bx + C) - 2e^{-x} (2Ax + B) + e^{-x} (2A).
		\end{align*}
	Pluging in the original ODE, we get
		\begin{align*}
		e^{-x} (Ax^2 + Bx + C) - 2e^{-x} (2Ax + B) + 2Ae^{-x} + 4e^{-x} (Ax^2 + Bx + C) = e^{-x} (5x^2 - 4x + 7) .
		\end{align*}
	Dividing by $e^{-x}$ and collecting similar terms, we get
		\begin{align*}
		(5A)x^2 + (5B-4A)x + (5C - 2B + 2A) = 5x^2 - 4x + 7 .
		\end{align*}
	We see from this equation that $A = 1$. Then, we must have $5B - 4 = -4$ which implies that $B = 0$. Finally, we must also have $5C + 2 = 7$ which implies that $C = 1$. Therefore, we obtain
		\begin{align*}
		y_p (x) = e^{-x} \op x^2 + 1 \fp .
		\end{align*}
		
	\underline{\textbf{General solution.}}\\
	Combining $y_h$ and $y_{par}$, we get
		\begin{align*}
		y (x) = y_h (x) + y_{par} (x) = c_1 \cos (2x) + c_2 \sin (2x) + e^{-x} \op x^2 + 1 \fp .
		\end{align*}
		
	\newpage
		
	\exo{5.4}{11}{10}
	\\
	\underline{\textbf{Find the general solution to the complementary equation.}}\\
	The complementary equation is
		\begin{align*}
		y'' + 2y' + y = 0 .
		\end{align*}
	The characteristic equation associated to the complementary equation is $r^2 + 2r + 1 = 0$. There is only one root: $r_1 = -1$. The solution to the complementary equation is therefore
		\begin{align*}
		y_h (x) = c_1 e^{-x} + c_2 xe^{-x} .
		\end{align*}
		
	\underline{\textbf{Find a particular solution.}}\\
	We have an exponential multiplying a polynomial of degree $1$. However, the number $\alpha = -1$ is a root of the characteristic polynomial. Moreover, we have $e^{-x}$ and $xe^{-x}$ are solutions to the complementary equation. Therefore, we suggest
		\begin{align*}
		y_{par} (x) = x^2 e^{-x} (Ax + B) = e^{-x} (Ax^3 + Bx^2).
		\end{align*}
	The respective derivatives are
		\begin{align*}
		y' (x) &= -e^{-x} (Ax^3 + Bx^2) + e^{-x} (3Ax^2 + 2Bx) \\
		y''(x) &= e^{-x} (Ax^3 + Bx^2) - 2 e^{-x} (3Ax^2 + 2Bx) + e^{-x} (6Ax + 2B) .
		\end{align*}
	Plugging in the initial ODE, we obtain
		\begin{align*}
		e^{-x} (Ax^3 + Bx^2) - 2 e^{-x} (3Ax^2 + 2Bx) + e^{-x} (6Ax + 2B)  + 2e^{-x} (3Ax^2 + 2Bx - Ax^3 - Bx^2)\\
		 + e^{-x} (Ax^3 + Bx^2) = e^{-x} (3x + 2) .
		\end{align*}
	Dividing by $e^{-x}$ and collecting similar terms, we obtain
		\begin{align*}
		%Ax^3 + Bx^2 - 6Ax^2 - 4Bx + 6Ax + 2B + 6Ax^2 + 4Bx - 2Ax^3 - 2Bx^2 + Ax^3 + Bx^2 
		6Ax + 2B = 3x + 2
		\end{align*}
	We therefore obtain $A = 1/2$ and $B = 1$. So, the particular solution is
		\begin{align*}
		y_{par} (x) = e^{-x} (0.5x^3 + x^2) .
		\end{align*}
		
	\underline{\textbf{General solution.}}\\
	The general solution is therefore
		\begin{align*}
		y(x)= y_h (x) + y_{par} (x) = c_1 e^{-x} + c_2 xe^{-x} + e^{-x} ((1/2)x^3 + x^2) .
		\end{align*}
		
	\newpage
	
	\exo{5.4}{21}{15}
	\\
	
	\underline{\textbf{Complementary Equation.}}\\
	The complementary equation is
		\begin{align*}
		y'' + 3y' - 4y = 0 .
		\end{align*}
	The characteristic polynomial associated to the complementary equation is $r^2 + 3r - 4 = 0$. The roots are $r = 1$ and $r = -4$. Therefore, the solution is
		\begin{align*}
		y_h (x) = c_1 e^{x} + c_2 e^{-4x} .
		\end{align*}
		
	\underline{\textbf{Find a particular solution.}}\\
	We have an exponential times a polynomial of degree two. Also, the number $\alpha = 2$ is a not a root of the characteristic polynomial. We therefore suggest
		\begin{align*}
		y_{par} (x) = e^{2x} (Ax + B ) .
		\end{align*}
	The respective derivatives are
		\begin{align*}
		y'(x) &= 2e^{2x} (Ax + B) + Ae^{2x} \\
		y''(x) &= 4e^{2x} (Ax + B) + 4Ae^{2x} + 2Ae^{2x} .
		\end{align*}
	Plugging in the original ODE, we obtain
		\begin{align*}
		4e^{2x} (Ax + B) + 4Ae^{2x} + 2Ae^{2x} + 6e^{2x} (Ax + B) + 3 A e^{2x} - 4e^{2x} (Ax + B) = e^{2x} (6x + 7) .
		\end{align*}
	Dividing by $e^{2x}$ and collecting similar terms, we obtain
		\begin{align*}
		6Ax + (7A + 6B) = 6x + 7
		\end{align*}
	and therefore $A = 1$ and $B = 0$. The particular solution is therefore
		\begin{align*}
		y_{par} (x) = xe^{2x} .
		\end{align*}
	
	\underline{\textbf{General solution.}}\\
	Combining $y_h$ and $y_{par}$, we obtain
		\begin{align*}
		y(x) = y_h (x) + y_{par} (x) = c_1 e^{x} + c_2e^{-4x} + xe^{2x} .
		\end{align*}
		
	\underline{\textbf{Initial Value Problem.}}\\
	We have $y(0) = 2$, so
		\begin{align*}
		c_1 + c_2 = 2 .
		\end{align*}
	We have $y' (x) = c_1 e^{x} - 4c_2 e^{-4x} + e^{2x} + 2xe^{2x}$ and with $y' (0) = 8$, we obtain
		\begin{align*}
		c_1 - 4c_2 = 7 .
		\end{align*}
	Subtracting the first equation to the second equation, we obtain
		\begin{align*}
		-5c_2 = 5 \quad \Ra \quad c_2 = -1 .
		\end{align*}
	Replacing $c_2$ in the first equation by $-1$, we obtain
		\begin{align*}
		c_1 = 2 + 1 = 3 .
		\end{align*}
	Therefore, the solution to the IVP is
		\begin{align*}
		y(x) = 3e^{x} -  e^{-4x} + xe^{2x} .
		\end{align*}
		
	\newpage
	
	\exo{5.4}{30(a)}{5}
	\\
	Suppose that $y$ is a solution to the constant coefficient equation
		\begin{align*}
		ay'' + by' + cy = e^{\alpha x} G(x) .
		\end{align*}
	Dividing by $e^{\alpha x}$, we obtain
		\begin{align*}
		e^{-\alpha x} (ay'' + by' + cy) = G(x) .
		\end{align*}
	Define $u = e^{-\alpha x} y (x)$, so that $y(x) = ue^{\alpha x}$. The first and second derivatives of $y$ are therefore
		\begin{align*}
		y' &= \alpha e^{\alpha x} u + e^{\alpha x} u' \\
		y'' &= \alpha^2 e^{\alpha x} u + 2\alpha e^{\alpha x} u' + e^{\alpha x} u'' .
		\end{align*}
	Replacing those derivatives in the ODE, we have
		\begin{align*}
		e^{-\alpha x} (ay'' + by' + c ) &= e^{-\alpha x} \op a \alpha^2 e^{\alpha x} u + 2 a \alpha e^{\alpha x} u' + ae^{\alpha x} u'' + b \alpha e^{\alpha x} u + be^{\alpha x} u' + ce^{\alpha x} u \fp \\
		&= a \alpha^2 u + 2a \alpha u' + a u'' + b\alpha u + bu' + cu) \\
		&= a u'' + (2a\alpha + b)u' + (a \alpha^2 + b \alpha + c ) u .
		\end{align*}
	Since the left-hand side is equal to $G(x)$, we conclude that $u$ is a solution to the following ODE:
		\begin{align*}
		a u'' + (2a\alpha + b)u' + (a \alpha^2 + b \alpha + c ) u = G(x) .
		\end{align*}
	Denote by $p(r) = ar^2 + br + c$ the characteristic polynomial. Then we can see that
		\begin{align*}
		2a \alpha + b = p'(\alpha )
		\end{align*}
	and
		\begin{align*}
		a\alpha^2 + b\alpha + c = p(\alpha ) .
		\end{align*}
	Therefore, the function $u$ is a solution to
		\begin{align*}
		au'' + p'(\alpha ) u' + p(\alpha ) u = G(x) .
		\end{align*}
		
	Now suppose that $y (x) = e^{\alpha x} u$ where $u$ is a solution to 
		\begin{align*}
		a u'' + p'(\alpha ) u' + p (\alpha ) u = G(x) .
		\end{align*}
	From the calculation above, we see that
		\begin{align*}
		a u'' + p'(\alpha ) u' + p (\alpha ) u = e^{-\alpha x} (a y'' + by' + c) .
		\end{align*}
	Therefore, we get
		\begin{align*}
		e^{-\alpha x} (ay'' + by' + cy) = G(x)
		\end{align*}
	and so $y$ is a solution to the ODE
		\begin{align*}
		ay'' + by' + cy = e^{\alpha x} G(x) .
		\end{align*}
	\newpage
	
	\exo{5.5}{7}{10}
	\\
	
	\underline{\textbf{Complementary Equation.}}\\
	The complementary equation is
		\begin{align*}
		y'' + 4y = 0 .
		\end{align*}
	The characteristic polynomial associated to the complementary equation is $r^2 + 4$. The roots of this polynomial are $r_1 = 2i$ and $r_2 = -2i$. Therefore, the solution is
		\begin{align*}
		y_h (x) = c_1 \cos (2x) + c_2 \sin (2x) .
		\end{align*}
		
	\underline{\textbf{Find a particular solution.}}\\
	We have a linear combination of $\cos(2x)$ and $\sin (2x)$. However, $\cos (2x)$ and $\sin (2x)$ are in the solutions to the complementary equation. Therefore, we suggest
		\begin{align*}
		y_{par} (x) = x (A\cos (2x) + B \sin (2x)) .
		\end{align*}
	The respective derivatives are
		\begin{align*}
		y' (x) &= A\cos (2x) + B\sin (2x) + x (2B \cos (2x) - 2A \sin (2x)) \\
		y'' (x) &= 4B \cos (2x) - 4A \sin (2x) - x (4A \cos (2x) + 4B \sin (2x)) .
		\end{align*}
	Plugging in the original ODE, we get
		\begin{align*}
		4B \cos (2x) - 4A \sin (2x) - 4A x \cos (2x) - 4B x \sin (2x) + 4Ax \cos (2x) + 4Bx \sin (2x) = -12 \cos x - 4\sin x
		\end{align*}
	which simplifies to
		\begin{align*}
		4B \cos (2x) - 4A \sin (2x) = -12 \cos (2x) - 4\sin (2x) .
		\end{align*}
	Therefore, we obtain $B = -3$ and $A = 1$. The particular expression is
		\begin{align*}
		y_{par} (x) = x \cos (2x) - 3 x \sin (2x) .
		\end{align*}
		
	\underline{\textbf{General solution.}}\\
	The general solution is
		\begin{align*}
		y (x) = y_h (x) + y_{par} (x) = c_1 \cos (2x) + c_2 \sin (2x) + x \cos (2x) - 3x \sin (2x) .
		\end{align*}
	
	\vfill
	
	\hfill \textcolor{red}{\textsc{Total (Points): 50.}}
	
	
	
\end{document}
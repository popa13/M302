\documentclass[12pt]{article}
\usepackage[utf8]{inputenc}

\usepackage{lmodern}

\usepackage{enumitem}
\usepackage[margin=2cm]{geometry}

\usepackage{amsmath, amsfonts, amssymb}
\usepackage{graphicx}
\usepackage{subfigure}
\usepackage{tikz}
\usepackage{pgfplots}
\usepackage{multicol}

\usepackage{comment}
\usepackage{url}
\usepackage{calc}
%\usepackage{subcaption}
\usepackage[indent=0pt]{parskip}

\usepackage{array}
\usepackage{blkarray,booktabs, bigstrut}

\pgfplotsset{compat=1.16}

% MATH commands
\newcommand{\ga}{\left\langle}
\newcommand{\da}{\right\rangle}
\newcommand{\oa}{\left\lbrace}
\newcommand{\fa}{\right\rbrace}
\newcommand{\oc}{\left[}
\newcommand{\fc}{\right]}
\newcommand{\op}{\left(}
\newcommand{\fp}{\right)}

\newcommand{\bi}{\mathbf{i}}
\newcommand{\bj}{\mathbf{j}}
\newcommand{\bk}{\mathbf{k}}
\newcommand{\bF}{\mathbf{F}}

\newcommand{\mR}{\mathbb{R}}

\newcommand{\ra}{\rightarrow}
\newcommand{\Ra}{\Rightarrow}

\newcommand{\sech}{\mathrm{sech}\,}
\newcommand{\csch}{\mathrm{csch}\,}
\newcommand{\curl}{\mathrm{curl}\,}
\newcommand{\dive}{\mathrm{div}\,}

\newcommand{\ve}{\varepsilon}
\newcommand{\spc}{\vspace*{0.5cm}}

\DeclareMathOperator{\Ran}{Ran}
\DeclareMathOperator{\Dom}{Dom}

\newcommand{\exo}[3]{\noindent\textcolor{red}{\fbox{\textbf{Section {#1} | Problem {#2} | {#3} points}}}\\}

\begin{document}
	\noindent \hrulefill \\
	MATH-302 \hfill Pierre-Olivier Paris{\'e}\\
	Homework 8 Solutions \hfill Fall 2022\\\vspace*{-1cm}
	
	\noindent\hrulefill
	
	\spc
	
	\exo{5.5}{1}{10}
	\\
	\underline{\textbf{Find the general solutions to the complementary equation.}}\\
	The complementary equation is
		\begin{align*}
		y'' + 3y' + 2y = 0
		\end{align*}
	The characteristic equation associated to the complementary equation is $r^2 + 3r' + 2 = 0$. Therefore, the roots are $r_1 = -1$ and $r_2 = -2$. The general solution is
		\begin{align*}
		y_h (x) = c_1 e^{-x} + c_2 e^{-2x} .
		\end{align*}
	
	\underline{\textbf{Find a particular solution.}}\\
	The right-hand side contains $\cos x$ and $\sin x$. Those functions are not contained in the complementary equation. We therefore suggest
		\begin{align*}
		y_{par} (x) = A \cos x + B \sin x .
		\end{align*}
	We have
		\begin{align*}
		y' &= -A \sin x + B \cos x \\
		y'' &= -A \cos x - B \sin x .
		\end{align*}
	Replacing in the ODE, we get
		\begin{align*}
		y'' + 3y' + 2y &= -A \cos x - B \sin x - 3A \sin x + 3B \cos x + 2 A \cos x + 2B \sin x \\
		&= (A + 3B ) \cos x + (-3A + B) \sin x .
		\end{align*}
	This last expression should be equal to the right-hand side
		\begin{align*}
		(A + 3B ) \cos x + (-3A + B) \sin x = 7 \cos x - \sin x .
		\end{align*}
	We therefore have to find $A$, $B$ satisfying
		\begin{align*}
		\left\{
		\begin{matrix}
		A + 3B = 7 \\
		-3A + B = -1
		\end{matrix} \right.
		\end{align*}
	The solution is $A = 1$ and $B = 2$. The expression of the particular solution is 
		\begin{align*}
		y_{par} (x) = \cos x + 2\sin x .
		\end{align*}
		
	\underline{\textbf{General solution.}}\\
	Combining $y_h$ and $y_{par}$, we get
		\begin{align*}
		y (x) = y_h (x) + y_{par} (x) = c_1 e^{-x} + c_2e^{-2x}+  \cos x + 2 \sin x .
		\end{align*}
		
	\newpage
		
	\exo{5.5}{3}{10}
	\\
	\underline{\textbf{Find the general solution to the complementary equation.}}\\
	The complementary equation is
		\begin{align*}
		y'' + 2y' + y = 0 .
		\end{align*}
	The characteristic equation associated to the complementary equation is $r^2 + 2r + 1 = 0$. There is only one root: $r_1 = -1$. The solution to the complementary equation is therefore
		\begin{align*}
		y_h (x) = c_1 e^{-x} + c_2 xe^{-x} .
		\end{align*}
		
	\underline{\textbf{Find a particular solution.}}\\
	The right-hand side contains the functions $e^x \cos x$ and $e^x \sin x$. Those are not contained in the solution to the complementary equation. Therefore, we suggest
		\begin{align*}
		y_{par} (x) = e^x (A\cos x + B \sin x) .
		\end{align*}
	We have
		\begin{align*}
		y' (x) &= e^x (A\cos x + B \sin x) + e^x (-A\sin x + B \cos x) \\
		y''(x) &=  e^x (A \cos x + B \sin x) + 2 e^x (-A \sin x + B \cos x) + e^x (-A \cos x - B \sin x ) .
		\end{align*}
	Replacing in the ODE, we get
		\begin{align*}
		y'' + 2y' + y &= e^x (A \cos x + B \sin x) + 2 e^x (-A \sin x + B \cos x) + e^x (-A \cos x - B \sin x) \\
		& \phantom{=} + 2 e^x (A \cos x + B \sin x) + 2 e^x (-A \sin x + B \cos x) + e^x(A\cos x + B \sin x) \\
		&= e^x \big( (3A + 3B) \cos x + (3B - 4A) \sin x \big)
		\end{align*}
	The right-hand side is $e^x (6\cos x + 17 \sin x)$. Therefore, we must have
		\begin{align*}
		3A + 3B = 6 \quad \text{ and } \quad 3B - 4A = 17 .
		\end{align*}
	The solution is $A = -11/7$ and $B = 25/7$. The particular solution is therefore
		\begin{align*}
		y_{par} (x) = -\frac{11e^x}{7} \cos x + \frac{25e^x}{7} \sin x .
		\end{align*}
		
	\underline{\textbf{General solution.}}\\
	The general solution is therefore
		\begin{align*}
		y(x)= y_h (x) + y_{par} (x) = c_1 e^{-x} + c_2 xe^{-x} -\frac{11e^x}{7} \cos x + \frac{25e^x}{7} \sin x  .
		\end{align*}
		
	\newpage
	
	\exo{5.5}{11}{10}
	\\
	
	\underline{\textbf{Complementary Equation.}}\\
	The complementary equation is
		\begin{align*}
		y'' -2y' + 5y = 0 .
		\end{align*}
	The characteristic polynomial associated to the complementary equation is $r^2 - 2r + 5 = 0$. The roots are complex numbers and they are
		\begin{align*}
		r_1 = 1 + 2i \quad \text{ and } \quad r_2 = 1 - 2i .
		\end{align*} 
	Therefore, the solution is
		\begin{align*}
		y_h (x) = c_1 e^x \cos (2x) + c_2 e^x \sin (2x) .
		\end{align*}
		
	\underline{\textbf{Find a particular solution.}}\\
	The right-hand side unfortunately contains $e^x \cos (2x)$ and $e^x \sin (2x)$. These functions are also multiplied by a polynomial of degree $1$. We therefore suggest 
		\begin{align*}
		y_{par} (x) = xe^x \big( (Ax + B) \cos (2x) + (Cx + D) \sin (2x) \big) .
		\end{align*}
	Following the hint, we have
		\begin{align*}
		A = 1 , \, B = -1 \, , C = 1 \text{ and } D = 1 .
		\end{align*}
	Therefore, the solution is
		\begin{align*}
		y_{par} (x) = xe^x \big( (x - 1) \cos (2x) + (x + 1) \sin (2x) \big) .
		\end{align*}
	%	\begin{align*}
	%	y' &= xe^x (-2 \sin(2 x) (A x + B) + A \cos(2 x) + 2 \cos(2 x) (C x + D) + C \sin(2 x)) \\
	%	&\phantom{=} + e^x (\cos(2 x) (A x + B) + \sin(2 x) (C x + D)) + xe^x (\cos(2 x) (A x + B) + \sin(2 x) (C x + D)) \\
	%	&= xe^x \Big( (A + 2C)x + (A + B + 2D) \big) \cos (2x)
	%	\end{align*}
	
	\underline{\textbf{General solution.}}\\
	Combining $y_h$ and $y_{par}$, we obtain
		\begin{align*}
		y(x) = y_h (x) + y_{par} (x) = c_1 e^x \cos (2x) + c_2 e^x \sin (2x) + xe^x \big( (x - 1) \cos (2x) + (x + 1) \sin (2x) \big).
		\end{align*}
		
	\newpage
	
	\exo{5.5}{23}{15}
	\\
	\underline{\textbf{Complementary Equation.}}\\
	The complementary equation is
		\begin{align*}
		y'' -2y' + 2y = 0 .
		\end{align*}
	The characteristic polynomial associated to the complementary equation is $r^2 - 2r + 2 = 0$. The roots are complex and they are
		\begin{align*}
		r_1 = 1 + i \quad \text{ and } \quad r_2 = 1 - i .
		\end{align*} 
	Therefore, the solution is
		\begin{align*}
		y_h (x) = c_1 e^x \cos (x) + c_2 e^x \sin (x) .
		\end{align*}
		
	\underline{\textbf{Find a particular solution.}}\\
	The right-hand side unfortunately contains $e^x \cos (x)$ and $e^x \sin (x)$. Since these functions are only multiplied by constants, we therefore suggest 
		\begin{align*}
		y_{par} (x) = xe^x \big( A \cos (x) + B \sin (x) \big) .
		\end{align*}
	We find that
		\begin{align*}
		y' (x) &= e^x \big( A \cos x + B \sin x \big) + xe^x \big( A \cos x + B \sin x \big) + xe^x \big( -A \sin x + B \cos x \big) \\
		&= e^x \Big( \big( (A + B)x + A \big) \cos (x) + \big( (B-A)x + B \big) \sin (x) \Big)
		\end{align*}
	and
		\begin{align*}
		y'' (x) &= e^x \Big( \big( (A + B)x + A \big) \cos (x) + \big( (B-A)x + B \big) \sin (x) \Big) \\
		& \phantom{=} + e^x \Big( (A + B) \cos (x) - \big( (A + B) x + A \big) \sin x + (B - A) \sin (x) + \big( (B-A)x + B \big) \cos (x) \Big) \\
		&= e^x \Big( \big( 2Bx + 2(A + B) \big) \cos (x) + \big( -2A x + 2 (B - A) \big) \sin (x) \Big) 
		\end{align*}
	We replace in the left-hand side of the ODE to get
		\begin{align*}
		y'' - 2y' + 2y &= e^x \Big( 2B \cos (x) - 2A \sin (x) \Big) .
		\end{align*}
	The right-hand side is $e^x (-6\cos x - 4 \sin x )$. Therefore, we must have
		\begin{align*}
		2B = -6 \quad \text{ and } \quad -2A = -4 .
		\end{align*}
	We conclude that $B = -3$ and $A = 2$ and the solution is
		\begin{align*}
		y_{par} (x) = xe^x \big( 2 \cos x - 3 \sin x \big) .
		\end{align*}
	
	\underline{\textbf{General solution.}}\\
	Combining $y_h$ and $y_{par}$, we obtain
		\begin{align*}
		y(x) = y_h (x) + y_{par} (x) = c_1 e^x \cos (x) + c_2 e^x \sin (x) + xe^x \big( 2 \cos x - 3 \sin x \big) .
		\end{align*}
		
	\underline{\textbf{Initial Value Problem.}}\\
	We have $y(0) = 1$, so
		\begin{align*}
		c_1 = 1 .
		\end{align*}
	The expression of the derivative of the particular solution was computed in the second step. Replacing $A$ and $B$ by their values, we have
		\begin{align*}
		y' (x) = c_1 e^x \cos x + c_2 e^x \sin x - c_1 e^x \sin x + c_2 e^x \cos x + e^x \Big( \big( 2 - x \big) \cos x + \big( -5x - 3 \big) \sin x \Big) .
		\end{align*}	
	Since $y'(0) = 4$, we obtain
		\begin{align*}
		c_1 + c_2 + 2 = 4 \iff c_1 + c_2 = 2 .
		\end{align*}
	Solving for $c_1$ and $c_2$, we get
		\begin{align*}
		c_2 = 1 .
		\end{align*}
	Therefore, the solution to the IVP is
		\begin{align*}
		y(x) =  e^x \cos (x) +  e^x \sin (x) + xe^x \big( 2 \cos x - 3 \sin x \big) .
		\end{align*}
		
	\newpage
	
	\exo{5.6}{3}{5}
	\\
	
	We set 
		$$
		y = u y_1 = ux .
		$$
	Then the derivative and second derivative are
		\begin{align*}
		y' &= u' x + u \\
		y'' &= u'' x + 2u' .
		\end{align*}
	Subtituting in the ODE, we obtain
		\begin{align*}
		x^2 (u'' x + 2u') - x (u'x + u) + ux = x
		\end{align*}
	This simplifies to
		\begin{align*}
		x^3 u'' + x^2 u' = x .
		\end{align*}
	Dividing through $x$, we get
		\begin{align*}
		x^2 u'' + x u' = 1 .
		\end{align*}
		
	Set $v = u'$ so that $v' = u''$. Therefore, the ODE is reduced to
		\begin{align*}
		x^2 v' + x v = 1 .
		\end{align*}
	Dividing again by $x$, we get
		\begin{align*}
		x v' + v = \frac{1}{x} .
		\end{align*}
	The left-hand side can be rewritten as
		\begin{align*}
		\frac{d}{dx} \big( xv \big) = \frac{1}{x} .
		\end{align*}
	Integrating with respect to $x$, we obtain
		\begin{align*}
		xv = \ln|x| + c_1 \quad \Ra \quad v(x) = \frac{\ln|x|}{x} + \frac{c_1}{x} .
		\end{align*}
	Since $v(x) = u' (x)$, integrating again with respect to $x$ the expression of $v$, we obtain
		\begin{align*}
		u(x) = \frac{\op \ln |x| \fp^2}{2} + c_1 \ln|x| + c_2 .
		\end{align*}
		
	Replacing in $y(x)$, we conclude that the general solution is
		\begin{align*}
		y(x) = \frac{x (\ln |x|)^2}{2} + c_1 x \ln|x| + c_2 x .
		\end{align*}
	From this, we see that a foundamental set of solution for the complementary equation is $\{ x  , x \ln |x| \}$.
	
	\vfill
	
	\hfill \textcolor{red}{\textsc{Total (Points): 50.}}
	
	
	
\end{document}